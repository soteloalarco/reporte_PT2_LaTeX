% Chapter 1

\chapter{Introducción} % Main chapter title

\label{Chapter1} % Change X to a consecutive number; for referencing this chapter elsewhere, use \ref{ChapterX}

%----------------------------------------------------------------------------------------

% Define some commands to keep the formatting separated from the content 
\newcommand{\keyword}[1]{\textbf{#1}}
\newcommand{\tabhead}[1]{\textbf{#1}}
\newcommand{\code}[1]{\texttt{#1}}
\newcommand{\file}[1]{\texttt{\bfseries#1}}
\newcommand{\option}[1]{\texttt{\itshape#1}}

%----------------------------------------------------------------------------------------
%	SECTION 1
%----------------------------------------------------------------------------------------
\myworries{TODO: FALTA ACTUALIZAR TODO EL CAPITULO 1}\newline

\section{ANTECEDENTES}

Imaginar nuestra vida sin los beneficios brindados por los sistemas de comunicación de hoy en día, o tan sólo sin la tecnología presente en este ámbito desde los últimos 20 años es ya muy difícil, y esto se debe a que en el presente una gran parte de las tareas y actividades, muchas de ellas cruciales para el funcionamiento de nuestras sociedades, operan eficientemente sí y sólo sí se está propiamente conectado y en facultades de compartir información \parencite{Fettweis2014}.\newline

Los sistemas de comunicación celular han tenido saltos generacionales desde la conocida como primera generación, la cual saldría al mercado a finales de la década de los 70’s e inicio de los 80’s, hasta el presente con el desarrollo de la próxima generación (5G), la cual comenzará su implementación en el año 2020. En \parencite{Fettweis2014} encontramos que:\newline

\textit{“La primera y segunda generación de comunicaciones móviles estuvieron dominadas por señales analógicas de audio y posteriormente señales digitales de audio y texto. La tercera generación se trató más de escalar el número de usuarios en la red […] pero fue abrumada por un tsunami de contenido de imágenes y videos.”}\newline

Cada uno de estos saltos de generación ha estado motivado por distintos requerimientos de servicio, necesidades de los usuarios y la aparición de nuevas tecnologías que han buscado ser una vía para facilitar la comunicación entre individuos de todo el mundo y ahora, más recientemente, la comunicación entre máquinas.\newline

El aumento de la tasa de transmisión de datos ha sido siempre un factor a tener en cuenta para el desarrollo de los estándares de las nuevas generaciones de redes móviles, por ejemplo, para la nueva generación se espera “un pico de transmisión de al menos 1 Gb/s al tiempo de su introducción en 2020, esperando que crezca hasta los 10Gb/s para 2025”, \parencite{Fettweis2016}. Pero el sistema de comunicaciones móviles de quinta generación ha estado, además de eso, motivado por un mayor volumen de transmisión de datos, un incremento radical en la cantidad de dispositivos conectados a la red, una menor latencia y una mayor duración de batería para los dispositivos de bajo consumo.\newline

Las limitaciones presentes hasta ahora para las comunicaciones IoT celulares, se deben principalmente a que la red de comunicación móvil fue creada para voz y aplicaciones de texto, evolucionando eventualmente a una transmisión de archivos como imágenes y videos predominante en el enlace de bajada. Por otro lado la comunicación de dispositivos IoT tiene un conjunto de requerimientos muy distintos. Pero es ahora con la quinta generación que se promete brindar las herramientas que esta tecnología necesita para alcanzar su máximo potencial. 5G se trata entonces de la propuesta de crear una red de comunicaciones que logre implementar tanto los servicios inherentes a las necesidades de comunicación entre los humanos y aquellas necesidades de comunicación entre las máquinas. El cumplir con estas últimas necesidades, aseguraría brindar una calidad de servicio óptima para la nueva ola de dispositivos de IoT que se espera estén ya conectados a la red para 2020.\newline

Este proyecto presenta el diseño de un simulador de eventos discretos, el cual modeló el servicio prestado por la arquitectura de red celular que aquí se propone a nodos IoT. Su arquitectura contempló la próxima generación móvil a implementarse (5G) y los servicios seleccionados para atender a aplicaciones del caso de uso mIoT. Este simulador se enfocó en el tráfico generado por dispositivos NB-IoT y con los resultados obtenidos se esclareció sobre qué configuraciones de red son ideales para conseguir una óptima calidad de servicio.\newline

%----------------------------------------------------------------------------------------
%	SECTION 2
%----------------------------------------------------------------------------------------

\section{PLANTEAMIENTO DEL PROBLEMA}

En los recientes años se ha estado presenciando la definición de la tecnología de comunicación móvil 5G en estándares, para su posterior introducción a partir de 2020, y como se resalta en \parencite{Fettweis2016} la nueva generación no sólo seguirá la línea de incrementar la velocidad de transmisión como se ha venido haciendo en cada salto generacional, sino que también traerá consigo la posibilidad de una conectividad adicional sin precedentes, todo esto motivado por la cantidad masiva de dispositivos de IoT que se esperan. Según \textit{Ericsson Mobility Report} \parencite{Ericsson2019}, 22.3 mil millones de dispositivos en el 2024, pertenecerán a una aplicación de IoT. De manera que 5G dará servicio a una enorme cantidad de dispositivos IoT, cada dispositivo enviará pequeños paquetes de datos a lo largo de mucho tiempo, además de que existirá una cierta sincronía en el tráfico generado. Lo anterior ha provocado el desarrollo de nuevas tecnologías que proponen distintas formas de agrupamiento de estos nodos, distintas formas de acceder a los recursos, distintas propuestas de comunicación entre ellos y nuevas formas de que ahorren energía. Siempre teniendo en cuenta los KPIs de la red, como dar servicio a una cantidad masiva de nodos, la duración de la batería de estos y la menor latencia en comunicaciones críticas \footnote{\ Comunicaciones\ que\ requieren\ de\ una\ urgente\ respuesta\ debido\ a\ su\ naturaleza,\ por\ ejemplo\ los\ coches\ auto-dirigidos.} , como se menciona en \parencite{NGMN}.\newline

Esto ha resultado en nuevos retos para la implementación de la red, para la cual se desarrolla tecnología o se mejora la ya existente. En \parencite{GSMAssociation2019}, se presenta la tecnología NB-IoT (\textit{Narrow-Band IoT}), originalmente creada como una solución que brindara servicio a nodos IoT en LTE. Esta tecnología formará parte de los estándares de 5G, como 3GPP\footnote{\ $  $The\ 3rd\ Generation\ Partnership\ Project\ (3GPP),\ desarrolladores\ del\ est\textrm{\'{a}}ndar\ NB-IoT\ y\ el\ 5G\ NR}
lo ha indicado a la ITU (\textit{International Telecommunication Union}). Se pretende que con esta tecnología y algunas mejoras, la red 5G sea capaz de brindar servicio a aplicaciones del caso de uso mMTC\footnote{Cabe mencionar que los casos de uso o tecnologías mIoT y mMTC son análogas.}, para el cual se esperan tener decenas de miles de dispositivos conectados por celda. \newline

En el futuro, los escenarios de IoT, tendrán una enorme cantidad de dispositivos conectados en comparación con los actuales escenarios de la red 4G \parencite{Whatis5G}. Por lo tanto, las tecnologías de 5G deberán brindar servicio a muchos dispositivos usando recursos limitados. Pero dada la elevada complejidad con la que el modelo de un sistema de comunicación como la red 5G puede contar, si de éste se quieren obtener resultados útiles, resulta casi imposible el analizar su comportamiento sino a través de una simulación.\newline

Los patrones de tráfico de nodos de IoT varían según su caso de uso, los cuales se dividen normalmente en tres: \textit{eMBB, uRLLC y mMTC}. Para mMTC también llamado mIoT, en el cual se enfocará el proyecto, se tienen nodos en su amplia mayoría estáticos y como se ve en \parencite{IoTTrafficHossfeld}, para estos nodos podemos considerar por los menos dos patrones de tráfico: el periódico, y el aleatorio ,de manera que el modelo de tráfico deberá tener en consideración esto. Como se puede apreciar en \parencite{IoTTrafficHossfeld}, un problema crítico del Internet de las cosas masivo (mIoT) en las redes móviles es que los dispositivos de IoT causarán una gran congestión en esta si es que no se incorporan mejoras en las arquitecturas de estas redes. Este problema se ve acrecentado debido a que el tráfico de los nodos mIoT presenta cierta sincronía espacial y temporal dependiendo de la aplicación a la que pertenezcan.\newline

Por lo anterior, para la red 5G se necesitan realizar simulaciones en sus distintos casos de uso, que generen resultados sobre qué arquitectura de red y tecnologías brindan un resultado óptimo. Con la realización de este proyecto se pretende aportar al campo de las comunicaciones móviles de quinta generación, que está específicamente interesado en el servicio prestado a los nodos de IoT, de una herramienta de simulación que genere resultados que permitan realizar comparaciones entre distintas configuraciones de red. Todo esto con el fin de que la red 5G próxima a ser desplegada, cumpla con una óptima calidad de servicio para el caso de uso mIoT.\newline

%----------------------------------------------------------------------------------------
%	SECTION 3
%----------------------------------------------------------------------------------------

\section{OBJETIVOS}
\subsection{OBJETIVO GENERAL}

Diseñar e implementar un simulador de teletráfico para el ambiente de Internet de las cosas masivo (mIoT) en una red celular de quinta generación (5G), por medio de la programación de eventos discretos, con la finalidad de evaluar el desempeño de esta red en términos de la cantidad de recursos requeridos para satisfacer niveles esperados de calidad de servicio (QoS).\newline

\subsection{OBJETIVOS ESPECÍFICOS}

\begin{itemize}
    \item Determinar el escenario a implementar, mediante el análisis de requerimientos, para definir después los parámetros de entrada en el modelado del simulador.
    \item Seleccionar los modelos (despliegue, propagación, técnica de acceso múltiple) aplicables a una simulación a nivel de sistema para la comunicación entre nodos IoT y la red 5G.
    \item Seleccionar y determinar los modelos de tráfico a analizar para nodos IoT que hayan sido propuestos en la red 5G, mediante la lectura de distintas publicaciones en revistas científicas, para simular el modelo más adecuado según los alcances propuestos.
    \item Determinar los parámetros de desempeño de la red (KPIs), mediante el análisis de distintas publicaciones científicas, con la finalidad de establecer métricas de QoS.
    \item Definir el procedimiento de la simulación mediante la especificación de su arquitectura y elaboración de diagramas de su funcionamiento y procesos, con el fin de integrar una metodología para su implementación.
    \item Implementar los modelos y protocolos que definen a la comunicación entre los nodos IoT y la red 5G, mediante algoritmos computacionales (incluyendo la técnica de eventos discretos) y de acuerdo a la arquitectura previamente definida.
    \item Implementar una técnica de paralelismo, mediante el uso de multiprocesamiento, con la finalidad de reducir los tiempos de ejecución de las simulaciones.
    \item Evaluar y analizar cada modelo analítico con el uso de escenarios de pruebas y calibración para poder realizar comparaciones con los resultados teóricos esperados y verificar la fiabilidad del simulador 
    \item Simular el modelo de sistema propuesto, mediante la variación de los parámetros de entrada, para caracterizar el desempeño del sistema en términos del tráfico que se puede ofrecer y la cantidad de recursos requeridos para satisfacer objetivos de QoS.
\end{itemize}

%----------------------------------------------------------------------------------------
%	SECTION 4
%----------------------------------------------------------------------------------------

\section{JUSTIFICACIÓN}

En la gran mayoría de los trabajos de investigación revisados, se han realizado estudios de rendimiento de los sistemas de comunicación móvil de quinta generación. Estos se concentran en evaluar distintos modelos, frecuentemente se considera, un modelo de distribución de BSs y UEs, uno de canal, y un esquema de acceso multiple, sin embargo no hay mucha investigación acerca de incorporar todos estos componentes en conjunto con modelos de tráfico para el caso específico de mIoT. La aportación que se pretende hacer con este proyecto se encuentra precisamente en ese ámbito.\newline

Los enlaces de comunicación inalámbrica experimentan fenómenos físicos perjudiciales al canal como lo son las multi-trayectorias y los desvanecimientos debido a grandes objetos que se interceptan en la trayectoria de la propagación, además el rendimiento de los sistemas celulares inalámbricos tiende a limitarse debido a la interferencia de otros usuarios. Estas condiciones complejas del canal son difíciles de describir con un simple modelo analítico, es por esto que las aproximaciones de las simulaciones son necesarias. Estas pueden analizar el rendimiento de los enlaces de comunicaciones celulares \parencite{WirelessSim}, modelando un gran número de eventos aleatorios a través del tiempo, mediante el uso de simulaciones orientadas a eventos discretos. \newline

Una simulación permite observar muchas de las interacciones de un sistema, que de otra forma tomaría mucho trabajo predecirlas o calcular, además de que proporciona un método importante de análisis, que resulta sencillo de comunicar y comprender. En todas las industrias y disciplinas, la creación de simulaciones brinda soluciones valiosas al proporcionar información clara sobre sistemas complejos \parencite{WirelessSim}. Los resultados de una simulación que haga las suposiciones adecuadas y modele correctamente el sistema propuesto, brindarán confianza y claridad, ahorrarán tiempo y muy posiblemente también dinero.\newline

Este proyecto servirá como referencia a los investigadores y estudiantes que busquen comparar los modelos y técnicas propuestos en este trabajo con otras selecciones posibles para la futura red 5G y el servicio brindado a los nodos IoT.\newline

Por otra parte, se debe enfatizar que el desarrollo de este proyecto requiere de la aplicación de conocimientos relacionados a la informática (entre los que se incluyen desarrollo de software y algoritmos computacionales), así como del dominio de conceptos propios de las telecomunicaciones (por ejemplo, análisis de tráfico y caracterización de enlaces inalámbricos). De acuerdo a lo anterior, se considera que este proyecto pertenece al campo de aplicación de la ingeniería telemática. Además, se debe notar que, si bien los fenómenos simulados corresponden al proceso de transmisión de información, dichos fenómenos están siendo analizados en el contexto de un sistema con características telemáticas (nodos mIoT conectados a la red 5G).\newline

%----------------------------------------------------------------------------------------
%	SECTION 5
%----------------------------------------------------------------------------------------

\section{PROPUESTA DE SOLUCIÓN}

Considerando lo expuesto anteriormente, se propone desarrollar un simulador a nivel sistema que será programado bajo el paradigma de eventos discreto. La elección de una simulación a nivel de sistema se deriva del enfoque que tendrá nuestro proyecto hacia los distintos tipos de tráfico de nodos NB-IoT y la simulación de cada uno de ellos como fuente de tráfico. Las simulaciones a nivel de sistema permiten modelar el comportamiento de múltiples radio bases, múltiples nodos como fuentes de tráfico, la propagación de las señales y la interferencia que estas causan, a la vez que se realizan abstracciones más simples de lo que sucede más allá de estas interacciones. Esto facilita la implementación de una gran cantidad de actores. La generación de variables aleatorias vendrá de la mano de las distintas aplicaciones mIoT y sus patrones de transmisión estocásticos, además de la localización de  nodos en un plano la cual no será uniforme.\newline

El simulador será capaz de evaluar la calidad del servicio que la red celular propuesta ha de brindar a nodos de mIoT. Dicha arquitectura de red, propuesta en este mismo trabajo está basada a su vez en los avances hechos, por grupos como 3GPP, hacia el despliegue de la red 5G. La base de la que se partió es la tecnología NB-IoT, la cual abordó el caso de uso mMTC en la red 4G donde ha estado prestando servicio a nodos mIoT a una escala menor que la esperada en 5G.\newline

Se modelará el servicio brindado a nodos estáticos de IoT cuyas aplicaciones pertenecen al caso de uso mIoT. Con la ayuda del estándar NB-IoT, para el que se proponen mejoras en el acceso múltiple, esto en búsqueda de cumplir con los KPI’s de la red 5G, ya que si bien NB-IoT pertenecerá al paradigma de 5G, no es viable tál y como existe ahora para cumplir con los requerimientos. La importancia de esto recae, como se menciona en \parencite{EjazIoT}, en que la tecnología de IoT ha creado una revolución en la última década con la creación de aplicaciones pensadas alrededor de todo tipo de sensores, lo que resulta en una proyección estimada de miles de millones de dispositivos IoT para el 2020 [3]. Esta misma referencia asegura que IoT está tomando un papel principal en el desarrollo de la quinta generación, debido a que se espera que los dispositivos de IoT formen la gran mayoría de dispositivos en esta nueva generación que se avecina.\newline

Se propone un análisis fundamental principalmente del modelado de cuatro componentes que son esenciales para la caracterización de un sistema de comunicación móvil. Estos componentes corresponden al: modelado de despliegue de usuaios, canal, tráfico y un esquema de acceso múltiple al medio. Esto se encontrará en el capítulo 4, que comprende un análisis de forma detallada de estos, pero a continuación se abordan de tal forma que se esclarezca la arquitectura del modelo de sistema, presentada también más adelante.\newline

Se considerará un modelo de despliegue de nodos IoT que seguirá un proceso puntual de Poisson con el fin de crear una geometría estocástica, se representarán las pérdidas por medio de un modelo de canal estadístico para ambientes celulares de quinta generación, se considerará en la simulación un modelo de tráfico fuente en el que cado nodo mIoT generará tráfico ya sea periódico o aleatorio, cada caso con distintas tasas y por último, referente al método de acceso múltiple, se aplicará una mejora a la tecnología NB-IoT, se trata de la implementación del esquema NOMA en el dominio de la potencia, de forma que agrupamientos (de longitud fija) de nodos estarán compartiendo un mismo recurso (una sub-banda).\newline

Este análisis conllevará en conjuntos de la red 5G y los dispositivos NB-IoT en un ambiente masivo, conseguirá resultados que podrán brindar una base fundamental para evaluar el desempeño de estas redes y por supuesto, su dimensionamiento en términos de objetivos de QoS.\newline

%----------------------------------------------------------------------------------------
%	SECTION 6
%----------------------------------------------------------------------------------------

\section{ALCANCES}

Se obtendrán resultados que permitan analizar las configuraciones de la arquitectura de red propuesta que conllevan a una óptima calidad de servicio. Teniendo como métricas principales la densidad de usuarios soportada y la tasa de transmisión máxima alcanzada. Estos resultados reflejarán a su vez las ventajas que puede traer la selección de cierta arquitectura de red y su despliegue. Es aquí donde se encuentra una de las ventajas de realizar un simulador, ya que con la ayuda de múltiples computadoras, se podrán simular miles de nodos mIoT en esta red. Las configuraciones y parámetros de la red podrán modificarse al inicio de cada simulación, y podrán ser inspeccionados mientras esta corre. La variación de estos parámetros a lo largo de múltiples simulaciones permitirá generar tablas y gráficas de los resultados obtenidos.\newline

Este proyecto no cubrirá los aspectos de movilidad entre celdas para los nodos de IoT, ya que el caso de uso mMTC representa a los nodos estáticos en su mayoría o con velocidades menores a 3Km/h.  No se desarrollarán nuevos modelos probabilísticos o matemáticos de ninguna clase, sino que se implementarán los existentes para el escenario propuesto.\newline
