% Chapter 6

\chapter{Implementación} % Main chapter title

\label{Chapter6} % Change X to a consecutive number; for referencing this chapter elsewhere, use \ref{ChapterX}

En este Captitulo se explica la implementación del modelo de sistema propuesto en el Capítulo \ref{Chapter5}
\myworries{AQUI VA TODO LO REFERENTE A LO QUE HEMOS HECHO EN PT2}\newline

%----------------------------------------------------------------------------------------
%	SECTION 
%----------------------------------------------------------------------------------------

\section{Generación de una geometría estocastica usando Procesos Puntuales de Poisson (PPP)}

\subsection{Validación de Procesos Puntuales de Poisson (PPP)}


%----------------------------------------------------------------------------------------
%	SECTION 
%----------------------------------------------------------------------------------------

\section{Generación de coeficientes de canal con desvanecimiento Rayleigh}

\subsection{Validación de desvanecimiento Rayleigh}


%----------------------------------------------------------------------------------------
%	SECTION 
%----------------------------------------------------------------------------------------

\section{Esquema de acceso múltiple al medio no ortogonal, basado en potencia (PD-NOMA)}
\subsection{Algoritmo de Agrupación SIC}

\subsection{Validación de Algoritmo}

\subsection{Algoritmo de Asignacion de Subportadoras}

\subsection{Validación de Algoritmo}

%%%%%%%%%%%%%%%%%%%%%%%%%%%%%%%%%%%%%%%%%%%%%%%%%%%%%%%%       EJEMPLO DE ALGORITMO EN LATEX       %%%%%%%%%%%%%%%%%%%%%%%%%%%%%%%%%%%%%%%%%%%%%%%%%%%%%%%%%%%%%
\makeatletter
% Reinsert missing \algbackskip
\def\algbackskip{\hskip-\ALG@thistlm}
\makeatother

\begin{algorithm}
    \caption{My algorithm}\label{euclid}
    \hspace*{\algorithmicindent} \textbf{Input} \\
    \hspace*{\algorithmicindent} \textbf{Output} 
    \begin{algorithmic}[1]
    \Procedure{MyProcedure}{}
%    \Procedure{MyProcedure}{$x,y$}
%     % Input:
%     \Comment{Input: x}
%     % Output:
%     \Comment{Output:y}
    \State $\textit{stringlen} \gets \textit{length of } \textit{string}$
    \State $i \gets \textit{patlen}$
    \BState \emph{top}:
    \If {$i > \textit{stringlen}$} \Return false
    \EndIf
    \State $j \gets \textit{patlen}$
    \BState \emph{loop}:
    \If {$\textit{string}(i) = \textit{path}(j)$}
    \State $j \gets j-1$.
    \State $i \gets i-1$.
    \State \textbf{goto} \emph{loop}.
    \State \textbf{close};
    \EndIf
    \State $i \gets i+\max(\textit{delta}_1(\textit{string}(i)),\textit{delta}_2(j))$.
    \State \textbf{goto} \emph{top}.
    \EndProcedure
    \end{algorithmic}
\end{algorithm}

%----------------------------------------------------------------------------------------
%	SECTION 
%----------------------------------------------------------------------------------------

\section{Generación de Tráfico Fuente}

\subsection{Tráfico CMMPP}
\subsection{Validación de Tráfico CMMPP}

%----------------------------------------------------------------------------------------
%	SECTION 
%----------------------------------------------------------------------------------------

\section{Interconexión de los 4 módulos del Simulador}
%Incluir lo de NPRACHH, NPUCSH y NORA

%----------------------------------------------------------------------------------------
%	SECTION 
%----------------------------------------------------------------------------------------

\section{Simulador de Eventos Discretos}

\subsection{Definición de eventos}

\subsection{Interfaz de usuario}

\subsection{Descripción de los \textit{logs} de salida}

%----------------------------------------------------------------------------------------
%	SECTION 
%----------------------------------------------------------------------------------------

\section{Optimización de tiempos de simulación}

