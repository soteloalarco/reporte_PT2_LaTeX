%%%%%%%%%%%%%%%%%%%%%%%%%%%%%%%%%%%%%%%%%
% Masters/Doctoral Thesis 
% LaTeX Template
% Version 2.5 (27/8/17)
%
% This template was downloaded from:
% http://www.LaTeXTemplates.com
%
% Version 2.x major modifications by:
% Vel (vel@latextemplates.com)
%
% This template is based on a template by:
% Steve Gunn (http://users.ecs.soton.ac.uk/srg/softwaretools/document/templates/)
% Sunil Patel (http://www.sunilpatel.co.uk/thesis-template/)
%
% Template license:
% CC BY-NC-SA 3.0 (http://creativecommons.org/licenses/by-nc-sa/3.0/)
%
%%%%%%%%%%%%%%%%%%%%%%%%%%%%%%%%%%%%%%%%%
%La plantilla base fue utilizada y modificada por Fernando Salazar y Rolando Sotelo
%----------------------------------------------------------------------------------------
%	PACKAGES AND OTHER DOCUMENT CONFIGURATIONS
%----------------------------------------------------------------------------------------

\documentclass[
	12pt, % The default document font size, options: 10pt, 11pt, 12pt
	%oneside, % Two side (alternating margins) for binding by default, uncomment to switch to one side
	spanish, % ngerman for German
	es-tabla,
	singlespacing, % Single line spacing, alternatives: onehalfspacing or doublespacing
	%draft, % Uncomment to enable draft mode (no pictures, no links, overfull hboxes indicated)
	%nolistspacing, % If the document is onehalfspacing or doublespacing, uncomment this to set spacing in lists to single
	%liststotoc, % Uncomment to add the list of figures/tables/etc to the table of contents
	%toctotoc, % Uncomment to add the main table of contents to the table of contents
	%parskip, % Uncomment to add space between paragraphs
	%nohyperref, % Uncomment to not load the hyperref package
	headsepline, % Uncomment to get a line under the header
	%chapterinoneline, % Uncomment to place the chapter title next to the number on one line
	%consistentlayout, % Uncomment to change the layout of the declaration, abstract and acknowledgements pages to match the default layout
	]{MastersDoctoralThesis} % The class file specifying the document structure
	
% Paquetes:
\usepackage[utf8]{inputenc} % Required for inputting international characters
\usepackage[T1]{fontenc} % Output font encoding for international characters
\usepackage{mathpazo} % Use the Palatino font by default
\usepackage[backend=biber,style=ieee,natbib=true]{biblatex} % Para Formato IEEE
\addbibresource{Referencias.bib} % The filename of the bibliography
\usepackage[autostyle=true]{csquotes} % Required to generate language-dependent quotes in the bibliography
\usepackage{xcolor}
\usepackage{lscape} %Para poner páginas horizontales
\usepackage{booktabs}
\usepackage{multirow}
\usepackage{rotating}
\usepackage{algorithm}
\usepackage[T1]{fontenc} % Codificación de fuente
\usepackage{lmodern} % Fuente compatible
\usepackage[noend]{algpseudocode}%Para incluir algoritmos
\usepackage{amssymb}
\usepackage{amsmath}
\renewcommand{\spanishtablename}{Tabla}
\renewcommand{\spanishlisttablename}{Índice de tablas}
%\renewcommand{\acknowledgmentsname}{Agradecimientos}
%\addto\captionsspanish{\renewcommand{\acknowledgmentsname}{Agrade}}
\newcommand\myworries[1]{\textcolor{red}{#1}} %Para incorporar TODOs en el documento (en color rojo)
\setlength\parindent{0pt} %Para evitar la sangría default que implementa la clase

%----------------------------------------------------------------------------------------
%	MARGIN SETTINGS
%----------------------------------------------------------------------------------------

\geometry{
	paper=letterpaper, % Change to letterpaper for US letter
	inner=2cm, % Inner margin
	outer=2cm, % Outer margin
	bindingoffset=.5cm, % Binding offset
	top=1.5cm, % Top margin
	bottom=1.5cm, % Bottom margin
	%showframe, % Uncomment to show how the type block is set on the page
}

%----------------------------------------------------------------------------------------
%	THESIS INFORMATION
%----------------------------------------------------------------------------------------

\thesistitle{SIMULADOR DE MODELOS DE TRÁFICO PARA NODOS IOT EN UNA RED CELULAR DE 5G} % Your thesis title, this is used in the title and abstract, print it elsewhere with \ttitle
%\supervisor{} % Your supervisor's name, this is used in the title page, print it elsewhere with \supname
%\examiner{} % Your examiner's name, this is not currently used anywhere in the template, print it elsewhere with \examname
\degree{Ingeniería en Telemática} % Your degree name, this is used in the title page and abstract, print it elsewhere with \degreename
\author{} % Your name, this is used in the title page and abstract, print it elsewhere with \authorname 
\addresses{} % Your address, this is not currently used anywhere in the template, print it elsewhere with \addressname

\subject{Comunicaciones Móviles} % Your subject area, this is not currently used anywhere in the template, print it elsewhere with \subjectname
\keywords{mMTC, NB-IoT, PD-NOMA, 5G, simulador de eventos discretos.} % Keywords for your thesis, this is not currently used anywhere in the template, print it elsewhere with \keywordnames
\university{\href{https://www.upiita.ipn.mx/}{INSTITUTO POLITÉCTINO NACIONAL - Unidad Profesional interdisciplinaria de Ingeniería y Tecnologías Avanzadas}} % Your university's name and URL, this is used in the title page and abstract, print it elsewhere with \univname
\department{\href{}{}} % Your department's name and URL, this is used in the title page and abstract, print it elsewhere with \deptname
\group{\href{}{}} % Your research group's name and URL, this is used in the title page, print it elsewhere with \groupname
\faculty{\href{}{}} % Your faculty's name and URL, this is used in the title page and abstract, print it elsewhere with \facname

\AtBeginDocument{
\hypersetup{pdftitle=\ttitle} % Set the PDF's title to your title
%\hypersetup{pdfauthor=\authorname} % Set the PDF's author to your name
\hypersetup{pdfkeywords=\keywordnames} % Set the PDF's keywords to your keywords
}

\begin{document}
%\bibliographystyle{IEEEtran}
\frontmatter % Use roman page numbering style (i, ii, iii, iv...) for the pre-content pages

\pagestyle{plain} % Default to the plain heading style until the thesis style is called for the body content

%----------------------------------------------------------------------------------------
%	TITLE PAGE
%----------------------------------------------------------------------------------------

\begin{titlepage}
\begin{center}

%\vspace*{.06\textheight}
{\scshape\LARGE \univname\par}\vspace{1cm} % University name
\includegraphics[scale=.8]{LogoUPIITA}\\ % University/department logo - uncomment to place it
\vspace*{.03\textheight}
\textsc{\Large Proyecto Terminal:}\\[0.5cm] % Thesis type

\HRule \\[0.3cm] % Horizontal line
{\huge \bfseries \ttitle\par}\vspace{0.4cm} % Thesis title
\HRule \\[0.3cm] % Horizontal line

\begin{minipage}[t]{0.45\textwidth}
\begin{flushleft} \large
\emph{Autores:}\\
Luis Fernando \textsc{Salazar Ordoñez} \newline
Rolando \textsc{Sotelo Alarcon} \newline
\end{flushleft}
\end{minipage}
\begin{minipage}[t]{0.45\textwidth}
\begin{flushright} \large
\emph{Asesores:} \\
Dr. Domingo \textsc{Lara Rodriguez} \newline
Dr. Noé \textsc{Torres Cruz}
\end{flushright}
\end{minipage}\\[.1cm]
\vfill
\large \textit{Una tesis presentada en cumplimiento de los requisitos \\para el grado de \degreename}\\[0.2cm] % University requirement text
%\textit{in the}\\[0.4cm]
%\groupname\\\deptname\\[2cm] % Research group name and department name
\vfill
{\large CDMX, Septiembre 2020}\\[4cm] % Date
\vfill
\end{center}
\end{titlepage}

%----------------------------------------------------------------------------------------
%	DECLARATION PAGE
%----------------------------------------------------------------------------------------

%\begin{declaration}
%\addchaptertocentry{\authorshipname} % Add the declaration to the table of contents
%\noindent I, \authorname, declare that this thesis titled, \enquote{\ttitle} and the work presented in it are my own. I confirm that:

%\begin{itemize} 
%\item This work was done wholly or mainly while in candidature for a research degree at this University.
%\item Where any part of this thesis has previously been submitted for a degree or any other qualification at this University or any other institution, this has been clearly stated.
%\item Where I have consulted the published work of others, this is always clearly attributed.
%\item Where I have quoted from the work of others, the source is always given. With the exception of such quotations, this thesis is entirely my own work.
%\item I have acknowledged all main sources of help.
%\item Where the thesis is based on work done by myself jointly with others, I have made clear exactly what was done by others and what I have contributed myself.\\
%\end{itemize}
 
%\noindent Signed:\\
%\rule[0.5em]{25em}{0.5pt} % This prints a line for the signature
 
%\noindent Date:\\
%\rule[0.5em]{25em}{0.5pt} % This prints a line to write the date
%\end{declaration}

%\cleardoublepage

%----------------------------------------------------------------------------------------
%	QUOTATION PAGE
%----------------------------------------------------------------------------------------

%\vspace*{0.2\textheight}

%\noindent\enquote{\itshape Thanks to my solid academic training, today I can write hundreds of words on virtually any topic without possessing a shred of information, which is how I got a good job in journalism.}\bigbreak

%\hfill Dave Barry

%----------------------------------------------------------------------------------------
%	ABSTRACT PAGE
%----------------------------------------------------------------------------------------

\begin{abstract}
	\addchaptertocentry{\abstractname} % Add the abstract to the table of contents
	\addchaptertocentry{Abstract} % Add the abstract to the table of contents

		En este documento se presenta el desarrollo de un simulador a nivel de sistema, programado bajo el paradigma de eventos discretos, que permite modelar el servicio que la red de comunicación celular de quinta generación (5G), ofrece a nodos de Internet de las cosas (IoT). El simulador se enfocó en el caso de uso mIoT, el cual comprende principalmente de nodos IoT estáticos de baja complejidad que además se encuentran en gran cantidad dentro de los escenarios de esta red. La arquitectura del simulador contempló cuatro modelos clave para su ejecución: un modelo de despliegue de UEs, un modelo de canal, un esquema de acceso múltiple al medio no ortogonal y modelos de tráfico adecuados para modelar distintos servicios. Asimismo, se consideró el fundamentar la fiabilidad de los resultados obtenidos por el simulador mediante la previa prueba e implementación de dichos modelos ya estudiados en la literatura concerniente al desempeño de sistemas celulares. Con los resultados del simulador se determinaron qué configuraciones y parámetros iniciales de la arquitectura de red propuesta podrían mejorar significativamente la calidad de servicio (QoS) para el caso de uso mIoT.\newline \\
		Palabras clave: \textbf{\keywordnames} \\
		\begin{center}
			{\huge\textit{Abstract} \par}
			\bigskip
		\end{center}
		\textit{This document presents the development of a system-level simulator, programmed under the discrete events paradigm, which allows modeling the service that the fifth generation cellular communication network (5G) offers to Internet of Things (IoT) nodes. The simulator was focused on the mIoT use case, which mainly comprises low complexity static IoT nodes that are also found in large numbers within the scenarios of this network. The simulator architecture included four key models for its execution: an user equipment deployment model, a channel model, a non-orthogonal multiple access scheme and some traffic models in accordance to different services. The project also contemplates the reliability of the results obtained by the simulator through the previous test and implementation of the models already studied in the literature on the performance of cellular systems. With the simulator results, was determined which initial configurations and parameters of the network architecture proposed, satisfy the expected quality of service (QoS) for the massive IoT use case.}\newline \\
		\textit{Keywords: \textbf{massiveMTC, NarrowBand-IoT, Power-Domain NOMA, 5G, discrete-event simulation (DES)} }\\ 

\end{abstract}

%----------------------------------------------------------------------------------------
%	ACKNOWLEDGEMENTS
%----------------------------------------------------------------------------------------

\begin{acknowledgements}
\addchaptertocentry{\acknowledgementname} % Add the acknowledgements to the table of contents
%Agradecimientos\ldots
\textsc{Luis Fernando:} \\

\textit{Es muy gratificante saber que se termina una etapa y comienzan otras nuevas, siempre me han gustado los retos y este sin duda lo fue.}\newline

\textit{Antes que nada agradezco a mis padres y mis hermanos que me han brindado los principios y valores que me forjan la persona que soy ahora. Gracias por todo el amor que me demuestran día con día y su apoyo incondicional.} \newline

\textit{Durante mi estancia en UPIITA he crecido profesionalmente y esto ha sido en parte por el apoyo de mis mentores y amigos, el Dr. Noé y el Dr. Domingo. Agradezco la inmensa paciencia que nos tuvieron y sus grandes lecciones de aprendizaje.} \newline

\textit{También, agradezco a mi gran amigo Rolando que desde el principio de UPIITA hemos creado una buena amistad, te deseo mucho amor y el mejor de los éxitos en tu vida profesional.}\newline

\textit{Solo me resta dar gracias al IPN por darme las herramientas necesarias para forjarme como ingeniero y las valiosas oportunidades que me fueron brindadas.}\newline \\

\textsc{Rolando:}\\

\textit{Dedicado a mi madre y mi padre, que son el baluarte de mi vida.}

\end{acknowledgements}

%----------------------------------------------------------------------------------------
%	LIST OF CONTENTS/FIGURES/TABLES PAGES
%----------------------------------------------------------------------------------------

\tableofcontents % Prints the main table of contents

\listoffigures % Prints the list of figures

\listoftables % Prints the list of tables

%----------------------------------------------------------------------------------------
%	ABBREVIATIONS
%----------------------------------------------------------------------------------------
\begin{abbreviations}{ll} % Include a list of abbreviations (a table of two columns)
\textbf{3GPP} & \textbf{3}rd \textbf{G}eneration \textbf{P}artnership \textbf{P}roject\\%Ejemplo de acrónimo correcto
\textbf{4G} & Cuarta generación de Comunicaciones Móviles\\
\textbf{5G NR} & 5G New Radio\\
\textbf{ARIB} &	Association of Radio Industries and Businesses\\
\textbf{ARPU} &	Average Revenue Per User\\
\textbf{ATIS} & Alliance for Telecommunications Industry Solutions\\
\textbf{AWGN} &	Additive White Gussian Noise\\
\textbf{apd} & Average Power Decay (a.k.a. PLE)\\
\textbf{BS} & Base Station\\
\textbf{BSs} & Base Stations\\
\textbf{BW} & Bandwidth\\
\textbf{CCSA} &	China Communications Standars Associations\\
\textbf{CD-NOMA} & Code Division NOMA\\
\textbf{CIoT} & Cellular IoT\\
\textbf{CMMPP} & Coupled Markov Modulated Poisson Process\\
\textbf{CT} & Central Terminal\\
\textbf{DES} & Discrete Event Simulation\\
\textbf{DL} & DownLink\\
\textbf{ED} & Event-Driven\\
\textbf{EIRP} & Efective Isotropic Radiated Power\\
\textbf{eMBB} &	Enhanced Mobile Broadband\\
\textbf{eMTC} &	Enhanced Machine Type Communications\\
\textbf{ETSI} &	European Telecommunications Standars institute\\
\textbf{EUTRA} & Evolved UMTS Terrestrial Radio Access\\
\textbf{fc} & Frecuencia Portadora\\
\textbf{GIL} & Global Interpreter Lock\\
\textbf{HPPP} & Homogeneous Poisson Point Process\\
\textbf{HTC} & Human Type Communication\\
\textbf{IEEE} &	Institute of Electrical and Electronics Engineers\\
\textbf{IFT} & Instituto Federal de Telecomunicaciones\\
\textbf{InH} & Indoor Hotspot\\
\textbf{IoT} & Internet of Things\\
\textbf{ITU} & International Telecommunication Union\\
\textbf{ITU-R} & ITU Radiocommunications\\
\textbf{KPI} & Key Performance Indicator\\
\textbf{KPIs} &	Key Performance Indicators\\
\textbf{LoS} & Line of Sight\\
\textbf{LPWAN} & Low Power Wide Area Networks\\
\textbf{LSP} & Large-Scale Parameters\\
\textbf{LTE} & Long Term Evolution\\
\textbf{LTE-M} & LTE-MTC\\
\textbf{M2M} & Machine to Machine\\
\textbf{MA} & Multiple Access\\
\textbf{MAC} & Media Access Control\\
\textbf{MIMO} &	Multiple Inputs Multiple Outputs\\
\textbf{mIoT} &	Massive IoT\\
\textbf{MMPP} &	Markov Modulated Poisson Process\\
\textbf{mMTC} &	Massive Machine Type Communications\\
\textbf{MS} & Mobile Station\\
\textbf{MT} & Mobile Terminal\\
\textbf{MTC} & Machine Type Communication\\
\textbf{MTs} & Mobile Terminals\\
\textbf{M2M} &Machine to machine\\
\textbf{NB-IoT} & Narrow-Band Internet of Things\\
\textbf{NGMN} &	Next Generation Mobile Networks\\
\textbf{NLoS} &	Non-Line of Sight\\
\textbf{NOMA} &	Non-Orthogonal Multiple Access\\
\textbf{OFDMA} & Orthogonal Frequency Division Multiple Access\\
\textbf{OMA} & Orthogonal Multiple Access\\
\textbf{PD-NOMA} & Power Division NOMA\\
\textbf{PE} & Payload Exchange\\
\textbf{PL} & Path Loss\\
\textbf{PLE} & Path Loss Exponent\\
\textbf{PPP} & Poisson Point Process\\
\textbf{PRB} & Physical Resource Block\\
\textbf{PSM} & Power Saving Mode\\
\textbf{PU} & Periodic Update\\
\textbf{QoS} & Quality of Service\\
\textbf{RAN} & Radio Access Networks\\
\textbf{RF} & Radio Frequency\\
\textbf{RFID} & Radio Frequency Identification\\
\textbf{Rx} & Receiver\\
\textbf{SA} & Stand-alone\\
\textbf{SA 2} & System Aspects\\
\textbf{SC-FDMA} & Single Carrier Frequency Division Multiple Access\\
\textbf{SF} & Shadow Fading\\
\textbf{SIC} & Successive Interference Cancelation\\
\textbf{SINR} & Signal to Interference plus Noise Ratio\\
\textbf{SISO} & Single Input Single Output\\
\textbf{SM} & Suburban Macro\\
\textbf{SMM} & Semi-Markov Model\\
\textbf{TDD} & Time Division Duplexing\\
\textbf{TDMA} &	Time Division Multiple Access\\
\textbf{TTI} & Transmission Time Interval\\
\textbf{TR} & Technical Report\\
\textbf{TSDO} &	Telecommunications Standards Development Organization\\
\textbf{TSDOs} & Telecommunications Standards Development Organizations\\
\textbf{TSDSI} & Telecommunications Standards Development Society, India\\
\textbf{TSG} & Technical Specifications Group\\
\textbf{TTA} & Telecommunications Technology Association of Korea\\
\textbf{TTC} & Telecommunication Technology Committee\\
\textbf{Tx} & Transmitter\\
\textbf{UE} & User Equipment\\
\textbf{UL} & UpLink\\
\textbf{UMa} & Urban Macro\\
\textbf{UMi} & Urban Micro\\
\textbf{UMTS} &	Universal Mobile Telecommunications System\\
\textbf{URLLC} & Ultra-Reliable and Low-Latency Communications\\
\end{abbreviations}

%----------------------------------------------------------------------------------------
%	PHYSICAL CONSTANTS/OTHER DEFINITIONS
%----------------------------------------------------------------------------------------

%\begin{constants}{lr@{${}={}$}l} % The list of physical constants is a three column table

% The \SI{}{} command is provided by the siunitx package, see its documentation for instructions on how to use it

%Speed of Light & $c_{0}$ & \SI{2.99792458e8}{\meter\per\second} (exact)\\
%Constant Name & $Symbol$ & $Constant Value$ with units\\

%\end{constants}

%----------------------------------------------------------------------------------------
%	SYMBOLS
%----------------------------------------------------------------------------------------

%\begin{symbols}{lll} % Include a list of Symbols (a three column table)
%\myworries{TODO: Agregar simbolos}
%$a$ & distance & \si{\meter} \\
%$P$ & power & \si{\watt} (\si{\joule\per\second}) \\
%Symbol & Name & Unit \\

%\addlinespace % Gap to separate the Roman symbols from the Greek

%$\omega$ & angular frequency & \si{\radian} \\

%\end{symbols}

%----------------------------------------------------------------------------------------
%	DEDICATION
%----------------------------------------------------------------------------------------

%\dedicatory{For/Dedicated to/To my \ldots} 

%----------------------------------------------------------------------------------------
%	THESIS CONTENT - CHAPTERS
%----------------------------------------------------------------------------------------

\mainmatter % Begin numeric (1,2,3...) page numbering

\pagestyle{thesis} % Return the page headers back to the "thesis" style

% Include the chapters of the thesis as separate files from the Chapters folder
% Uncomment the lines as you write the chapters

% Chapter 1

\chapter{Introducción} % Main chapter title

\label{Chapter1} % Change X to a consecutive number; for referencing this chapter elsewhere, use \ref{ChapterX}

%----------------------------------------------------------------------------------------

% Define some commands to keep the formatting separated from the content 
\newcommand{\keyword}[1]{\textbf{#1}}
\newcommand{\tabhead}[1]{\textbf{#1}}
\newcommand{\code}[1]{\texttt{#1}}
\newcommand{\file}[1]{\texttt{\bfseries#1}}
\newcommand{\option}[1]{\texttt{\itshape#1}}

%----------------------------------------------------------------------------------------
%	SECTION 1
%----------------------------------------------------------------------------------------
\myworries{TODO: FALTA ACTUALIZAR TODO EL CAPITULO 1}\newline

\section{ANTECEDENTES}

Imaginar nuestra vida sin los beneficios brindados por los sistemas de comunicación de hoy en día, o tan sólo sin la tecnología presente en este ámbito desde los últimos 20 años es ya muy difícil, y esto se debe a que en el presente una gran parte de las tareas y actividades, muchas de ellas cruciales para el funcionamiento de nuestras sociedades, operan eficientemente sí y sólo sí se está propiamente conectado y en facultades de compartir información \parencite{Fettweis2014}.\newline

Los sistemas de comunicación celular han tenido saltos generacionales desde la conocida como primera generación, la cual saldría al mercado a finales de la década de los 70’s e inicio de los 80’s, hasta el presente con el desarrollo de la próxima generación (5G), la cual comenzará su implementación en el año 2020. En \parencite{Fettweis2014} encontramos que:\newline

\textit{“La primera y segunda generación de comunicaciones móviles estuvieron dominadas por señales analógicas de audio y posteriormente señales digitales de audio y texto. La tercera generación se trató más de escalar el número de usuarios en la red […] pero fue abrumada por un tsunami de contenido de imágenes y videos.”}\newline

Cada uno de estos saltos de generación ha estado motivado por distintos requerimientos de servicio, necesidades de los usuarios y la aparición de nuevas tecnologías que han buscado ser una vía para facilitar la comunicación entre individuos de todo el mundo y ahora, más recientemente, la comunicación entre máquinas.\newline

El aumento de la tasa de transmisión de datos ha sido siempre un factor a tener en cuenta para el desarrollo de los estándares de las nuevas generaciones de redes móviles, por ejemplo, para la nueva generación se espera “un pico de transmisión de al menos 1 Gb/s al tiempo de su introducción en 2020, esperando que crezca hasta los 10Gb/s para 2025”, \parencite{Fettweis2016}. Pero el sistema de comunicaciones móviles de quinta generación ha estado, además de eso, motivado por un mayor volumen de transmisión de datos, un incremento radical en la cantidad de dispositivos conectados a la red, una menor latencia y una mayor duración de batería para los dispositivos de bajo consumo.\newline

Las limitaciones presentes hasta ahora para las comunicaciones IoT celulares, se deben principalmente a que la red de comunicación móvil fue creada para voz y aplicaciones de texto, evolucionando eventualmente a una transmisión de archivos como imágenes y videos predominante en el enlace de bajada. Por otro lado la comunicación de dispositivos IoT tiene un conjunto de requerimientos muy distintos. Pero es ahora con la quinta generación que se promete brindar las herramientas que esta tecnología necesita para alcanzar su máximo potencial. 5G se trata entonces de la propuesta de crear una red de comunicaciones que logre implementar tanto los servicios inherentes a las necesidades de comunicación entre los humanos y aquellas necesidades de comunicación entre las máquinas. El cumplir con estas últimas necesidades, aseguraría brindar una calidad de servicio óptima para la nueva ola de dispositivos de IoT que se espera estén ya conectados a la red para 2020.\newline

Este proyecto presenta el diseño de un simulador de eventos discretos, el cual modeló el servicio prestado por la arquitectura de red celular que aquí se propone a nodos IoT. Su arquitectura contempló la próxima generación móvil a implementarse (5G) y los servicios seleccionados para atender a aplicaciones del caso de uso mIoT. Este simulador se enfocó en el tráfico generado por dispositivos NB-IoT y con los resultados obtenidos se esclareció sobre qué configuraciones de red son ideales para conseguir una óptima calidad de servicio.\newline

%----------------------------------------------------------------------------------------
%	SECTION 2
%----------------------------------------------------------------------------------------

\section{PLANTEAMIENTO DEL PROBLEMA}

En los recientes años se ha estado presenciando la definición de la tecnología de comunicación móvil 5G en estándares, para su posterior introducción a partir de 2020, y como se resalta en \parencite{Fettweis2016} la nueva generación no sólo seguirá la línea de incrementar la velocidad de transmisión como se ha venido haciendo en cada salto generacional, sino que también traerá consigo la posibilidad de una conectividad adicional sin precedentes, todo esto motivado por la cantidad masiva de dispositivos de IoT que se esperan. Según \textit{Ericsson Mobility Report} \parencite{Ericsson2019}, 22.3 mil millones de dispositivos en el 2024, pertenecerán a una aplicación de IoT. De manera que 5G dará servicio a una enorme cantidad de dispositivos IoT, cada dispositivo enviará pequeños paquetes de datos a lo largo de mucho tiempo, además de que existirá una cierta sincronía en el tráfico generado. Lo anterior ha provocado el desarrollo de nuevas tecnologías que proponen distintas formas de agrupamiento de estos nodos, distintas formas de acceder a los recursos, distintas propuestas de comunicación entre ellos y nuevas formas de que ahorren energía. Siempre teniendo en cuenta los KPIs de la red, como dar servicio a una cantidad masiva de nodos, la duración de la batería de estos y la menor latencia en comunicaciones críticas \footnote{\ Comunicaciones\ que\ requieren\ de\ una\ urgente\ respuesta\ debido\ a\ su\ naturaleza,\ por\ ejemplo\ los\ coches\ auto-dirigidos.} , como se menciona en \parencite{NGMN}.\newline

Esto ha resultado en nuevos retos para la implementación de la red, para la cual se desarrolla tecnología o se mejora la ya existente. En \parencite{GSMAssociation2019}, se presenta la tecnología NB-IoT (\textit{Narrow-Band IoT}), originalmente creada como una solución que brindara servicio a nodos IoT en LTE. Esta tecnología formará parte de los estándares de 5G, como 3GPP\footnote{\ $  $The\ 3rd\ Generation\ Partnership\ Project\ (3GPP),\ desarrolladores\ del\ est\textrm{\'{a}}ndar\ NB-IoT\ y\ el\ 5G\ NR}
lo ha indicado a la ITU (\textit{International Telecommunication Union}). Se pretende que con esta tecnología y algunas mejoras, la red 5G sea capaz de brindar servicio a aplicaciones del caso de uso mMTC\footnote{Cabe mencionar que los casos de uso o tecnologías mIoT y mMTC son análogas.}, para el cual se esperan tener decenas de miles de dispositivos conectados por celda. \newline

En el futuro, los escenarios de IoT, tendrán una enorme cantidad de dispositivos conectados en comparación con los actuales escenarios de la red 4G \parencite{Whatis5G}. Por lo tanto, las tecnologías de 5G deberán brindar servicio a muchos dispositivos usando recursos limitados. Pero dada la elevada complejidad con la que el modelo de un sistema de comunicación como la red 5G puede contar, si de éste se quieren obtener resultados útiles, resulta casi imposible el analizar su comportamiento sino a través de una simulación.\newline

Los patrones de tráfico de nodos de IoT varían según su caso de uso, los cuales se dividen normalmente en tres: \textit{eMBB, uRLLC y mMTC}. Para mMTC también llamado mIoT, en el cual se enfocará el proyecto, se tienen nodos en su amplia mayoría estáticos y como se ve en \parencite{IoTTrafficHossfeld}, para estos nodos podemos considerar por los menos dos patrones de tráfico: el periódico, y el aleatorio ,de manera que el modelo de tráfico deberá tener en consideración esto. Como se puede apreciar en \parencite{IoTTrafficHossfeld}, un problema crítico del Internet de las cosas masivo (mIoT) en las redes móviles es que los dispositivos de IoT causarán una gran congestión en esta si es que no se incorporan mejoras en las arquitecturas de estas redes. Este problema se ve acrecentado debido a que el tráfico de los nodos mIoT presenta cierta sincronía espacial y temporal dependiendo de la aplicación a la que pertenezcan.\newline

Por lo anterior, para la red 5G se necesitan realizar simulaciones en sus distintos casos de uso, que generen resultados sobre qué arquitectura de red y tecnologías brindan un resultado óptimo. Con la realización de este proyecto se pretende aportar al campo de las comunicaciones móviles de quinta generación, que está específicamente interesado en el servicio prestado a los nodos de IoT, de una herramienta de simulación que genere resultados que permitan realizar comparaciones entre distintas configuraciones de red. Todo esto con el fin de que la red 5G próxima a ser desplegada, cumpla con una óptima calidad de servicio para el caso de uso mIoT.\newline

%----------------------------------------------------------------------------------------
%	SECTION 3
%----------------------------------------------------------------------------------------

\section{OBJETIVOS}
\subsection{OBJETIVO GENERAL}

Diseñar e implementar un simulador de teletráfico para el ambiente de Internet de las cosas masivo (mIoT) en una red celular de quinta generación (5G), por medio de la programación de eventos discretos, con la finalidad de evaluar el desempeño de esta red en términos de la cantidad de recursos requeridos para satisfacer niveles esperados de calidad de servicio (QoS).\newline

\subsection{OBJETIVOS ESPECÍFICOS}

\begin{itemize}
    \item Determinar el escenario a implementar, mediante el análisis de requerimientos, para definir después los parámetros de entrada en el modelado del simulador.
    \item Seleccionar los modelos (despliegue, propagación, técnica de acceso múltiple) aplicables a una simulación a nivel de sistema para la comunicación entre nodos IoT y la red 5G.
    \item Seleccionar y determinar los modelos de tráfico a analizar para nodos IoT que hayan sido propuestos en la red 5G, mediante la lectura de distintas publicaciones en revistas científicas, para simular el modelo más adecuado según los alcances propuestos.
    \item Determinar los parámetros de desempeño de la red (KPIs), mediante el análisis de distintas publicaciones científicas, con la finalidad de establecer métricas de QoS.
    \item Definir el procedimiento de la simulación mediante la especificación de su arquitectura y elaboración de diagramas de su funcionamiento y procesos, con el fin de integrar una metodología para su implementación.
    \item Implementar los modelos y protocolos que definen a la comunicación entre los nodos IoT y la red 5G, mediante algoritmos computacionales (incluyendo la técnica de eventos discretos) y de acuerdo a la arquitectura previamente definida.
    \item Implementar una técnica de paralelismo, mediante el uso de multiprocesamiento, con la finalidad de reducir los tiempos de ejecución de las simulaciones.
    \item Evaluar y analizar cada modelo analítico con el uso de escenarios de pruebas y calibración para poder realizar comparaciones con los resultados teóricos esperados y verificar la fiabilidad del simulador 
    \item Simular el modelo de sistema propuesto, mediante la variación de los parámetros de entrada, para caracterizar el desempeño del sistema en términos del tráfico que se puede ofrecer y la cantidad de recursos requeridos para satisfacer objetivos de QoS.
\end{itemize}

%----------------------------------------------------------------------------------------
%	SECTION 4
%----------------------------------------------------------------------------------------

\section{JUSTIFICACIÓN}

En la gran mayoría de los trabajos de investigación revisados, se han realizado estudios de rendimiento de los sistemas de comunicación móvil de quinta generación. Estos se concentran en evaluar distintos modelos, frecuentemente se considera, un modelo de distribución de BSs y UEs, uno de canal, y un esquema de acceso multiple, sin embargo no hay mucha investigación acerca de incorporar todos estos componentes en conjunto con modelos de tráfico para el caso específico de mIoT. La aportación que se pretende hacer con este proyecto se encuentra precisamente en ese ámbito.\newline

Los enlaces de comunicación inalámbrica experimentan fenómenos físicos perjudiciales al canal como lo son las multi-trayectorias y los desvanecimientos debido a grandes objetos que se interceptan en la trayectoria de la propagación, además el rendimiento de los sistemas celulares inalámbricos tiende a limitarse debido a la interferencia de otros usuarios. Estas condiciones complejas del canal son difíciles de describir con un simple modelo analítico, es por esto que las aproximaciones de las simulaciones son necesarias. Estas pueden analizar el rendimiento de los enlaces de comunicaciones celulares \parencite{WirelessSim}, modelando un gran número de eventos aleatorios a través del tiempo, mediante el uso de simulaciones orientadas a eventos discretos. \newline

Una simulación permite observar muchas de las interacciones de un sistema, que de otra forma tomaría mucho trabajo predecirlas o calcular, además de que proporciona un método importante de análisis, que resulta sencillo de comunicar y comprender. En todas las industrias y disciplinas, la creación de simulaciones brinda soluciones valiosas al proporcionar información clara sobre sistemas complejos \parencite{WirelessSim}. Los resultados de una simulación que haga las suposiciones adecuadas y modele correctamente el sistema propuesto, brindarán confianza y claridad, ahorrarán tiempo y muy posiblemente también dinero.\newline

Este proyecto servirá como referencia a los investigadores y estudiantes que busquen comparar los modelos y técnicas propuestos en este trabajo con otras selecciones posibles para la futura red 5G y el servicio brindado a los nodos IoT.\newline

Por otra parte, se debe enfatizar que el desarrollo de este proyecto requiere de la aplicación de conocimientos relacionados a la informática (entre los que se incluyen desarrollo de software y algoritmos computacionales), así como del dominio de conceptos propios de las telecomunicaciones (por ejemplo, análisis de tráfico y caracterización de enlaces inalámbricos). De acuerdo a lo anterior, se considera que este proyecto pertenece al campo de aplicación de la ingeniería telemática. Además, se debe notar que, si bien los fenómenos simulados corresponden al proceso de transmisión de información, dichos fenómenos están siendo analizados en el contexto de un sistema con características telemáticas (nodos mIoT conectados a la red 5G).\newline

%----------------------------------------------------------------------------------------
%	SECTION 5
%----------------------------------------------------------------------------------------

\section{PROPUESTA DE SOLUCIÓN}

Considerando lo expuesto anteriormente, se propone desarrollar un simulador a nivel sistema que será programado bajo el paradigma de eventos discreto. La elección de una simulación a nivel de sistema se deriva del enfoque que tendrá nuestro proyecto hacia los distintos tipos de tráfico de nodos NB-IoT y la simulación de cada uno de ellos como fuente de tráfico. Las simulaciones a nivel de sistema permiten modelar el comportamiento de múltiples radio bases, múltiples nodos como fuentes de tráfico, la propagación de las señales y la interferencia que estas causan, a la vez que se realizan abstracciones más simples de lo que sucede más allá de estas interacciones. Esto facilita la implementación de una gran cantidad de actores. La generación de variables aleatorias vendrá de la mano de las distintas aplicaciones mIoT y sus patrones de transmisión estocásticos, además de la localización de  nodos en un plano la cual no será uniforme.\newline

El simulador será capaz de evaluar la calidad del servicio que la red celular propuesta ha de brindar a nodos de mIoT. Dicha arquitectura de red, propuesta en este mismo trabajo está basada a su vez en los avances hechos, por grupos como 3GPP, hacia el despliegue de la red 5G. La base de la que se partió es la tecnología NB-IoT, la cual abordó el caso de uso mMTC en la red 4G donde ha estado prestando servicio a nodos mIoT a una escala menor que la esperada en 5G.\newline

Se modelará el servicio brindado a nodos estáticos de IoT cuyas aplicaciones pertenecen al caso de uso mIoT. Con la ayuda del estándar NB-IoT, para el que se proponen mejoras en el acceso múltiple, esto en búsqueda de cumplir con los KPI’s de la red 5G, ya que si bien NB-IoT pertenecerá al paradigma de 5G, no es viable tál y como existe ahora para cumplir con los requerimientos. La importancia de esto recae, como se menciona en \parencite{EjazIoT}, en que la tecnología de IoT ha creado una revolución en la última década con la creación de aplicaciones pensadas alrededor de todo tipo de sensores, lo que resulta en una proyección estimada de miles de millones de dispositivos IoT para el 2020 [3]. Esta misma referencia asegura que IoT está tomando un papel principal en el desarrollo de la quinta generación, debido a que se espera que los dispositivos de IoT formen la gran mayoría de dispositivos en esta nueva generación que se avecina.\newline

Se propone un análisis fundamental principalmente del modelado de cuatro componentes que son esenciales para la caracterización de un sistema de comunicación móvil. Estos componentes corresponden al: modelado de despliegue de usuaios, canal, tráfico y un esquema de acceso múltiple al medio. Esto se encontrará en el capítulo 4, que comprende un análisis de forma detallada de estos, pero a continuación se abordan de tal forma que se esclarezca la arquitectura del modelo de sistema, presentada también más adelante.\newline

Se considerará un modelo de despliegue de nodos IoT que seguirá un proceso puntual de Poisson con el fin de crear una geometría estocástica, se representarán las pérdidas por medio de un modelo de canal estadístico para ambientes celulares de quinta generación, se considerará en la simulación un modelo de tráfico fuente en el que cado nodo mIoT generará tráfico ya sea periódico o aleatorio, cada caso con distintas tasas y por último, referente al método de acceso múltiple, se aplicará una mejora a la tecnología NB-IoT, se trata de la implementación del esquema NOMA en el dominio de la potencia, de forma que agrupamientos (de longitud fija) de nodos estarán compartiendo un mismo recurso (una sub-banda).\newline

Este análisis conllevará en conjuntos de la red 5G y los dispositivos NB-IoT en un ambiente masivo, conseguirá resultados que podrán brindar una base fundamental para evaluar el desempeño de estas redes y por supuesto, su dimensionamiento en términos de objetivos de QoS.\newline

%----------------------------------------------------------------------------------------
%	SECTION 6
%----------------------------------------------------------------------------------------

\section{ALCANCES}

Se obtendrán resultados que permitan analizar las configuraciones de la arquitectura de red propuesta que conllevan a una óptima calidad de servicio. Teniendo como métricas principales la densidad de usuarios soportada y la tasa de transmisión máxima alcanzada. Estos resultados reflejarán a su vez las ventajas que puede traer la selección de cierta arquitectura de red y su despliegue. Es aquí donde se encuentra una de las ventajas de realizar un simulador, ya que con la ayuda de múltiples computadoras, se podrán simular miles de nodos mIoT en esta red. Las configuraciones y parámetros de la red podrán modificarse al inicio de cada simulación, y podrán ser inspeccionados mientras esta corre. La variación de estos parámetros a lo largo de múltiples simulaciones permitirá generar tablas y gráficas de los resultados obtenidos.\newline

Este proyecto no cubrirá los aspectos de movilidad entre celdas para los nodos de IoT, ya que el caso de uso mMTC representa a los nodos estáticos en su mayoría o con velocidades menores a 3Km/h.  No se desarrollarán nuevos modelos probabilísticos o matemáticos de ninguna clase, sino que se implementarán los existentes para el escenario propuesto.\newline

% Chapter 2

\chapter{Marco Teórico} % Main chapter title

\label{Chapter2} 

El objetivo de este capítulo fue revisar los fundamentos de la teoría de los sistemas de comunicaciones móviles, comenzando desde las distribuciones de probabilidad utilizadas para caracterizar los fenómenos más importantes en este ámbito, después se ahondó en las pérdidas en un sistema celular por medio de los modelos de canal más comunes y con su caracterización en parámetros a larga y pequeña escala, p.ej., la pérdida por trayectoria y el desvanecimiento de las señales de radio.\newline

Además, se repasó la teoría del concepto celular, es decir, la geometría celular clásica que sirve para la eficiencia en la planificación de los recursos y por lo tanto el problema más importante en estos sistemas: los efectos de la interferencia. \newline

Finalmente, se revisaron los aspectos de la teoría del tráfico en telecomunicaciones, los organismos más importantes de estandarización de redes móviles y algunos conceptos de las simulaciones a nivel de sistema orientados a eventos discretos en conjunto con los lenguajes de programación más utilizados.\newline


%----------------------------------------------------------------------------------------
%	SECTION 1
%----------------------------------------------------------------------------------------

\section{DISTRIBUCIONES ESTADÍSTICAS EN TELECOMUNICACIONES}

El uso de modelos estadísticos es importante para describir \parencite{Correia2018}:
\begin{itemize}
    \item Llamadas telefónicas y conexiones de datos
    \item Influencia del usuario en el rendimiento de la red
    \item Propagación no guiada en ambientes aleatorios
    \item Movilidad del usuario
\end{itemize}

Comúnmente se utilizan las siguientes distribuciones de probabilidad en telecomunicaciones \parencite{Correia2018}:

\begin{enumerate}
    \item Distribución Uniforme: Es usada para describir la fase de una señal. También, se ha utilizado para simular el despliegue de BSs \parencite{TurjmanSmallCells}.
    \item Distribución Normal (Gaussiana): Es usada para describir fluctuaciones alrededor de un valor medio, p.ej. shadowing. Esta distribución no puede ser usada para describir entidades que no pueden ser negativas.
    \item Distribución Log-Normal: Es usada para describir entidades como la potencia de una señal, amplitudes, principalmente el desvanecimiento lento.
    \item Distribución Rayleigh: Es usada para describir el desvanecimiento rápido-intenso.
    \item Distribución Susuki: Describe conjuntamente el desvanecimiento lento y rápido.
    \item Distribución Rice: Es usada para describir el desvanecimiento rápido - no-intenso.
    \item Distribución Exponencial: Es ampliamente usada para describir la duración de diferentes fenómenos, principalmente asociados con el desvanecimiento de señales y las llamadas telefónicas.
    \item Distribución de Bernoulli: Es usada para describir la ocupación de canales de telecomunicaciones.
    \item Distribución binomial: Es usada para describir llamadas telefónicas.
    \item Distribución de Poisson: Es usada para describir la generación de llamadas telefónicas.
\end{enumerate}

Las PDF y CDF son de suma importancia en el área de las telecomunicaciones ya que ayudan a caracterizar estadísticamente diferentes fenómenos.\newline

\myworries{TODO: incluir procedimientos para generar variables aleatorias a partir de una distribución uniforme}

%----------------------------------------------------------------------------------------
%	SECTION 
%----------------------------------------------------------------------------------------

\section{MODELADO DEL CANAL CELULAR}

Los modelos de propagación por radio se clasifican en modelos a gran escala y a pequeña escala. Los efectos a gran escala generalmente ocurren en el orden de cientos a miles de metros de distancia. Los efectos a pequeña escala se localizan y ocurren temporalmente (en el orden de unos pocos segundos) o espacialmente (en el orden de unos pocos metros). Los parámetros del canal generalmente se dividen en Pérdida por trayectoria (PL), parámetros de gran escala (LSP, como sombreado, dispersión de retardo, dispersión angular, etc.) y parámetros de pequeña escala (como demora, ángulo de llegada y salida, etc.), que reflejan conjuntamente las características de desvanecimiento del canal. El procedimiento de generación de los coeficientes del canal se puede apreciar en la Figura 1. La pérdida de ruta generalmente se expresa en una o dos fórmulas y un conjunto de valores numéricos de parámetros, que reflejan las relaciones con el entorno de transmisión, la distancia y la frecuencia, etc. \newline

\begin{figure}[th]
\centering
\includegraphics[scale=0.8]{Figures/Procedimiento de generación de coeficientes de canal}
\decoRule
\caption[Procedimiento de generación de coeficientes de canal]{Procedimiento de generación de coeficientes de canal, [Fuente: 3GPP TR36.873]}
\label{fig:Procedimiento de generacion de coeficientes de canal}
\end{figure}

El rendimiento a nivel de enlace es un fenómeno de pequeña escala el cual lidia con cambios instantáneos en el canal a través de áreas e instantes de tiempo pequeños donde se considera la potencia recibida como constante, por otra parte, en las simulaciones a nivel de sistema para determinar el rendimiento en general del sistema para un gran número de usuarios esparcidos en una área geográfica es necesario incorporar parámetros de larga escala como el comportamiento estadístico de la interferencia, así como los niveles de señal experimentados por cada usuario a través de largas distancias, ignorando las características transitorias del canal (las de pequeña escala) \parencite{Tranter2003}. En una simulación a nivel de sistema, principalmente se busca la probabilidad de que un usuario en particular alcance un servicio aceptable en el sistema, para esto es necesario contemplar los efectos de los múltiples usuarios para cada enlace individual entre un móvil y la estación base. Por lo tanto en las simulaciones a nivel de sistema se suelen omitir los parámetros a pequeña escala.\newline

\subsection{Relaciones Generales de Propagación}

La pérdida por trayectoria $L_p\ $ se define como \parencite{Correia2018}:

\begin{equation}
L_{p[dB]}=P_{tx[dBm]}+G_{tx[dBi]}-P_{rx\left[dBm\right]}+G_{rx\left[dBi\right]}
\label{eqn:Lp}
\end{equation}

Donde:\newline
\[P_{tx}\to Potencia\ de\ la\ antena\ transmisora\ \] 
\[G_{tx}\to Ganancia\ de\ la\ antena\ transmisora\ \] 
\[P_{rx}\to Potencia\ de\ la\ antena\ receptora\ \ \] 
\[G_{rx}\to Ganancia\ de\ la\ antena\ receptora\ \ \] 

En muchas aplicaciones la ganancia de la antena es referida al dipolo de media longitud de onda:\newline

\begin{equation}
G_{[dBi]} = G_{[dBd]}+{2.15} 
\label{eqn:Gain}
\end{equation}

La Potencia Isotrópica Radiada Efectiva (EIRP) se define como:

\begin{equation}
P_{EIRP[dBm]}=P_{tx\left[dBm\right]}{\ +\ G}_{tx[dBi]}
\label{eqn:EIRP}
\end{equation}

\subsection{Pérdida por trayectoria en el Espacio Libre (FSPL, \textit{Free Space Path Loss})}

El receptor puede recibir una señal atenuada directa (también llamada señal de línea de vista (LoS)) del transmisor. El FSPL se utiliza para predecir la pérdida de trayectoria cuando hay un LoS claro y sin obstrucciones entre el transmisor y el receptor. Se basa en la ley de distancia al cuadrado inverso que establece que la potencia recibida (PRX) decae por un factor de cuadrado de la distancia (d) desde el transmisor.\newline

Se considera a la propagación en el espacio libre como la mínima atenuación que una señal puede sufrir en el medio.\newline

La potencia disponible en la antena receptora ${\ P}_{rx}$ con una propagación en el espacio libre se define como (también conocida como Formula de Friis):\newline

\begin{equation}
P_{rx\left[W\right]}={\left(\frac{{\lambda }_{[m]}}{4\pi d_{[m]}}\right)}^2P_{tx\left[W\right]}G_{tx}G_{rx}  
\label{eqn:Friis}
\end{equation}

ó\newline

\begin{equation}
P_{rx\left[dBW\right]}=-32.44+P_{tx\left[dBW\right]}+G_{tx\left[dBi\right]}+G_{rx\left[dBi\right]}-20{\mathrm{log} \left(d_{\left[km\right]}\right)\ }-20{\mathrm{log} \left(f_{\left[MHz\right]}\right)\ }
\label{eqn:Friss_dB}
\end{equation}

Donde:\newline
\[d\to Distancia\ entre\ Rx\ y\ Tx\ \] 
\[f\to Frecuencia\ de\ operaci\textrm{\'{o}}n\ \] 
\[\lambda \to Longitud\ de\ onda,\ \ \ \ \ \ \ \ \lambda =\frac{c}{f}\ \] 
\[c\to Velocidad\ de\ la\ luz\ (\mathrm{299\ 792\ 458\ m/s})\] 

Por lo tanto, la p\'{e}rdida por trayectoria en el espacio libre $L_0$ se define como:\newline

\begin{equation}
L_{0[dB]}=32.44+20{\mathrm{log} \left(d_{\left[km\right]}\right)\ }+20\mathrm{log}\mathrm{}(f_{[MHz]})
\label{eqn:Lp}
\end{equation}

Tomando el modelo del decaimiento de potencia promedio con la distancia $a_{pd}$:\newline
\begin{equation}
L_{p[dB]}=L_{ref}+10a_{pd}{\mathrm{log} \left(d_{\left[km\right]}\right)\ }
\label{eqn:Lp_ref}
\end{equation}

\[a_{pd}=2,\ \ \ para\ una\ propagacion\ en\ el\ espacio\ libre\ \] 

El \textit{apd (también conocido como PLE)} es un valor que va de 2 a 4 frecuentemente. El valor mínimo (i.e. 2) proviene de la perdida FSPL y el máximo (i.e. 4) por la pérdida del modelo \textit{Flat Earth} (modelo de tierra plana). En algunos modelos se llega a incluir valores de PLE m\'{a}s altos que los aqu\'{i} definidos.\newline

\subsection{Caracterización del canal de radio}

Usualmente en ambientes urbanos no hay línea de vista (LoS) entre la estación base (BS) y la terminal móvil (MT\footnote{MT y UE son términos análogos.}) [véase Figura~\ref{fig:Propagacion}] por lo que la transmisión es realizada por reflexión, difracción y dispersión de las señales.\newline

\begin{figure}[th]
\centering
\includegraphics[scale=.5]{Figures/Propagación de señales celulares en ambientes urbanos.}
\decoRule
\caption[Propagación de señales celulares en ambientes urbanos]{Propagación de señales celulares en ambientes urbanos, [Fuente: L. Correia 2018]}
\label{fig:Propagacion}
\end{figure}

Sin embargo estas señales sufren de desvanecimiento con caídas de potencia. Este desvanecimiento depende de la posición y el ambiente del cual se propague la señal.\newline
Características de desvanecimiento:\newline

\begin{itemize}
    \item Desvanecimiento lento:
    Depende esencialmente de la distancia, sigue una distribución Log-normal
    \item Desvanecimiento rápido:
    Es asociado al movimiento del usuario, sigue una distribución Rice
\end{itemize}

\begin{figure}[th]
\centering
\includegraphics[scale=.5]{Figures/Ejemplo de niveles de señal con desvanecimiento lento y desvanecimiento rápido}
\decoRule
\caption[Ejemplo de niveles de señal con desvanecimiento lento y desvanecimiento rápido]{Ejemplo de niveles de señal con desvanecimiento lento y desvanecimiento rápido, [Fuente: V. Mathuranathan, 2016]}
\label{fig:Desvanecimientos}
\end{figure}

En la Figura~\ref{fig:Desvanecimientos} se observa que al principio, la señal parece muy aleatoria. Mirando más de cerca podemos dividirlo en tres componentes principales como se muestra en la mitad derecha de la Figura~\ref{fig:Desvanecimientos} \parencite{Mathuranathan2016}.\newline

El desvanecimiento lento puede ser causado por eventos como el \textit{shadowing}, donde una gran obstrucción, como una colina o un gran edificio, oscurece la trayectoria de la señal principal entre el transmisor y el receptor. Se considera un parámetro a gran escala.\newline

El desvanecimiento rápido ocurre cuando la amplitud y el cambio de fase impuestos por el canal varían considerablemente durante el período de uso. Una señal que viaja en un entorno puede verse reflejada por varios objetos en el camino. Esto da lugar a varias señales reflejadas. Las señales reflejadas llegan al receptor en diferentes instantes de tiempo y con diferentes intensidades que conducen a la propagación multitrayectoria. Se considera un parámetro a pequeña escala.\newline

Los márgenes de desvanecimiento deben tomarse en cuenta para caracterizar la variación de las señales alrededor de un valor promedio, esto depende de:\newline
•	Características del ambiente (LoS)\newline
•	QoS\newline

Para \textit{narrowband} (banda estrecha, donde prevalece el desvanecimiento plano en lugar de un desvanecimiento selectivo de frecuencia) el desvanecimiento se caracteriza de la siguiente manera:
\begin{itemize}
    \item Desvanecimiento rápido:
    \begin{itemize}
        \item LoS: Distribución Rice (no intenso)
        \item NLoS: Distribución Rayleigh (intenso)
    \end{itemize}
    \item Desvanecimiento lento:
    \begin{itemize}
        \item Distribución Log-Normal
    \end{itemize}
    \item Ambos desvanecimiento rápido y lento:
    \begin{itemize}
        \item Distribución Susuki
    \end{itemize}
\end{itemize}

Los modelos de estimación de señal pueden ser divididos en dos categorías:
\begin{enumerate}
    \item Teóricos:Son una aproximación a la realidad, no toman en cuenta todos los factores de la propagación pero permiten cambios fáciles de los parámetros. 
    \begin{itemize}
        \item Ray Tracing
        \item Modelo Ikegami [1984]
        \item Modelo Walfish-Bertoni [1988]
    \end{itemize}
    \item Empíricos: Están basados en la observación de mediciones, conduciendo al mejor ajuste de ecuaciones. Tienen la ventaja de tomar en cuenta todos los factores que influyen en la propagación.\newline
    Para ambientes exteriores hay dos modelos básicos:\newline
    \begin{itemize}
        \item COST 231 Okumura-Hata
        \begin{itemize}
            \item Largas distancias (>5km)
            \item Ambientes rurales, urbanos y suburbanos
            \item Alta desviacion estandar
            \item Rango de frecuencias aplicables [1.5,2.0] GHz
        \end{itemize}
        \item COST 231 Walfish-Ikegami [1999]
        \begin{itemize}
            \item Cortas distancias (<5km)
            \item Ambientes urbanos y suburbanos
            \item Rango de frecuencias aplicables [.8,2.0] GHz
        \end{itemize}
        \item COST 207 [1989]
    \end{itemize}
\end{enumerate}

%----------------------------------------------------------------------------------------
%	SECTION 
%----------------------------------------------------------------------------------------

\section{GEOMETRÍA CLÁSICA CELULAR}
\subsection{Planeación Celular}
\subsection{Planeación de frecuencia}

%----------------------------------------------------------------------------------------
%	SECTION 
%----------------------------------------------------------------------------------------

\section{INTERFERENCIA EN SISTEMAS CELULARES}

%----------------------------------------------------------------------------------------
%	SECTION 
%----------------------------------------------------------------------------------------

\section{CAPACIDAD EN SISTEMAS DE COMUNICACIONES}

%----------------------------------------------------------------------------------------
%	SECTION 
%----------------------------------------------------------------------------------------

\section{INTERFAZ DE RADIO}
\subsection{Esquemas de Acceso Múltiple al Medio}
\subsection{Generaciones anteriores de sistemas de comunicaciones móviles}

%----------------------------------------------------------------------------------------
%	SECTION 
%----------------------------------------------------------------------------------------

\section{TELETRÁFICO}
\subsection{Caracterización del Tráfico}
\subsection{Notación Kendall}

%----------------------------------------------------------------------------------------
%	SECTION 
%----------------------------------------------------------------------------------------

\section{SIMULACIONES ORIENTADAS A EVENTOS DISCRETOS}

%----------------------------------------------------------------------------------------
%	SECTION 
%----------------------------------------------------------------------------------------

\section{SIMULACIÓNES A NIVEL DE SISTEMA}

%----------------------------------------------------------------------------------------
%	SECTION 
%----------------------------------------------------------------------------------------

\section{LENGUAJES DE PROGRAMACIÓN PARA SIMULACIONES ORIENTADAS A EVENTOS DISCRETOS (DES)}
\subsection{Python}
\subsection{Librería Simpy}
%----------------------------------------------------------------------------------------
%	SECTION 
%----------------------------------------------------------------------------------------

\section{ORGANISMOS INTERNACIONALES DE ESTANDARIZACIÓN}
 
% Chapter 3
\chapter{Estado del Arte}
\label{Chapter3} % Change X to a consecutive number; for referencing this chapter elsewhere, use \ref{ChapterX}

En este Capítulo se encuentran trabajos anteriormente realizados que están relacionados con este proyecto, esto con la finalidad de presentar antecedentes, y dejar ver las discrepancias y similitudes existentes.\newline

El trabajo de dimensionar los sistemas de comunicación móviles es una necesitad recurrente en cada nueva generación celular. En \parencite{Celis2016}, se encuentra un proyecto terminal realizado por alumnos de la UPIITA, en el cual se realizó un simulador, bajo el paradigma de eventos discretos. El sistema objetivo era el celular 4G, con un enfoque en distintos esquemas de reúso de frecuencias y calendarizadores para obtener resultados sobre qué combinación de estos y bajo qué condiciones el sistema tenía un mayor desempeño. En este proyecto la simulación se llevó a cabo utilizando Matlab.\newline

Por otra parte, el reciente crecimiento de los casos de uso de IoT en una amplia gama de aplicaciones ha traído la necesidad de una mejor caracterización del tráfico tipo máquina. Reconociendo la importancia de esta cuestión, la 3GPP ha propuesto dos modelos para el tráfico tipo máquina. Se tratan de modelos de tráfico agregado . El primero modela el tráfico generado aleatorio, mientras que el segundo modela el tráfico síncrono con el tiempo. Dado que los modelos sólo se centran en el tráfico agregado, estos pueden no ser apropiados para el análisis práctico en algún sistema que esté más enfocado en el tráfico y requiera mayor precisión.\newline

En \parencite{Gupta2018}, los autores consideraron modelar el tráfico de dispositivos IoT conectados a través de tecnologías LPWAN. Debido a las diversas aplicaciones de IoT, no es trivial tener un solo modelo de tráfico para representarlas a todas, el tráfico puede clasificarse ampliamente como periódico, activado por eventos o una combinación de ambos. Evaluaron el rendimiento de LoRaWAN, en presencia de un híbrido de ambos tipos de tráfico, donde los eventos se propagan espacialmente a lo largo del tiempo. Utilizaron el modelo CMMPP para representar dicho tráfico característico de dispositivos IoT, que suelen ser activados por eventos.  \newline

De igual manera, pero ahora con un enfoque en sistemas celulares LTE, en \parencite{Smiljkovic2014} los autores analizaron el tráfico M2M con velocidad de datos variable bajo el supuesto de que la red LTE tiene recursos limitados. Los resultados muestran las características del tráfico M2M de una manera más realista, identificando las diferencias del tráfico estándar en la red celular. Revelan que el tráfico cuenta con la propiedad de auto similitud sólo para una gran cantidad de MTC.\newline

Estos trabajos muestran el uso de modelos de tráfico CMMPP para evaluar el impacto de la tecnología IoT. La integración del tráfico M2M será una parte inevitable de la evolución de las redes. En este proyecto se implementó el modelo CMMPP, en una red de 5G, teniendo en cuenta la arquitectura NB-IoT y distintas aplicaciones de IoT.\newline

Ahora bien, del lado de los esquemas de acceso no ortogonales (NOMA) en redes de quinta generación, se sigue trabajando en las propuestas de su implementación. En \parencite{Zhang2017}, los autores proponen un sistema usando geometría estocástica (PPP) para modelar un ambiente inalámbrico denso que admita NOMA tanto en el enlace de subida como en el enlace de bajada. \newline

En la implementación de NOMA propuesta, se tienen dos esquemas de emparejamiento de usuario: uno aleatorio y otro selectivo:
\begin{enumerate}
\item  Cuando el agrupamiento es aleatorio, los UE son seleccionados aleatoriamente.
\item  Cuando el agrupamiento es selectivo, el primer UE deberá tener una relación señal-interferencia más ruido (SINR) por encima del umbral T1 y el segundo UE tiene un SINR por debajo del umbral T2, T2 $\mathrm{\le}$ T1.
\end{enumerate}

Consideraron un error de propagación SIC durante el proceso de decodificación por parte del UE. Además, optaron por una estrategia de asignación de potencia fija, donde la potencia de enlace de bajada asignada a un UE está predefinida y permanece sin cambios.\newline

\begin{figure}
\centering
\begin{minipage}{.45\linewidth}
  \includegraphics[width=\linewidth]{Modelo de sistema para el sistema de enlace descendente NOMA}
  \captionof{figure}{Modelo de sistema para el sistema de enlace descendente NOMA}
  \label{fig:img1}
\end{minipage}
\hspace{.05\linewidth}
\begin{minipage}{.45\linewidth}
  \includegraphics[width=\linewidth]{Modelo de sistema para el sistema de enlace ascendente NOMA}
  \captionof{figure}{Modelo de sistema para el sistema de enlace ascendente NOMA}
  \label{fig:img2}
\end{minipage}
\end{figure}

Las ganancias implementan el desvanecimiento de Rayleigh entre la $BS_0$ y $UE_i$. La ganancia en potencia de desvanecimiento Rayleigh entre BS y UE sigue una distribución exponencial con media 1 y se distribuye de forma independiente e idéntica (i.i.d.)\newline

En el enlace descendente (DL), agregaron perdidas por trayectoria con un exponente de pérdida y calcularon la interferencia entre celdas acumulativa de todas las bases adyacentes. En el enlace ascendente (UL), la interferencia inter-celdas proviene de todos los otros UEs que comparten la misma sub-banda. \textit{[Véanse Figuras~\ref{fig:img1}, \ref{fig:img2}]}\newline

Este trabajo se enfocó en realizar NOMA para sistemas 5G, sin embargo, no consideraron la actuación de dispositivos IoT, los cuales al tener diferentes calidades de servicio, impactarían en la toma de decisiones del modelo de acceso múltiple, más en concreto en el emparejamiento de usuarios.\newline

En \parencite{Mostafa2019} se trabajó en emplear NOMA para mejorar la densidad de conexión en los sistemas NB-IoT. En su propuesta cada subportadora puede dar servicio máximo a dos dispositivos con distintos requisitos de QoS. Formularon problemas de asignación de potencia de transmisión y subportadoras conjuntas para los modos singletone y multitone.  Además, propusieron algoritmos heurísticos con baja complejidad y dieron como resultado un rendimiento cercano a las soluciones óptimas y subóptimas en ambos casos. Los resultados de la simulación mostraron que el uso de NOMA aumentó la densidad de conexión hasta en un 87\% en comparación con OMA en el modo singletone y en el modo multitono, la densidad de conexión también se incrementó hasta en un 24\%.\newline

En \parencite{Shahini2019} desarrollaron un esquema NOMA en el dominio de potencia con agrupación de usuarios en un sistema NB-IoT. Resolvieron un problema de optimización para maximizar el rendimiento total de la red al optimizar la asignación de recursos de los dispositivos MTC y la agrupación de NOMA al tiempo que se satisfacen los requisitos de potencia de transmisión y QoS. Además, diseñaron un algoritmo heurístico eficiente para resolver el problema de optimización propuesto mediante la agrupación NOMA y la asignación de recursos a dispositivos de tipo máquina.\newline

En su modelo de sistema, consideraron el escenario de una única celda (eNB), que admite dispositivos de tipo máquina operando con la tecnología NB-IoT. Asumieron que no hay interferencia proveniente de otras células vecinas. \newline

\begin{figure}[th]
\centering
\includegraphics[scale=.6]{Figures/Modelo de sistema de una sola celda usando clusterización}
\decoRule
\caption[Grupos NOMA que incluyen dispositivos mMTC y URLLC, donde los dispositivos MTC comparten los subcanales asignados a cada clúster NOMA.]{Grupos NOMA que incluyen dispositivos mMTC y URLLC, donde los dispositivos MTC comparten los subcanales asignados a cada clúster NOMA.}
\label{fig:NOMA_NBIOT}
\end{figure}

Propusieron un esquema NOMA en el dominio de la potencia agrupando (de entre 1 a 4) dispositivos mMTC y URLLC en una red NB-IoT como se muestra en la Figura~\ref{fig:NOMA_NBIOT} . Según el esquema NOMA, los dispositivos mMTC y URLLC comparten cada subportadora (subcanal) y transmiten datos de manera no ortogonal. Por lo tanto, los dispositivos se dividen en diferentes grupos, llamados ``clusters''. Para decodificar con éxito los mensajes de la suma de mensajes recibida, el eNB emplea el esquema SIC. Por lo tanto, los usuarios deben ordenarse en cada grupo teniendo en cuenta el método SIC.\newline

Se ha investigado bastante acerca del desempeño de los esquemas NOMA para dispositivos IoT, sin embargo nuestra propuesta resulta diferente ya que además de utilizar PD-NOMA, se implementó un despliegue de UEs usando un PPP, un modelo de canal más realista y por último, un modelo de trafico de tipo fuente.

% Chapter 4
\chapter{Análisis} % Main chapter title
\label{Chapter4} % Change X to a consecutive number; for referencing this chapter elsewhere, use \ref{ChapterX}

Este capítulo contiene un análisis de las diversas aplicaciones que tienen los dispositivos IoT, además se abordan las técnicas y modelos necesarios para diseñar un sistema que brinde servicio a estos dispositivos. Se comienza con una breve descripción de la red 5G y el papel que los dispositivos IoT tienen en esta. Posteriormente se profundiza en el caso de uso mMTC, donde se mencionan los escenarios más comunes de implementación, su clasificación y sus características. Se revisa también el estándar actual (NB-IoT\footnote{ Muchos\ de\ los\ modelos\ aqu\textrm{í}\ propuestos\ est\textrm{á}n\ basados\ en\ trabajos\ de\ la\ 3GPP,\ como\ se\ revis\textrm{ó}\ en\ el Capítulo II,\ la\ 3GPP,\ es\ una\ organizaci\textrm{ó}n\ que\ est\textrm{á}\ respaldada\ por\ organismos\ alrededor\ de\ todo\ el\ mundo,\ adem\textrm{á}s\ de\ que\ se\ trata\ del\ grupo\ que\ estandariza\ tecnolog\textrm{í}as\ como\ LTE-M\ y\ NB-IoT,\ de\ manera\ que\ es\ una\ indudable\ referencia\ en\ su\ ahora\ inmersi\textrm{ó}n\ en\ la\ estandarizaci\textrm{ó}n\ de\ 5G.}) que se ha seleccionado como una de las tecnologías que brindará servicio a los dispositivos IoT en la red 5G. \newline

Por último, se presentan los modelos que fueron usados para caracterizar el despliegue, la condición del canal y la generación de tráfico de comunicaciones de tipo máquina. Se presenta también una propuesta de técnica de acceso múltiple al medio no ortogonal (NOMA), que permitiría a la tecnología NB-IoT brindar servicio a más dispositivos IoT usando agrupamientos \textit{(clustering)}.

%----------------------------------------------------------------------------------------
%	SECTION 
%----------------------------------------------------------------------------------------

\section{REDES 5G/IoT}

El Internet de las cosas (IoT) tendrá un importante papel en las redes 5G. La actual red celular LTE (4G) no está diseñada para satisfacer las demandas de conectividad de múltiples dispositivos, velocidad de datos, calidad de servicio (QoS) de baja latencia y de eficiencia energética. Para abordar estos desafíos, 5G contempló a IoT como uno de sus pilares \parencite{Chetri2020}. \newline


%----------------------------------------------------------------------------------------
%	SECTION 
%----------------------------------------------------------------------------------------

\section{CLASIFICACIÓN Y ANÁLISIS DE LOS ÁMBITOS DE IoT}

Se tomó como referencia el trabajo realizado en \parencite{NetTrafficIoT} como una guía de los servicios que se espera los nodos IoT brinden en un futuro próximo. Los servicios se presentan en 8 dominios: edificios inteligentes y vivienda (\textit{Smart buildings and living}), cuidado de la salud inteligente (\textit{Smart healthcare}), medio ambiente inteligente (\textit{Smart environment}), ciudades inteligentes (\textit{Smart city}), energía inteligente (\textit{Smart energy}), transporte y movilidad  inteligentes (\textit{Smart transport and mobility}), fabricación y venta inteligentes (\textit{Smart manufacturing and retail}), agricultura inteligente (\textit{Smart agriculture}). Para cada uno de los dominios se especifican aplicaciones típicas que se podrían encontrar, sus características de tráfico y las tecnologías de red más adecuadas para darles servicio entre otras cosas.\newline

La primera parte del análisis correspondió a la selección de los dominios que resultasen adecuados para el sistema que se diseñó. Los dominios seleccionados fueron aquellos que se acoplan primordialmente es una red de área amplia de bajo consumo (LPWAN, \textit{Low Power Wide Area Network}). Por el contrario, algunas de las aplicaciones en los dominios antes mencionados están pensadas para redes con tecnologías como RFID, \textit{Bluetooth }o \textit{ZigBee }. A continuación se presenta la caracterización de cada uno de los dominios que en \parencite{NetTrafficIoT} se consideran viables para redes LPWAN.

\subsection{Ciudades inteligentes \textit{(Smart Cities)}:}

Con la rápida concentración de la poblacion en zonas urbanas, se ha convertido en una prioridad la reducción del uso de recursos públicos, así como la reducción de costos de operación del día a día de una ciudad. Las aplicaciones en este dominio tratan justamente de abordar estos problemas y los servicios que brindan son bastante variados. Los ejemplos van desde el control de luminarias hasta el manejo de desechos, estos y otros  pueden encontrarse en la \textit{Tabla~\ref{tab:smartcity}}, acompañados de más información tal como la caracterización de su tráfico y su demanda de QoS.

\begin{table}
\caption{Características de las aplicaciones de Ciudades Inteligentes}
\label{tab:smartcity}
\centering
\begin{tabular}{*{5}{m{3cm}}}\\
\textbf{\textit{Servicio}} & \textbf{\textit{Tamaño de red}} & \textbf{\textit{Tasa de tráfico}} & \textbf{\textit{Demanda de QoS}} & \textbf{\textit{Fuente de energía}} \\ \hline \hline
\textit{Monitoreo del consumo de agua y electricidad en la ciudad} & \footnotesize{Media a grande, cientos a miles de dispositivos} & \footnotesize{Periódico, 1 msj cada 10 min por dispositivo} & \footnotesize{Baja, tolerante al retardo 1 min} & \footnotesize{Alimentado por la red eléctrica/ autoalimentado} \\ \hline
\textit{Control de iluminación} & \footnotesize{Grande, miles de dispositivos} & \footnotesize{Aleatorio, poco frecuente} & \footnotesize{Media, tolerante al retardo 15 seg} & \footnotesize{Alimentado por la red eléctrica }\\ \hline
\textit{Vigilancia de estacionamientos} & \footnotesize{Grande, miles de dispositivos} & \footnotesize{Aleatorio, poco frecuente} & \footnotesize{Media, tolerante al retardo 10 seg} & \footnotesize{Alimentado por batería }\\ \hline
\textit{Control del tráfico} & \footnotesize{Grande, miles de dispositivos} & \footnotesize{Periódico, 1 msj cada 10 min por dispositivo, aleatorio para alarmas} & \footnotesize{Media, tolerante al retardo 15 seg, alta para alarmas} & \footnotesize{Alimentado por batería} \\ \hline
\textit{Mantenimiento de deshechos} & \footnotesize{Grande, miles de dispositivos} & \footnotesize{Aleatorio, poco frecuente} & \footnotesize{Media, tolerante al retardo 30 seg} & \footnotesize{Alimentado por batería }\\ \hline
\textit{Monitoreo de condiciones urbanas} & \footnotesize{Media a grande, cientos a miles de dispositivos} & \footnotesize{Periódico, 1 msj cada 15 min por dispositivo, aleatorio para alarmas} & \footnotesize{Media, tolerante al retardo 30 seg, alta para alarmas} & \footnotesize{Alimentado por batería }\\ \hline
\textit{Monitoreo de la salud estructural de edificios} & \footnotesize{Media a grande, cientos a miles de dispositivos} & \footnotesize{Periódico, 1 msj cada 15 min por dispositivo, aleatorio para alarmas} & \footnotesize{Media, tolerante al retardo 30 seg, alta para alarmas} & \footnotesize{Alimentado por batería} \\ 
\end{tabular}
\end{table}

\subsection{Ambiente inteligente \textit{(Smart Environment)}}

Este dominio comprende las aplicaciones que se encargan de monitorear lo que ocurre a nuestro alrededor. Una de las ventajas de monitorear el ambiente es el poder reaccionar con antelación a eventos que de otra forma causarían muchos daños. En la \textit{Tabla~\ref{tab:smartenv}} podemos encontrar la caracterización de las aplicaciones consideradas en \parencite{NetTrafficIoT} para Ambiente Inteligente.

\begin{table}
\caption{Características de las aplicaciones de Ambiente Inteligente}
\label{tab:smartenv}
\centering
\begin{tabular}{*{5}{m{3cm}}} \\ 
\textbf{\textit{Servicio}} & \textbf{Tamaño de red} & \textbf{Tasa de tráfico} & \textbf{Demanda de QoS} & \textbf{Fuente de energía} \\ \hline \hline
\textit{Detección de incendios forestales}  & \footnotesize{ Media a grande, cientos a miles de dispositivos } & \footnotesize{ Aleatorio, poco frecuente } & \footnotesize{ Media, tolerante al retardo 15 seg } & \footnotesize{ Alimentado por batería } \\ \hline 
\textit{Detección de terremotos}  & \footnotesize{ Media a grande, cientos a miles de dispositivos } & \footnotesize{ Aleatorio, poco frecuente } & \footnotesize{ Alta, tolerante al retardo 5 seg } & \footnotesize{ Alimentado por batería } \\ \hline 
\textit{Detección de Tsunamis}  & \footnotesize{ Media a grande, cientos a miles de dispositivos } & \footnotesize{ Aleatorio, poco frecuente } & \footnotesize{ Alta, tolerante al retardo 5 seg } & \footnotesize{ Alimentado por batería } \\ \hline 
\textit{Detección de derrumbes y avalanchas}  & \footnotesize{ Media a grande, cientos a miles de dispositivos } & \footnotesize{ Aleatorio, poco frecuente } & \footnotesize{ Alta, tolerante al retardo 5 seg } & \footnotesize{ Alimentado por batería } \\ \hline 
\textit{Monitoreo de actividad volcánica}  & \footnotesize{ Pequeña, 10s de dispositivos } & \footnotesize{ Aleatorio, poco frecuente } & \footnotesize{ Alta, tolerante al retardo 5 seg } & \footnotesize{ Alimentado por batería } \\ \hline 
\textit{Monitoreo de la contaminación del aire } & \footnotesize{ Media a grande, cientos a miles de dispositivos } & \footnotesize{ Periódico, 1 msj cada 15 min por dispositivo } & \footnotesize{ Media, tolerante al retardo 15 seg } & \footnotesize{ Alimentado por batería } \\ \hline 
\textit{Rastreo de vida salvaje.}  & \footnotesize{ Media, cientos de dispositivos } & \footnotesize{ Periódico, 1 msj cada 30 min por dispositivo } & \footnotesize{ Baja, tolerante a unas horas } & \footnotesize{ Alimentado por batería } \\ 
\end{tabular}
\end{table}

\subsection{Energía inteligente \textit{(Smart Energy)}:}

El dominio de Energía Inteligente (\textit{Smart Energy}) contempla mejoras en la distribución y el consumo de fuentes de energía o recursos vitales, tales como la electricidad, el gas y el agua. Actualmente el foco de atención está en la electricidad ya que existe una tendencia más marcada hacia su ahorro y la utilización de fuentes renovables. \newline

Los nodos de IoT que operen dentro de este dominio podrían monitorear las condiciones cambiantes de la red eléctrica, para posteriormente generar una reconfiguración apropiada del servicio. En la \textit{Tabla~\ref{tab:smartenergy}} podemos encontrar la caracterización descrita en \parencite{NetTrafficIoT} para distintas aplicaciones de Energía Inteligente.

\begin{table}
\caption{Características de las aplicaciones de Energía Inteligente}
\label{tab:smartenergy}
\centering
\begin{tabular}{*{5}{m{3cm}}}  \\  
\textbf{\textit{Servicio}} & \textbf{Tamaño de red} & \textbf{Tasa de tráfico} & \textbf{Demanda de QoS} & \textbf{Fuente de energía} \\ \hline \hline
\textit{Medición inteligente}  & \footnotesize{ Media a grande, 1 dispositivo por hogar } & \footnotesize{ Periódico, 1 msj cada 15 min por dispositivo } & \footnotesize{ Media, tolerante al retardo 15 seg } & \footnotesize{ Alimentado por la red eléctrica/ Baterías } \\ \hline 
\textit{Gestión de activos}  & \footnotesize{ Media a grande, cientos a miles de dispositivos } & \footnotesize{ Periódico, 1 msj cada 15 min por dispositivo } & \footnotesize{ Media, tolerante al retardo 15 seg } & \footnotesize{ Alimentado por la red eléctrica/ Baterías } \\ \hline 
\textit{Detección de interrupciones en el servicio } & \footnotesize{ Media a grande, 1 dispositivo por hogar } & \footnotesize{ Aleatorio, poco frecuente } & \footnotesize{ Alta, tiempo real } & \footnotesize{ Alimentado por la red eléctrica/ Baterías } \\ 
\end{tabular}
\end{table}

\subsection{Transporte y movilidad inteligentes \textit{(Smart Transport and Mobility)}:}

Tanto el crecimiento urbano como el crecimiento de las fuentes de transporte de pasajeros y de mercancías, han creado la necesidad de administrar más eficientemente la movilidad en ciudades y carreteras. \newline

El objetivo de aplicaciones IoT en el dominio de Transporte y movilidad inteligentes (\textit{ (Smart Transport and Mobility)}) es ayudar a resolver el problema de movilidad tanto de pasajeros como de mercancías, haciéndolo más rápido, más barato y más seguro. En la \textit{Tabla~\ref{tab:smartrans}} se encuentra la caracterización presentada en \parencite{NetTrafficIoT} para distintas aplicaciones de este dominio.

\begin{table}
\caption{Características de las aplicaciones de Transporte y Movilidad inteligentes}
\label{tab:smartrans}
\centering
\begin{tabular}{*{5}{m{3cm}}}\\ 
\textit{Servicio} & Tamaño de red & Tasa de tráfico & Demanda de QoS & Fuente de energía \\ \hline \hline
\textit{Automatización de vehículos}  & \footnotesize{ Grande, miles de dispositivos } & \footnotesize{ Periódico, 1 msj cada 24 hrs por vehículo. } & \footnotesize{ Baja, tolerante al retardo 1 min } & \footnotesize{ Alimentado por batería del vehículo } \\ \hline 
\textit{Localización y monitoreo de vehículos}  & \footnotesize{ Grande, miles de dispositivos } & \footnotesize{ Periódico, 1 msj cada 30 seg por vehículo } & \footnotesize{ Media, tolerante al retardo 10 seg } & \footnotesize{ Alimentado por batería del vehículo } \\ \hline 
\textit{Monitoreo de la calidad del embarque}  & \footnotesize{ Media, cientos de dispositivos } & \footnotesize{ Periódico, 1 msj cada 15 min por dispositivo } & \footnotesize{ Media, tolerante al retardo 15 seg } & \footnotesize{ Alimentado por batería } \\ \hline 
\textit{Control dinámico de semáforos}  & \footnotesize{ Grande, miles de dispositivos } & \footnotesize{ Periódico, 1 msj cada min por dispositivo } & \footnotesize{ Alta, tolerante al retardo 5 seg } & \footnotesize{ Alimentado por la red eléctrica } \\ \hline 
\textit{Monitoreo de las condiciones del camino} & \footnotesize{ Grande, miles de dispositivos } & \footnotesize{ Aleatorio, poco frecuente } & \footnotesize{ Media, tolerante al retardo 30 seg } & \footnotesize{ Alimentado por batería } \\
\end{tabular}
\end{table}

%----------------------------------------------------------------------------------------
%	SECTION 
%----------------------------------------------------------------------------------------

\section{CARACTERÍSTICAS DEL ESCENARIO A IMPLEMENTAR}\label{AppsEscenario}

Las distintas aplicaciones presentadas en la sección anterior muy seguramente se verán desplegadas en un futuro próximo en ciudades alrededor de todo el mundo. Una aserción que resulta importante es la gran cantidad de aplicaciones que se esperan para IoT \parencite{Ericsson2019}, pero en este análisis se tomaron en consideración únicamente aquellas que operan en una red LPWAN. Para estas aplicaciones, una red celular de 5G podría ser las más idónea para brindarles servicio. \newline

En las aplicaciones seleccionadas, lo primordial es tener un bajo consumo y complejidad de los dispositivos, además de una amplia cobertura y permitir una gran densidad de dispositivos \parencite{5GAmericas}. Esto contrasta con las aplicaciones más inclinadas a los casos de uso eMBB y URLLC, para las cuales lo primordial es el amplio ancho de banda disponible para las aplicaciones y una menor latencia, respectivamente.

\subsection{Análisis de las aplicaciones de IoT y selección de casos considerados}

De todos los servicios descritos en las tablas anteriores \textit{[Tabla~\ref{tab:smartcity}\ldots \ref{tab:smartrans}]}, se decidió implementar solamente un grupo diverso de aplicaciones provenientes de distintos dominios. Este grupo se seleccionó de tal manera que fuera representativo de distintos comportamientos y requerimientos de QoS. El escenario propuesto consta de los servicios de: control de iluminación, monitoreo del consumo de agua y electricidad, detección de terremotos, monitoreo de contaminación del aire, detección de interrupciones en el servicio (agua, luz, gas), control dinámico de semáforos y un último servicio que se llamó ``otros dispositivos mMTC''. Este último representó el conglomerado de todas las demás aplicaciones IoT que estarán presentes en la red y no corresponden a una de las primeras aplicaciones seleccionadas. \newline

Se decidió entonces situar la simulación en un escenario urbano macro celular (UMa) con nodos en exteriores, en el que se encontrarán todas las aplicaciones mencionadas en el párrafo anterior. El escenario urbano macro representa a zonas urbanas densamente pobladas. La elección de un escenario urbano macro celular se debió a que en este se puede experimentar una congestión producida por las  MTC y HTC más facilmente. Otra razón fue que en tal escenario se tiene presente una mayor diversidad de aplicaciones IoT, con distintas características de tráfico y requerimientos de QoS.\newline

En la \textit{Tabla~\ref{tab:appssim}} se presentan las aplicaciones IoT del escenario urbano que se implementaron.\newline

A continuación se hace una descripción de la \textit{Tabla~\ref{tab:appssim}}. Se tienen tres aplicaciones con una alta demanda de QoS, después dos con una demanda media y finalmente una con demanda baja. Las distintas demandas de QoS se traducen en diferentes tolerancias a la latencia, las cuales van desde los minutos hasta aquellas que requieren de una respuesta en el orden de segundos. Otra característica que se consideró para decidir entre las aplicaciones fue su tráfico, se trató entonces de tener nodos con distintos periodos de transmisión, por ejemplo el servicio que brindan nodos monitoreando la contaminación del aire tiene un periodo de 15 minutos, mientras que el de nodos controlando los semáforos es de 1 minuto. Existen también nodos con tasas de trasmisión aleatorias como el caso de los nodos de detección de terremotos. \newline

La decisión de agregar un servicio más (Otros dispositivos mMTC) que fuera genérico surge a raíz de la necesidad de representar en la red los dispositivos restantes con comportamientos de lo más diversos.\newline

Finalmente se consideraron también dispositivos URLLC en el simulador. Esto permitió realizar un agrupamiento de dispositivos mMTC y URLLC para que pudieran compartir recursos. Los detalles de la implementación son discutidos en los siguientes capítulos.

\begin{table}
\caption{Aplicaciones seleccionadas para la simulación}
\label{tab:appssim}
\centering
\begin{tabular}{*{5}{m{3cm}}} \\ 
\textbf{\textit{Servicio}} & \textbf{Tamaño de red} & \textbf{Tasa de tráfico} & \textbf{Demanda de QoS} & \textbf{Fuente de energía} \\ \hline \hline
\textit{Control de iluminación (Ciudad Inteligente)}  & \footnotesize{ Grande, miles de dispositivos } & \footnotesize{ Aleatorio, poco frecuente } & \footnotesize{ Media, tolerante al retardo 15seg } & \footnotesize{ Alimentado por la red eléctrica } \\ \hline 
\textit{Monitoreo del consumo de agua y electricidad en la ciudad (Ciudad Inteligente)}  & \footnotesize{ Media a grande, cientos a miles de dispositivos } & \footnotesize{ Periódico, 1 msj cada 10 min por dispositivo } & \footnotesize{ Baja, tolerante al retardo 1min } & \footnotesize{ Alimentado por la red eléctrica/ autoalimentado } \\ \hline 
\textit{Detección de terremotos (Ambiente Inteligente)}  & \footnotesize{ Media a grande, cientos a miles de dispositivos } & \footnotesize{ Aleatorio, poco frecuente } & \footnotesize{ Alta, tolerante al retardo 3seg } & \footnotesize{ Alimentado por batería } \\ \hline 
\textit{Monitoreo de contaminación del aire (Ambiente Inteligente) } & \footnotesize{ Media a grande, cientos a miles de dispositivos } & \footnotesize{ Periódico, 1 msj cada 15 min por dispositivo } & \footnotesize{ Media, tolerante al retardo 15seg } & \footnotesize{ Alimentado por batería } \\ \hline 
\textit{Control dinámico de semáforos (Transporte y Movilidad Inteligentes)} & \footnotesize{ Grande, miles de dispositivos } & \footnotesize{ Periódico, 1 msj cada min por dispositivo } & \footnotesize{ Alta, tolerante al retardo 5seg } & \footnotesize{ Alimentado por la red } \\ \hline 
\textit{Otros dispositivos mMTC}  & \footnotesize{ Grande, miles de dispositivos } & \footnotesize{ Aleatorio, poco frecuente } & \footnotesize{ Alta, tolerante al retardo 5seg } & \footnotesize{ Alimentado por batería } \\ 
\end{tabular}
\end{table}


\subsection{Análisis de las tecnologías para IoT y selección de casos considerados}

NB-IoT y LTE-M (\textit{LTE for MTC}) son dos tecnologías LPWA desarrolladas para aplicaciones IoT. Ambas tecnologías contemplan comunicaciones celulares con un ancho de banda bajo, y conectan dispositivos que necesitan transmitir pequeñas cantidades de datos, a bajo coste. Debido a que se espera que NB-IoT y LTE-M cumplan con los requerimientos LPWAN del estándar 5G, 3GPP indicó a ITU-R que ambas tecnologías formarán parte del estándar. Estas tecnologías coexistirán con los demás componentes de 5G NR que en conjunto brindarán servicio a los distintos casos de uso. Esto convierte a NB-IoT y LTE-M parte de 5G \parencite{EricssonAB2016}. De entre las dos tecnologías, la que mejor satisface los requerimientos de las aplicaciones seleccionadas en la sección anterior es NB-IoT como se desarrolla a continuación.\newline

Las redes de largo alcance LPWAN utilizan tecnología capaz de transferir mensajes a decenas de kilómetros de distancia y cubrir una amplia área. Son redes especializadas en interconectar dispositivos en ambientes restringidos, de difícil acceso o que simplemente buscan reducir el consumo de energía en estos, manteniendo un bajo costo y complejidad. Por lo tanto se concentran en un eficiente consumo de la energía y una cobertura amplia \parencite{NetTrafficIoT}. Esto se acopla bastante bien a lo que requieren los nodos de IoT pertenecientes a las aplicaciones seleccionadas en la sección anterior. \newline

Las tecnologías que darán servicio a dispositivos IoT son entonces:

\begin{enumerate}
    \item \textit{\underbar{eMTC (enhanced Machine Type Communication:)}} Forma parte de la familia LTE-M y es una evolución de LTE optimizada para IoT. Se desarrolló con el objetivo de una eficiencia energética.
    \item \textit{\underbar{NB-IoT (Narrow-Band Internet of Things:)}} fue estandarizado en el \textit{release }13 de 3GPP y se espera que consiga dar servicio a más dispositivos con energía limitada que eMTC. NB-IoT no requiere ningún desarrollo adicional de redes ya que se implementa en funcionalidades ya existentes de LTE \parencite{NetTrafficIoT} y en un futuro será implementado directamente en la banda de frecuencias de 5G NR [\textit{véase Figura~\ref{fig:5gnr}}] \parencite{EricssonAB2016}.
\end{enumerate}

El soporte para IoT masivo (mIoT) ya se brinda en las redes LTE de hoy con NB-IoT y eMTC. Estas tecnologías se complementan entre sí y existe una tendencia emergente de los proveedores de servicios de implementar una red común que admita ambas tecnologías. eMTC es adecuada para casos de uso que requieren un rendimiento relativamente más alto, una latencia más baja y soporte de voz, mientras que la tecnología NB-IoT es conveniente para casos de uso que toleren demoras pero requieren una cobertura extendida \parencite{EricssonAB2016}. \newline

Según \parencite{Ericsson2019} a finales de 2024, se espera que NB-IoT y eMTC representen cerca del 45 por ciento de todas las conexiones CIoT (\textit{Celullar IoT}). Además, en el futuro NB-IoT y eMTC podrán coexistir completamente en las bandas de frecuencia de 5G NR, \textit{Figura~\ref{fig:5gnr}}.\newline

\begin{figure}[th]
\centering
\includegraphics[scale=1]{Figures/5G NR con LTE-M y NB-IoT en banda}
\decoRule
\caption[5G NR con LTE-M y NB-IoT en banda]{5G NR con LTE-M y NB-IoT en banda}
\label{fig:5gnr}
\end{figure}

En la \textit{Tabla~\ref{tab:tecIoT}} se pueden encontrar características de estas tecnologías antes descritas, tales como la banda de frecuencia a la que operan y su tasa de transmisión. Si bien pareciera que la diferencia entre ambas tecnologías es sutil, en realidad, esta marca una clara pauta en el servicio que pueden brindar. \newline

\begin{table}
\caption{Características de las tecnologías de red para IoT en la red celular}
\label{tab:tecIoT}
\centering
\begin{tabular}{|p{0.9in}|p{0.6in}|p{0.4in}|p{0.6in}|p{0.4in}|p{0.8in}|p{1.3in}|p{0.4in}|} \\ \hline \hline
\textbf{\textit{Tecnología}} & \textbf{\textit{Banda de Frecuencia}} & \textbf{\textit{Rango}} & \textbf{\textit{Tasa de transmisión}} & \textbf{\textit{Vida de la batería}} & \textbf{\textit{Topología}} & \textbf{\textit{Estandarización}} & \textbf{\textit{Grupo}} \\ 
\textbf{NB-IoT}  & \footnotesize{ 450 MHZ -- 3.5 GHz (Espectro de 2G/3G/4G) } & \footnotesize{ 10-15 km } & \footnotesize{ 250 kbps } & \footnotesize{ 10+ años } & \footnotesize{ Estrella } & \footnotesize{ Abierta } & \footnotesize{ 3GPP } \\ \hline
\textbf{eMTC}  & \footnotesize{ 450 MHZ -- 3.5 GHz (El mismo que LTE) } & \footnotesize{ 10-15 km } & \footnotesize{ 1 Mbps } & \footnotesize{ 10+ años } & \footnotesize{ Estrella } & \footnotesize{ Abierta } & \footnotesize{ 3GPP } \\
\end{tabular}

\end{table}

En la \textit{Figura~\ref{fig:lpwa}} se puede observar otra comparación entre ambas tecnologías pero en esta ocasión se comparan las aplicaciones a las que  NB-IoT y LTE-M estarían dando servicio preferentemente. A la izquierda de la \textit{Figura~\ref{fig:lpwa}} tenemos las aplicaciones LPWAN a las que NB-IoT daría servicio. Estas aplicacioness coinciden con una menor velocidad de transferencia y mayor tolerancia a la latencia mientras que a la derecha se aglomeran las aplicaciones que requieren una comunicación de baja latencia y una mayor tasa de transmisión. A estas últimas aplicaciones les estaría dando servicio preferentemente la tecnología eMTC. \newline

Se decidió entonces concentrarse en la tecnología NB-IoT puesto que la totalidad de los servicios que se considerarán en nuestro sistema pueden situarse a la izquierda de la \textit{Figura~\ref{fig:lpwa}}, donde se presenta una mínima movilidad de los dispositivos. Por ejemplo el control de la iluminación y el control dinámico de los semáforos podríamos colocarlos en \textit{Iluminación pública y Ciudades inteligentes }respectivamente, mientras que el monitoreo de consumo energético y el de la condición del aire podrían corresponder a \textit{Medidores inteligentes}, de manera que el único servicio que se situa en los límites de la tecnología NB-IoT o del otro lado incluso es el de dispositivos URLLC. Debido a esto se seleccionó a NB-IoT como la tecnología fundamental para diseñar el simulador a la que después se agregaron mejoras propuestas en otros trabajos.  \newline


\begin{figure}[th]
\centering
\includegraphics[scale=1]{Tecnología líder para el caso de uso LPWA}
\decoRule
\caption[Tecnologías líderes para el caso de uso LPWA]{Tecnologías líder para el caso de uso LPWA, [Fuente: https://www.iotforall.com/cellular-iot-explained-nb-iot-vs-lte-m/]}
\label{fig:lpwa}
\end{figure}


La \textit{Figura~\ref{fig:5gqos}} muestra las distintas tecnologías con las que 5G NR estaría trabajando para poder brindar servicio al amplio espectro de casos de uso de MTC \parencite{5GAmericas}. La tecnología NB-IoT podemos situarla en las frecuencias de operación baja y con un una tolerancia al retardo mayor que la mayoría de las demás tecnologías.

\begin{figure}[th]
\centering
\includegraphics[scale=1]{5G NR soportará múltiples servicios con distintos requerimiento de QoS}
\decoRule
\caption[5G NR soportará múltiples servicios con distintos requerimiento de QoS]{5G NR soportará múltiples servicios con distintos requerimiento de QoS}
\label{fig:5gqos}
\end{figure}

%----------------------------------------------------------------------------------------
%	SECTION 
%----------------------------------------------------------------------------------------

\section{ANÁLISIS DEL ESTÁNDAR NB-IoT} \label{NBIoT}

El estándar NB-IoT fue especificado en el reporte TR 45.820 (\textit{release} 13) de la 3GPP \parencite{3GPP2019}. Sus parámetros fundamentales son:\newline

Para el enlace de subida (\textit{uplink}), como su nombre lo indica, tiene un ancho de banda estrecho de 180 kHz y un espacio de sub-portadora de 3.75 kHz (ancho de banda de transmisión mínimo para un dispositivo). Por lo tanto puede asignar 48 sub-portadoras [\textit{véase Figura~\ref{fig:NBIoT}}].

\begin{figure}[th]
\centering
\includegraphics[scale=.7]{Estructura de ancho de banda y subportadoras en NB-IoT}
\decoRule
\caption[Estructura de ancho de banda y subportadoras en NB-IoT.]{Estructura de ancho de banda y subportadoras en NB-IoT.}
\label{fig:NBIoT}
\end{figure}

En el enlace de bajada (\textit{downlink}), se conserva la estructura de transmisión del enlace descendente de \textit{Long Term Evolution} (LTE) con un espaciado de sub-portadora de 15 kHz. Por lo tanto, NB-IoT puede proporcionar velocidades de datos de casi 250 kb / s en el enlace descendente y 20 kb / s en el enlace ascendente.\newline

Es preciso puntualizar que para lograr una mayor tasa de datos, de acuerdo con el Teorema de Shannon-Hartley (Ecuación~\ref{eqn:Shannon}), el ancho de banda debe ser elevado o se debe tener una relación S/N alta. Para el caso de NB-IoT se cuenta con un ancho de banda muy pequeño (3.75KHz), por lo cual alcanzar una buena relación S/N (S/I para sistemas celulares) es de suma importancia.\\

\subsection{Modos de operación}

NB-IoT puede implementarse como una portadora autónoma utilizando cualquier espectro disponible con un ancho de banda superior a 180 kHz. Esto se conoce como la implementación stand-alone. Un caso de uso de este despliegue autónomo es que un operador GSM despliegue NB-IoT en su banda GSM reajustando parte de su espectro \parencite{Liberg2018}.\newline


NB-IoT también está diseñado para su despliegue en las redes LTE existentes, ya sea utilizando uno de los bloques de recursos físicos (PRB) de LTE o utilizando la banda de guarda LTE.\newline

Para el modo de operación independiente y de banda de guarda, el PRB de enlace descendente y ascendente debe establecerse simétricamente y para el modo en banda, el despliegue del PRB estará restringido a algunos prefijos de PRB’s de acuerdo al ancho de banda LTE, (ya sea 3, 5, 10, 15 o 20 MHz.) esto debido a la sincronización entre el UE y la eNB \parencite{NBIoTDeploymentGSMA}.

\begin{figure}[th]
    \centering
    \includegraphics[scale=.5]{modooperacionNBIOT}
    \decoRule
    \caption[Modos de Operación en NB-IoT.]{Modos de Operación en NB-IoT. \parencite{Liberg2018}}
    \label{fig:NBIoT2}
\end{figure}


\subsection{Clases de Potencia}

Algunas aplicaciones de IoT son particularmente sensibles al consumo de energía. Para minimizar el impacto de la conectividad en la duración de la batería del dispositivo, en el \textit{release} 13, se determinó que los UE podrán usar dos opciones de clase de potencia. Uno es el nivel de potencia del dispositivo móvil LTE tradicional de \textbf{23dBm} (\textit{Power Class 3}) y uno nuevo, con menos potencia de salida, de \textbf{20dBm} (\textit{Power Class 5}). El \textit{release} 14 de 3GPP agrega una nueva clase de potencia aún menor, de \textbf{14dBm} (\textit{PowerClass 6}) \parencite{NBIoTDeploymentGSMA}.

\subsection{Modos de transmisión en enlace ascendente (UL) }
Existen dos modos de transmisión para el enlace UL, en el modo \textit{singletone} sólo se puede asignar una subportadora a cada dispositivo NB-IoT, por el contrario en el modo \textit{multitone}, la agregación de subportadoras es posible, esto con el fin de alcanzar una mayor tasa de transmisión \parencite{RohdeNB}. En \parencite{Shahini2019} se evaluó el uso \textit{multitone} en anchos de banda de subportadoras de 3.75 kHz, y es este escenario el que se evaluó en este trabajo puesto que permitiría dar servició a una mayor cantidad de dispositivos al mismo tiempo.\newline

Finalmente, se admiten los modos de \textit{singletone} y \textit{multitone}, cuando el PRB se divide en 12 subportadoras, cada una de ellas con un ancho de banda de 15 kHz. En \parencite{Mostafa2019} se estudió más a fondo el uso de \textit{singletone} y \textit{multitone} en transmisiones UL.\newline

\subsection{Características del tráfico IoT}

En los Reportes autónomos móviles (MAR, \textit{Mobile Autonomous Reporting}) se definen cuatro tipos de paquetes diferentes:

\subsubsection{Informes de excepción}

Se espera que muchas aplicaciones de tipo sensor monitoreen una condición física y activen un informe de excepción cuando se detecte un evento en específico. Estos eventos serán, en general, raros y ocurrirán cada par de horas, pocos días o incluso meses. Ejemplos de tales paquetes incluyen los producidos por detectores de humo, notificaciones de fallas de energía de medidores inteligentes, notificaciones de manipulación, etc.\newline


\subsubsection{Informes periódicos}\label{Informesperiodicos}

Se espera que los informes periódicos de enlace ascendente sean comunes para aplicaciones de IoT celular como informes de medición de servicios inteligentes (gas / agua / electricidad), agricultura inteligente, entorno inteligente, etc. El modelo de tráfico de informes de enlace ascendente periódico MAR se utiliza en simulaciones a nivel de sistema para análisis de capacidad.\newline

El tamaño de la carga útil sigue una distribución de Pareto con parámetro $\alpha$ = 2.5 y tamaño mínimo de carga útil = 20 bytes con un corte en 200 bytes, es decir, las cargas superiores a 200 bytes serán limitadas a 200 bytes.\newline

Una vez revisadas estas clasificaciones de tráfico compatibles para NB-IoT, se asignaron los modelos de tamaño de paquete a los servicios seleccionados en la \textit{Tabla~\ref{tab:appssim}} de la sección anterior.\newline

La adición de la columna "Tamaños de paquete"  para la \textit{Tabla~\ref{tab:appssim}} se da en la \textit{Tabla~\ref{tab:trafpkt}}.
\begin{table}
\caption{Caracterización del tráfico de paquetes en aplicaciones seleccionadas para la simulación.}
\label{tab:trafpkt}
\centering
\begin{tabular}{*{2}{m{7cm}}}\\ 
\textbf{\textit{Servicio}} & \textbf{Tamaño de paquetes} \\ \hline \hline
\textit{Control de iluminación (Smart City) } & \footnotesize{ Activación aleatoria \textbf{UL}: 20 bytes \textit{payload} \textbf{DL}: ACK de 0 bytes } \\ \hline 
\textit{Monitoreo del consumo de agua y electricidad en la ciudad (Smart City) } & \footnotesize{ Activación periódica \textbf{UL}: distribución de Pareto con parámetro alfa = 2.5 y tamaño mínimo de carga útil de la aplicación = 20 bytes con un corte a 200 bytes \textbf{DL}: ACK de 0 bytes 50\% de las veces. } \\ \hline 
\textit{Detección de terremotos (Smart Environment)}  & \footnotesize{ Activación aleatoria \textbf{UL}: 20 bytes \textit{payload} \textbf{DL}: ACK de 0 bytes } \\ \hline 
\textit{Monitoreo de contaminación del aire (Smart Environment) } & \footnotesize{ Activación periódica \textbf{UL}: distribución de Pareto con parámetro alfa = 2.5 y tamaño mínimo de carga útil de la aplicación = 20 bytes con un corte a 200 bytes \textbf{DL}: ACK de 0 bytes 50\% de las veces. } \\ \hline 
\textit{Control dinámico de semáforos (Smart Transport and Mobility)}  & \footnotesize{ Activación aleatoria \textbf{UL}: distribución de Pareto con parámetro alfa = 2.5 y tamaño mínimo de carga útil de la aplicación = 20 bytes con un corte a 200 bytes \textbf{DL}: ACK de 0 bytes 50\% de las veces. } \\ \hline 
\textit{Otros dispositivos mMTC}  & \footnotesize{ Activación aleatoria \textbf{UL}: 20 bytes \textit{payload} \textbf{DL}: ACK de 0 bytes } \\  
\end{tabular}
\end{table}

\subsection{Indicadores clave de rendimiento (KPIs)}

Debido a que 5G representará un cambio radical a las generaciones anteriores, se puede prever que habrá nuevos indicadores de evaluación. El diseño de estos indicadores directamente medibles, por un lado, necesita combinar las características de los nuevos servicios, y por otro lado, debe aprender completamente de la experiencia de los KPI clásicos de generaciones anteriores como lo son: el \textit{throughput, }dada una probabilidad de salida y la latencia. Además la densidad de conexión, la densidad de volumen de tráfico y el consumo de energía son KPI's que preocupan también a las redes 5G/IoT \parencite{WirelessSim}.\newline

Para cumplir con el conjunto de requisitos de mMTC, NB-IoT debe admitir principalmente cuatro indicadores clave de rendimiento (KPI):

\begin{enumerate}
\item  Vida útil de la batería del dispositivo más allá de 10 años, suponiendo una capacidad de energía almacenada de 5 Wh.
\item  Densidad de conexión masiva de hasta 1M dispositivos por km cuadrado en un entorno urbano.
\item  Latencia de como máximo 10 s.
\item  Una tasa máxima alcanzable de hasta 200kbps (subida).
\end{enumerate}

De los KPIs antes enumerados este simulador obtuvo resultados acerca de la densidad de conexión masiva de dispositivos y de las tasas objetivo alcanzadas por estos.

\break
%----------------------------------------------------------------------------------------
%	SECTION 
%----------------------------------------------------------------------------------------

\section{ANÁLISIS DE MODELOS PARA LA EVALUACIÓN DE REDES 5G/IoT}\label{AnalisisMODELOS}
Los modelos necesarios para caracterizar el comportamiento de redes móviles, desde el punto de vista de simulaciones a nivel de sistema, se presentan a continuación \parencite{WirelessSim}:
\begin{enumerate}
    \item  Modelo de despliegue de BSs y UEs.
    \item  Modelo de antenas (MIMO, MISO, entre otras) y formación de haz.
    \item  Modulación y codificación.
    \item  Modelo de canal.
    \item  Patrones de Movilidad.
    \item  Calendarizadores (planificadores de recursos).
    \item  Esquema de acceso múltiple al medio.
    \item  Modelos de tráfico.
\end{enumerate}


Los modelos que se incluyeron en el simulador se describen en las siguientes subsecciones:

\subsection{MODELO DE DESPLIEGUE DE BSs Y UEs}

Los modelos del posicionamiento de las estaciones base y los nodos IoT utilizan diferentes estrategias de despliegue (como se puede ver en la \textit{Figura~\ref{fig:BSs}}). Este aspecto, como los demás, es de mayor o menor importancia dependiendo de los objetivos de la simulación, es decir, para un alcance comercial resulta importante simular el despliegue determinístico de los actores de la red. Por otro lado para un alcance con fines de análisis en el diseño y dimensionamiento de estas redes resulta más adecuado un despliegue aleatorio. \newline

Diversos autores coinciden en que varias distribuciones de redes móviles siguen un proceso estocástico \parencite{Kouzayha2018}\parencite{Zhang2017}. La geometría estocástica es una rama de la probabilidad con muchas aplicaciones que permiten el estudio de fenómenos aleatorios en el plano o en dimensiones superiores \parencite{Haenggi2009}. Recientemente se ha utilizado con éxito para modelar la distribución espacial de células pequeñas como las femtoceldas \parencite{TurjmanSmallCells}. \newline

En este proyecto con fines de diseño y análisis, se simularon ambos despliegues, uno uniforme y otro siguiendo una geometría estocástica, es decir, un PPP.

\subsection{MODELO DE CANAL}

Un buen diseño de sistema necesitó de conocer las características del canal de propagación a través de las frecuencias de microondas y ondas milimétricas. \newline

Los modelos de canal son necesarios para simular la propagación de una manera reproducible y rentable, y se utilizan para diseñar y comparar con precisión las interfaces de radio y el despliegue del sistema. Los parámetros comunes del modelo de canal inalámbrico incluyen frecuencia de portadora, ancho de banda, distancia entre el transmisor (Tx) y el receptor (Rx), los efectos ambientales y otros requisitos necesarios. El desafío definitivo para un modelo de canal 5G es proporcionar una base física fundamental, a la vez flexible y precisa, especialmente en un amplio rango de frecuencias como 0.5--100 GHz \parencite{Rappaport2017}. Los modelos de canal investigados se dividen principalmente de acuerdo al escenario en el que se están diseñando, ya sea \textit{Urban Macro (UMa) o Urban Micro (UMi)}, además de la condición del ambiente si es que hay línea de vista (\textit{LoS}) entre el UE y la BS.\newline

Existe una gama amplia de modelos de canal propuestos para redes 5G, (p.ej. 3GPP, WINNER I/II, QuaDRiga/ mmMagic, 5GCM, METIS, MiWEBA, IEEE \parencite{WirelessSim}). Aunque existen diversos modelos, los modelos de canal 3GPP y WINNER II son los más conocidos y empleados en la industria de comunicaciones móviles \parencite{Sun2016}, conteniendo una gran diversidad de escenarios de despliegue como lo son \textit{UMi, UMa, indoor office (InH)}, etc. Además proveen parámetros clave del canal incluyendo probabilidades de línea de vista (\textit{LoS}), modelos de pérdida por trayectoria, retardos y niveles de potencia por trayectoria \parencite{Sun2016}. \newline

El modelo implementado se trató de uno teórico y estocástico, en vez de uno empírico, ya que este proyecto está enfocado en el teletráfico. Los modelos empíricos suelen ser más sofisticados y piden una gran cantidad de parámetros de entrada. Por lo tanto se eligieron modelos estocásticos que se adaptaran al rango de frecuencia de transmisión y a los ambientes urbanos que se propusieron.\newline

En \parencite{Sun2016}, los autores evaluaron tres diferentes modelos de propagación estocásticos de perdida por trayectoria a larga escala para ser implementados a través de la banda de frecuencias de microondas y ondas milimétricas. ABG, CI y CIF son modelos estadísticos de propagación para multi-frecuencias (estocásticos) que describen los parámetros de larga escala con pérdida de trayectoria de acuerdo a la distancia.

Los modelos evaluados fueron: 
\begin{enumerate}
\item  ABG: Modelo Alpha-Beta-Gamma.
\item  CI: Modelo de pérdida por trayectoria de distancia de referencia de espacio libre cercano.
\item  CIF: Modelo CI con un exponente de pérdida de trayectoria ponderado en frecuencia.
\end{enumerate}

\begin{flushleft}
 Para el primero, la ecuación del modelo ABG está dada por:
 \begin{equation}
    L^{ABG}_p(f,d)_{\left[dB\right]}=10 \alpha {\ log}_{10}\left(\frac{d}{1m}\right)+\beta +\ 10\gamma {\ log}_{10}\left(\frac{f}{1GHz}\right)+\ x^{ABG}_{\sigma .}, donde\ d\ge 1m
    \label{eqn:ABG}
\end{equation}
\[\alpha \to coeficiente\ que\ representa\ la\ dependencia\ de\ la\ perdida\ por\ trayectoria\ con\ la\ distancia\] 
\[\gamma \to coeficiente\ que\ representa\ la\ dependencia\ de\ la\ perdida\ por\ trayectoria\ con\ la\ frecuencia\] 
\[\beta \to es\ un\ valor\ de\ compensación\ para\ la\ pérdida\ por\ trayectoria\ (en\ dB's)\] 
\[x^{ABG}_{\sigma .}\to es\ una\ variable\ aleatoria\ gaussiana\ de\ media\ cero\ con\ una\ desviaci\textrm{ó}n\ est\textrm{á}ndar \]
\[sigma \ [dB], que\ describe\ las\ fluctuaciones\ de\ se\textrm{ñ}al\ a\ gran\ escala\ (es\ decir,\ ensombrecimiento)\]

Para el segundo, la ecuación del modelo CI está dada por:
\begin{equation}
    L^{CI}_p(f,d)_{\left[dB\right]}=32.4+\ 10\ n{\ log}_{10}\left(\frac{d}{d_0}\right)+{\ 20\ log}_{10}\left(d_0\right)+{20\ log}_{10}\left(f\right)+x^{CI}_{\sigma .}, donde\ d\ge d_0
    \label{eqn:CI}
\end{equation}
\[x^{CI}_{\sigma .}\to es\ una\ variable\ aleatoria\ gaussiana\ de\ media\ cero\ con\ una\ desviaci\textrm{ó}n\ est\textrm{á}ndar \]
\[sigma \ [dB], que\ describe\ las\ fluctuaciones\ de\ se\textrm{ñ}al\ a\ gran\ escala\ (es\ decir,\ ensombrecimiento)\]

Para el tercero, la ecuación del modelo CIF está dada por:
\begin{equation}
    L^{CIF}_p(f,d)_{\left[dB\right]}=32.4+\ 10\ n{\left(1+b \left(\frac{f-f_0}{f_0}\right) \right)log}_{10}\left(d\right)+{20\ log}_{10}\left(f\right)+x^{CIF}_{\sigma .},  donde\ d\ge 1m
    \label{eqn:CIF}
\end{equation}
\[x^{CIF}_{\sigma .}\to es\ una\ variable\ aleatoria\ gaussiana\ de\ media\ cero\ con\ una\ desviaci\textrm{ó}n\ est\textrm{á}ndar  \]
\[ sigma \ [dB], que\ describe\ las\ fluctuaciones\ de\ se\textrm{ñ}al\ a\ gran\ escala\ (es\ decir,\ ensombrecimiento)\]
\end{flushleft}

Cada uno de estos modelos han sido recientemente estudiados por organizaciones de estandarización como 3GPP y son propuestos para su uso en el diseño de sistemas inalámbricos de comunicación de 5G enfocados en escenarios \textit{UMa, UMi, InH,y SM}.\newline

De acuerdo al análisis de sensibilidad en \parencite{Sun2016}, se demostró que el modelo CI es el más adecuado para entornos al aire libre debido a su precisión, simplicidad y rendimiento de sensibilidad, dado que la pérdida de trayectoria medida depende poco de la frecuencia en ambientes exteriores más allá del primer metro de propagación de espacio libre.\newline

Por otro lado, el modelo CIF es muy adecuado para entornos interiores, ya que proporciona una desviación estándar más pequeña que el modelo ABG en muchos casos, incluso con menos parámetros del modelo y tiene una precisión superior cuando se analiza con el análisis de sensibilidad.\newline

Los modelos CI y CIF son más robustos y precisos en comparación con el modelo ABG, por lo que es confiable la aplicación del modelo CI para simular entornos en exteriores y el CIF para interiores \parencite{Sun2016}.\newline

De acuerdo a lo propuesto en la \textit{sección~\ref{AppsEscenario} }, el ambiente macro urbano (UMa) que consideramos está dirigido a un entorno en exteriores. Por lo que se seleccionó al modelo CI (Modelo de pérdida por trayectoria de distancia de referencia de espacio libre cercano). \newline

Los parámetros que requiere este modelo (\textit{Ecuación~\ref{eqn:CI}}) son: la distancia entre la BS y el UE, la frecuencia fundamental de operación y una variable aleatoria de media cero con una desviación estándar $\sigma$ [dB], que describirá las fluctuaciones de señal a gran escala (es decir, ensombrecimiento[\textit{shadowing}]).\newline

Sin embargo, para este modelo de canal no se implementaron las pérdidas por ensombrecimiento, se optó por incorporar las perdidas por multitrayectoria (\textit{multipath}) ya que en el caso del estándar NB-IoT, que usa anchos de banda pequeños, se espera que sea suceptible a las variaciones rápidas del canal. Por lo tanto, en lugar de utilizar la variable aleatoria gaussiana propia del modelo, se implementó el desvanecimiento rápido incorporando una variable aleatoria tipo Rayleigh.\newline

Por último, los autores proponen a $d_{0} = 1 m$ en los modelos de pérdida por trayectoria para sistemas 5G ya que se espera que las distancias de cobertura serán más cortas a frecuencias más altas. Además, lo que se espera son futuras celdas pequeñas, es probable que las BS se monten más cerca de las obstrucciones. La estandarización a una distancia de referencia de 1 m simplifica las comparaciones de mediciones y modelos y proporciona una definición estándar para el PLE, al tiempo que permite la intuición y el cálculo rápido de la pérdida de trayectoria.

\subsection{ESQUEMA DE ACCESO MÚLTIPLE AL MEDIO}

\subsubsection{Acceso Múltiple No Ortogonal (NOMA)}\label{NOMA_C4}

Se estudió en la Sección~\ref{NOMA_C2}, el uso de NOMA y su soporte eficientemente de conectividad masiva. El diseño de NOMA en transmisiones de subida (\textit{uplink}) ha sido propuesto en \parencite{Al-Imari2014} y el diseño óptimo de NOMA en transmisiones de bajada (\textit{downlink}) ha sido propuesto en \parencite{Zhu2019}.\newline

En \parencite{Zhang2017} se implementó NOMA emparejando selectivamente dos usuarios, es decir, se escogía a un usuario con una condición de canal muy buena (cerca de la BS) y otro con una condición de canal muy pobre (en el borde de la celda). Por otro lado en \parencite{Shahini2019}, se implementó NOMA usando la técnica de agrupamiento de usuarios, considerando un entorno donde conviven dispositivos mMTC y URLLC (estos tienen mayores requisitos de tasa en comparación con los dispositivos mMTC), de igual manera se agrupan con otros usuarios (p.ej., 2, 3 o 4 usuarios) de diferente tipo (mMTC y URLLC) y se ordenan convenientemente para implementar SIC.\newline

NB-IoT no es capaz de proveer conectividad a una cantidad masiva de dispositivos IoT como se espera en el futuro, así para el diseño se consideró la metodologia en \parencite{Shahini2019}, donde los dispositivos activos de URLLC y mMTC comparten un PRB para la transmisión de datos en el enlace ascendente. El ancho de banda disponible se divide en un conjunto de frecuencias (subcanales) $S$. De hecho, el ancho de banda del sistema se puede dividir por igual en 48 o 12 subportadoras en los sistemas NB-IoT. La relación de la distribución de dispositivos mMTC con los URLLC se fijó a 3 a 1. Todo esto con base en las consideraciones del modelo de sistema en \parencite{Shahini2019}.\newline

En \parencite{Shahini2019}, se utiliza la nomenclatura que se enlista a continuación para desarrollar su propuesta, misma que se utilizó en este documento:\label{VariablesSHahini}

\begin{itemize}
    \item U: Lista de dispositivos uRLLC
    \item M: Lista de dispositivoss mMTC
    \item S: Lista de Subportadoras s
    \item C: Lista de Grupos NOMA
    \item $k_{max}$ Rango máximo de usuarios en un grupo NOMA
    \item $R_{m}^{th}:$ Tasa objetivo del enésimo dispositivo m mMTC 
    \item $R_{u}^{th}:$ Tasa objetivo del enésimo dispositivo u uRLLC
    \item $P_{m}^{max}:$ Potencia máxima del enésimo dispositivo m mMTC 
    \item $P_{u}^{max}:$ Potencia máxima del enésimo dispositivo u uRLLC (i.e. 23dBm)
    \item $P_{m}^{s}:$ Potencia del enésimo dispositivo m mMTC 
    \item $P_{u}^{s}:$ Potencia del enésimo dispositivo u uRLLC 
    \item $h_{m}^{s}:$ Ganancia de canal del enésimo dispositivo m mMTC sobre la portadora s
    \item $h_{u}^{s}:$ Ganancia de canal del enésimo dispositivo u uRLLC sobre la portadora s
    \item ${\hat S}$: Lista de subportadoras asignadas
    \item $S_{a}^{c}$: Lista de subportadoras asignadas al enésimo cluster
    \item ${C_{ns}}$: Lista de cluster aún no asignados
    \item $W$ Ancho de banda en un tono NB-IoT (3.75KHz)
    \item $\alpha$, $\gamma$, $\beta$ son variables binarias que indican asignaciones.
\end{itemize}

La tasa de datos alcanzable por un dispositivo $m$ (mMTC) en términos de la tasa agregada sobre las subportadoras asignadas se puede expresar como \parencite{Shahini2019}:
\begin{equation}
{R_{m}}=\sum \limits _{c \in \mathcal {C}} {\sum \limits _{k \in \mathcal {K}} {\alpha _{m}^{c,k}\sum \limits _{s \in \mathcal {S}} {{\gamma ^{s,c}}W} } } \times {\log _{2}}\left ({{1 + \frac {{{{\left |{ {h_{m}^{s}} }\right |}^{2}}p_{m}^{s}}}{{N_{0}W + \sum \limits _{d \in \mathcal {M}\backslash m} {\sum \limits _{h = k + 1}^{{k_{\max }}} {\alpha _{d}^{c,h}{{\left |{ {h_{d}^{s}} }\right |}^{2}}p_{d}^{s}} } }}} }\right)
\label{eqn:Rm}
\end{equation}

Del mismo modo, la Tasa de datos alcanzable por un dispositivo $u$ (URLLC) puede determinarse mediante el teorema de Shannon-Hartley. Hay que tomar en cuenta que los rangos de URLLC siempre son mayores que los de mMTC en cada clúster NOMA. Por lo tanto, reciben interferencia de todos los miembros del clúster mMTC, así como de los miembros del clúster URLLC con rangos más altos. 
Por lo tanto, la tasa de datos alcanzable de un dispositivo $u$ URLLC sobre las subportadoras asignadas es \parencite{Shahini2019}:
\begin{equation}
{R_{u}}=\sum \limits _{c \in \mathcal {C}} {\sum \limits _{k \in \mathcal {K}} {\beta _{u}^{c,k}\sum \limits _{s \in \mathcal {S}} {{\gamma ^{s,c}}W} } } \times {\log _{2}}\left ({{1 + \frac {{{{\left |{ {h_{u}^{s}} }\right |}^{2}}p_{u}^{s}}}{{N_{0}W + \sum \limits _{d \in \mathcal {U}\backslash u} {\sum \limits _{h = k + 1}^{{k_{\max }}} {\beta _{d}^{c,h}{{\left |{ {h_{d}^{s}} }\right |}^{2}}p_{d}^{s}}} \sum \limits _{m \in \mathcal {M}} {\sum \limits _{h = k + 1}^{{k_{\max }}} {\alpha _{d}^{c,h}{{\left |{ {h_{m}^{s}} }\right |}^{2}}p_{m}^{s}} } }}} }\right)
\label{eqn:Ru}
\end{equation}

\subsection{MODELOS DE TRÁFICO}

Los modelos de tráfico en comunicaciones móviles buscan acercarse, lo más posible a cómo transmiten datos o realizan peticiones de acceso los dispositivos que intentan modelar. Estos modelos de tráfico pueden clasificarse en modelos de tráfico\textbf{agregado } y modelos de tráfico \textbf{fuente} \parencite{Laner2013}. El tráfico agregado simula un flujo de tráfico que se agrupa para recibir un tratamiento común, mientras que en los modelos de tráfico fuente es justamente cada una de las fuentes generadoras del tráfico la que se simula y frecuentemente se hace acompañada de una cadena de Markov que intenta representar los distintos estados del dispositivo fuente y la probabilidad de transición entre ellos.\newline

Sin importar el modelo de tráfico a utilizar, en \parencite{Laner2013} se señala que los modelos de tráfico que pretendan simular el comportamiento de dispositivos de MTC (\textit{Machine Type Communications}) deben:

\begin{itemize}
\item  Capturar con precisión el comportamiento de un solo dispositivo de MTC 
\item  Permitir la simulación concurrente de una cantidad masiva de dispositivos con su potencial reacción síncrona a un evento.
\end{itemize}

Antes de elegir un modelo es importante conocer las propiedades del tráfico máquina a máquina (M2M, \textit{Machine to Machine}), el cual se considera una forma de transmisión de datos que no requiere necesariamente de la interacción humana (ETSI, 2010) y corresponde justamente al tráfico de los nodos IoT, de \parencite{Laner2013} se tiene:

\begin{itemize}
\item  Cantidad masiva de dispositivos
\item  Pocos paquetes de un tamaño pequeño a ser transmitidos por dispositivo
\item  Periodos largos entre dos transmisiones consecutivas
\item  Tráfico de subida (\textit{uplink)} dominante
\item  Transmisiones en tiempo real y transmisiones tolerantes al retraso
\item  Paquetes no sincronizados y paquetes sincronizados
\item  Activación de tráfico que depende del espacio y tiempo
\end{itemize}

Además, se hace  la distinción de 3 patrones de tráfico que pueden presentarse en estos dispositivos:

\begin{enumerate}
    \item \underbar{Actualización periódica (PU, }\textit{\underbar{Periodic Update}}\underbar{):} Este tipo de tráfico ocurre cuando el dispositivo transmite reportes de estado y/o actualizaciones de estado de manera periódica. Puede verse como una activación por evento que ocurre por el mismo dispositivo en un intervalo periódico. Típicamente, el tráfico PU no necesita transmitirse en tiempo real y cuenta además de un patrón periódico de tiempo con un tamaño constante en sus paquetes. Un ejemplo típico de estos dispositivos son medidores inteligentes (por ejemplo gas, electricidad, agua).
    \item \underbar{Activación por evento (ED, }\textit{\underbar{Event-Driven}}\underbar{):} En caso de que un evento desencadene la transmisión de datos de un dispositivo, el patrón de tráfico corresponde a esta segunda clase. Un evento puede ser causado ya sea por la medición de un parámetro que sobrepasó un límite y activó alguna alarma o bien por el nodo que actúa como servidor y envía comandos al dispositivo. El tráfico \textit{Event-Driven} puede requerir ser transmitido tanto en tiempo real o no, un ejemplo de mensajes de subida que debieran ser transmitidos en tiempo real son alarmas y notificaciones médicas de emergencia, en cuanto a los mensajes de bajada, estos podrían ser la distribución de mensajes de emergencia locales, por ejemplo en caso de sismo o tsunamis. En algunos casos, como ya se mencionó, este tráfico no necesita ser transmitido en tiempo real. Por ejemplo, cuando un dispositivo IoT envía una actualización de su ubicación al servidor o se reciba una actualización de \textit{firmware }desde este.
    \item \underbar{Intercambio de carga útil (PE, }\textit{\underbar{Payload Exchange}}\underbar{):} este último tipo de tráfico ocurre después de una transmisión previa (PU o ED). Comprende todos los casos en los que es necesario un mayor intercambio de datos entre el dispositivo que envía y su servidor, este tráfico se espera sea predominantemente de subida y puede ser de tamaño constante o variable según la aplicación.
\end{enumerate}

Las aplicaciones en el mundo real que brindan servico a dispositivos de IoT generarán casi siempre tráfico en una combinación de estos tipos, más un estado de reposo o de ahorro de batería. Este simulador como se verá en la sección \ref{Chapter5} generó tráfico únicamente de los tipos: Actualización periódica y Activación por evento.\newline
\begin{figure}[th]
\centering
\includegraphics[scale=1]{Figures/Estructura de los estados principales del tráfico M2M}
\decoRule
\caption[Estructura de los estados principales del tráfico M2M]{Estructura de los estados principales del tráfico M2M}
\label{fig:}
\end{figure}

Ahora se presentan los modelos de tráfico más recurrentes para la simulación de comunicaciones M2M.

\subsubsection{Modelos de tráfico agregado}

Han sido propuestos por la 3GPP al reconocer la importancia de caracterizar el tráfico M2M. Se trata en realidad de 2 modelos de tráfico agregado generado por una gran cantidad de usuarios, el primero modela el tráfico generado de forma aleatoria y el segundo modela tráfico síncrono en el tiempo, esto se puede observar en la \textit{Tabla~\ref{tab:trafico3gpp}}.

\begin{itemize}
\item  \textit{Modelo 1} - Modelo de tráfico agregado sin correlación 3GPP: Genera tráfico sin correlación en un intervalo específico de tiempo. Lo que significa que no se tomarían en cuenta la correlación entre los dispositivos IoT. Utiliza una distribución uniforme para modelar el tráfico agregado en un intervalo de tiempo específico.
\item  \textit{Modelo 2 -} Modelo de tráfico agregado con correlación 3GPP: Este modelo genera tráfico correlacionado en un intervalo de tiempo, asumiendo que todas las máquinas se encuentran sincronizadas. Utiliza una distribución beta para modelar en tráfico agregado en un intervalo de tiempo específico.
\end{itemize}

\begin{table}
\caption{Modelos de tráfico agregado propuestos por la 3GPP para comunicaciones M2M}
\label{tab:trafico3gpp}
\centering
\begin{tabular}{*{2}{m{8.5cm}}}\\
\textbf{Sincronizado/Coordinado/Correlacionado\newline (En un intervalo limitado en el tiempo)} & \textbf{No sincronizado/No coordinado/ No correlacionado\newline (En un intervalo limitado en el tiempo)} \\ \hline \hline
Distribución de probabilidad de arribo de paquetes/peticiones f(t) en [0,1] : \underbar{Beta} (3,4) & Distribución de probabilidad de arribo de paquetes/peticiones f(t) en [0,1]: \underbar{Uniforme} \\ 
Número de dispositivos: 1 000, 3 000, 5 000, 10 000, 30 000. & Número de dispositivos: 1 000, 3 000, 5 000, 10 000, 30 000. \\ 
Periodo \textit{T }: 10 s & Periodo \textit{T }: 60 s \\ 
\end{tabular}
\end{table}

La principal ventaja de los modelos de tráfico agregado es su fácil implementación (en términos de una baja complejidad computacional) cuando se simulan una gran cantidad de dispositivos. Por otro lado, como se menciona en \parencite{IoTTrafficHossfeld}, la imprecisión de estos modelos para reflejar el comportamiento real del sistema es su principal desventaja.

\subsubsection{Modelos de tráfico fuente}

Los modelos de tráfico fuente, modelan justamente el tráfico que genera cada uno de los dispositivos. Este tipo de modelado es más preciso que el de tráfico agregado ya que modela el comportamiento de cada nodo, sin embargo, puede  volverse muy complejo cuando se agrega una gran cantidad de dispositivos (fuentes) y comportamientos. A continuación se presentan y analizan dos modelos de tráfico fuente.

\begin{itemize}
\item  \textit{Modelo 3:} Modelo de fuente de \textit{Semi-Markov} (\textit{Semi-Markov Models, SMM)}

En este modelo de fuente cada dispositivo se modela utilizando una cadena de Markov en la que se define la probabilidad de transición entre estados. Los estados que se encontrarán casi siempre modelados son los mencionados anteriormente: el de actualización periódica (PU), el de activación por evento (ED) y el de intercambio de carga útil. La \textit{Figura~\ref{fig:SMM}} muestra cómo se verían modelados los estados de un dispositivo en una cadena de Markov.\newline

La probabilidad de transición entre el mismo estado es 0, además los tiempos de espera y la longitud de los mensajes son generados de acuerdo a una distribución de probabilidad que es independiente de cada estado y potencialmente distinta para cada uno de ellos \parencite{IoTTrafficHossfeld}.\newline

\begin{figure}[th]
\centering
\includegraphics[scale=1]{Figures/Cadena de Markov del modelo SMM}
\decoRule
\caption[Cadena de Markov del modelo SMM]{Cadena de Markov del modelo SMM}
\label{fig:SMM}
\end{figure}

La principal ventaja del modelo de tráfico fuente SMM es que permitiría una descripción más detallada del comportamiento de los dispositivos IoT de manera individual, sin embargo no es capaz de capturar la relación que existe entre dos dispositivos cercanos que pudieran tener una cierta sincronía. Otra desventaja es que la complejidad del sistema aumenta considerablemente entre más dispositivos se simulan a diferencia de los modelos de tráfico agregado.

\item  \textit{Modelo 4:} Modelo de fuente de Procesos de Poisson emparejados Markov-modulados (CMMPP\textit{, Coupled Markov Modulated Poisson Process)}

En el modelo de tráfico CMMPP cada dispositivo MTC es representado como una entidad por separado y a diferencia del modelo SMM sí puede representarse una sincronización espacial y temporal entre dispositivos similares. La clave en el diseño del modelo CMMPP se presenta en encontrar un balance entre el emparejamiento entre distintos dispositivos y una complejidad tolerable del sistema cuando se tiene una gran cantidad de dispositivos \parencite{Gupta2018}.

Los procesos de Poisson Modulados con Markov (\textit{Markov modulated Poisson processes, MMPP}) consisten en procesos de Poisson que son modulados por la tasa $\lambda_{i[t]}$, que viene determinada por el estado de una cadena de Markov $sn[t]$. este principio se ve presentado en la \textit{Figura~\ref{fig:CMMPP}} donde \textit{p${}_{i,j}$}${}_{\ }$son las probabilidades de transición entre los estados de la cadena. En este modelo cada dispositivo \textit{n} del total\textit{ N} se encuentra representado por una cadena de Markov y un correspondiente proceso de Poisson. Debido a que existe una alta correlación en el cambio de estados de distintos dispositivos, tanto en el espacio como en el tiempo, es necesario realizar un emparejamiento. En los modelos genéricos, el emparejamiento se realiza introduciendo enlaces bidireccionales entre los dispositivos, pero esto sería sin lugar a dudas muy complejo de simular, de manera que en \parencite{Gupta2018} se propone un proceso de fondo actuando como \textit{maestro} el cual modula todos los dispositivos del mismo tipo.

\begin{figure}[th]
\centering
\includegraphics[scale=1]{Figures/Modelo MMPP en dispositivos MTC}
\decoRule
\caption[Modelo CMMPP en dispositivos MTC]{Modelo CMMPP: cada dispositivo MTC n está representado por una cadena de Markov con estados sn, que establecen el parámetro $\lambda$. Este parámetro es el promedio de la tasa de arribos, el cual modela el respectivo proceso de Poisson}
\label{fig:CMMPP}
\end{figure}

\end{itemize}

La principal ventaja como ya se mencionó del tráfico fuente frente al tráfico agregado es su precisión, por otra parte, la del tráfico agregado es su fácil implementación para un gran número de dispositivos. El modelo CMMPP es un intermedio entre estos dos casos, es decir mantiene la precisión del modelado de tráfico fuente mientras se mantiene la viabilidad para un gran número de máquinas. A continuación se presenta en la \textit{Tabla~\ref{tab:Traficos}} una comparación entre los distintos modelos mencionados.\newline
\begin{table}
\caption{Comparativa entre los modelos de tráfico MTC abordados}
\label{tab:Traficos}
\centering
\begin{tabular}{|p{3in}|p{1in}|p{0.8in}|p{1in}|} \\  \hline \hline
\textbf{\textit{Métrica}} & \textbf{Agregado~} & \textbf{SMM} & \textbf{CMMPP} \\ \hline 
\textit{Modelado de los dispositivos} &  & \checkmark & \checkmark \\ \hline 
\textit{Modelado de dispositivos coordinados} & \checkmark &  & \checkmark \\ \hline 
\textit{Coordinación espacial y temporal} &  &  & \checkmark \\ \hline 
\textit{Modelado de los paquetes} &  & \checkmark &  \\ \hline 
\textit{Modelado de la tasa de arribo} & \checkmark & \checkmark & \checkmark \\ \hline 
\textit{Tiempo de ejecución aleatorio factible} &  & \checkmark & \checkmark \\ \hline 
\textit{Ubicación del dispositivo} &  & \checkmark & \checkmark \\ \hline 
\textit{Emparejamiento de estados} &  &  & \checkmark \\ \hline 
\textit{Complejidad (N número de dispositivos)} & O(1) & O(N) & O(N) \\  
\end{tabular}
\end{table}

Como puede observarse en la \textit{Tabla~\ref{tab:Traficos}}, el modelo CMMPP es bastante conveniente a la hora de generar tráfico de dispositivos mIoT, pues este es capaz de simular la relación espacial y temporal que existiría entre los nodos. Si se regresa a la \textit{Tabla~\ref{tab:appssim}}, en la que se presentan los servicios que se simularán, se encuentra que servicios como el monitoreo de la condición del aire en la ciudad, la detección de terremotos, la manipulación de la iluminación y demás tendrán un comportamiento similar en un espacio limitado. De manera que el modelo CMMPP se seleccionó para la simulación de este sistema, complementándolo con un modelo determinístico para las aplicaciones que sólo producen tráfico periódico.\newline

Una explicación resumida del modelado  de tráfico fuente CMMPP se realiza a continuación \parencite{Gupta2018}:

\begin{enumerate}
\item  Un conjunto de \textit{k }estados se definen, cada uno asociado con una tasa de generación de paquetes ${\lambda }_k$. Un dispositivo IoT se encuentra en todo momento en algún de estos estados representados en una cadena de Markov formada por los estados antes mencionados.
\item  La transición entre los \textit{k} estados para cualquier dispositivo está definida por una matriz de probabilidad de cambio de estado $P_n$ ,la cual es a su vez una función de dos matrices de transición $P_u$ y $P_c$ ( que modelan el comportamiento no coordinado/no sincronizado y comportamiento coordinado/sincronizado respectivamente).
\item  Un factor de correlación espacial ${\delta }_n$ se asigna a cada dispositivo \textit{n. }Esto modela qué tanto se involucra un dispositivo durante la generación de tráfico coordinado en la red y dicta efectivamente la contribución de $P_c$ en la matriz resultante de probabilidad de transición de ese dispositivo.
\item  Se define un proceso $\mathit{\Theta}\left(t\right)$ el cual controla la matriz de  transición instantánea de estado del \textit{n-ésimo}\textbf{\textit{ }}dispositivo en el instante \textit{t.}
\end{enumerate} 
% Chapter 5

\chapter{Diseño} % Main chapter title

\label{Chapter5} % Change X to a consecutive number; for referencing this chapter elsewhere, use \ref{ChapterX}

%Intro del capitulo
En este capítulo se muestra el diseño del simulador programado. El objetivo  de esta sección fue crear un modelo de sistema, para el cual se especificó un escenario y el modelado de cuatro principales componentes que acabaron siendo implementados en el simulador: el despliegue de UEs, el modelo de canal, el esquema de acceso al múltiple al medio y los modelos de tráfico.\newline

%----------------------------------------------------------------------------------------
%	SECTION 
%----------------------------------------------------------------------------------------

\section{Modelo de sistema}

En principio, se comenzó con la etapa de análisis (\textit{Capítulo \ref{Chapter4}}), donde se analizaron los requisitos de la red, las características que debía tener el escenario que se propuso y los distintos modelos que formaron parte de la simulación.\newline

En términos generales, se consideró una red celular uni-celda para transmisiones de subida (\textit{uplink}), el escenario se trata de uno urbano macro celular (UMa) en donde frecuentemente los UEs se situan estáticos y en el exterior. De acuerdo a los modelos analizados y ya elegidos en la \textit{Sección~\ref{AnalisisMODELOS}}, el modelo de sistema del simulador  implementó los siguientes sub-sistemas \textit{[Figura~\ref{fig:DiagramaGral}]}:

\begin{enumerate}
    \item  Uso de una geometría estocástica, es decir, despliegue de UEs siguiendo un PPP.
    \item  Pérdidas de canal usando un modelo CI para ambientes exteriores.
    \item  Incorporación de un esquema de acceso al medio no ortogonal (NOMA) usando una técnica de agrupamiento \textit{(clustering)} de usuarios.
    \item  Diferentes modelos de tráfico que simulen distintos servicios para NB-IoT.
\end{enumerate}

\begin{figure}[th]
    \centering
    \includegraphics[scale=.9]{Figures/Diagrama general de bloques del simulador}
    \decoRule
    \caption[Diagrama general de bloques del simulador, constando principalmente de 4 módulos.]{Diagrama general de bloques del simulador, constando principalmente de 4 módulos.}
    \label{fig:DiagramaGral}
\end{figure}

El simulador se enfocó en el caso de uso de mMTC (o mIoT) el cual se caracteriza por brindar servicio a un gran número de dispositivos, esto es, teniendo una alta densidad de volumen de tráfico en escenarios con aglomeración de dispositivos. Por lo tanto, como se revisó en el capítulo anterior un excelente candidato para cumplir con los requerimientos del caso de uso mIoT y que forma ahora parte del estándar 5G, fue la tecnología NB-IoT. De este estándar, se tomaron sus especificaciones técnicas (\textit{Sección~\ref{NBIoT} }), tales como los parámetros fundamentales para la comunicación entre la BS y los UEs en la simulación.\newline

A continuación se definen las características generales de cada sub-sistema:

\subsection{Despliegue de dispositivos}

De igual manera que se realizó en \parencite{Kouzayha2018} y \parencite{Zhang2017} con el fin de obtener un análisis fundamental y más realista, se propuso la generación de una geometría estocástica en 2D para la distribución de los UEs. Se tomó como modelo de despliegue de UE un \textit{ Homogeneous Poisson Point Process }(proceso puntual de Poisson homogeneo, HPPP) con distintas densidades para los diferentes tipos de dispositivos que se implementaron.\newline

\subsection{Pérdidas del canal}

En la \textit{Sección \ref{AppsEscenario} } se establecieron las aplicaciones de los diferentes dominios de IoT que se utilizaron en el simulador. Todas estas aplicaciones se encuentran presentes, no exclusivamente pero sí particularmente, en escenarios urbanos y en exteriores, por lo tanto se implementó el Modelo CI (ecuación \ref{eqn:CI} ) el cual de acuerdo a lo estudiado en \parencite{Sun2016} es el modelo que mejor estima la señal en este tipo de ambientes. Además, se agregarón perdidas por el desvanecimiento rápido, usando el modelo de desvanecimiento de Rayleigh (siguiendo una distribución Rayleiggh con desviación estándar unitaria y que se distribuyó de forma independiente e idéntica [i.i.d.]), este es ideal para entornos urbanos en situaciones donde hay un gran número de multi-trayectorias de la señal y reflexiones causadas por los edificios y objetos que obstruyen la línea de vista. \newline

Con base en \parencite{Sun2016}, para el Modelo CI en ambientes \textit{urban macro} se tienen los parámetros mostrados en la Tabla~\ref{tab:ModeloCI} que dependiendo el tipo de ambiente (ya sea con línea de vista [LoS] o sin línea de vista [NLoS]) los parámetros rango de distancia y el exponente de pérdida (PLE) varían. El modelo de sistema propuesto se trata de un ambiente en exteriores, se considera que sí habrá línea de vista entre los UE y la BS, por lo tanto, se utilizó LoS con distancias entre [60 - 900 m] y un PLE igual a 2.\newline
\begin{table}
    \caption{Parámetros Modelo CI [Fuente: \parencite{Sun2016}]}
    \label{tab:ModeloCI}    
    \centering
    \begin{tabular}{*{7}{m{2cm}}}\\ 
    \textbf{Escenario} & \textbf{Ambiente} & \textbf{Rango de Frecuencias} & \textbf{\# Puntos de Datos\newline analizados} & \textbf{Rango de distancias\newline entre BS y UE} & \textbf{Modelo Canal} & \textbf{PLE} \\ \midrule
    \multirow{2}{*}{\textbf{\begin{tabular}[c]{@{}c@{}}UMa\end{tabular}}} & \textit{LoS} & \textit{2 - 38 GHz} & \textit{1032} & \textit{60 - 930 m} & \textit{CI} & \textit{2.0} \\
     & \textit{NLoS} & \textit{2 - 38 GHz} & \textit{1869} & \textit{61 - 1238 m} & \textit{CI} & \textit{2.9} \\ \cmidrule(l){2-7} 
    \end{tabular}
\end{table}

Algunos parámetros siguieron las características de las mediciones que se hicieron para validar el modelo CI, en \parencite{Sun2016}. Las mediciones se realizaron en Vestby, Aalborg, Dinamarca, en las bandas de frecuencia de 2, 10, 18 y 28 GHz en marzo. Vestby representa una típica ciudad europea de tamaño mediano con construcciones y anchos de calle regulares, que son de aproximadamente 17 m (cinco pisos) y 20 m, respectivamente. Para el escenario UMa, la antena BS está por encima de la altura de la azotea, típicamente 25 m más o menos por encima del suelo \parencite{Sun2016}.\newline

Por último, para la definición del rango de frecuencia de las transmisiones, como se pudo leer en la Sección~\ref{NBIoT}, en la actualidad el despliegue de redes NB-IoT se ha realizado en bandas EUTRA (Acceso de radio en LTE) y bandas GSM (SA), por lo que este sistema se implementó en la banda de microondas, se ocupó la banda mas baja en la cual el modelo CI es válido, es decir, la banda de 2GHz.

\subsection{Esquema de acceso al medio no ortogonal}
Se predefinió que exista un bloque de recursos (PRB) para la BS dedicado para el estándar NB-IoT en la banda de 2GHz, dando servicio a comunicaciones tipo maquina (MTC). El recurso de radio tiene un ancho de banda de 180kHz y este recurso se dividió en 48 sub-portadoras de 3.75kHz con operación \textit{singletone y multitone}.\newline

Por lo tanto, para la compartición de recursos, en vías de dar servicio a un gran número de dispositivos, propuso implementar un esquema de acceso múltiple al medio no ortogonal (NOMA) descrito en la \textit{Sección~\ref{NOMA_C4}} y la implementación de una técnica de agrupamiento de usuarios (\textit{clustering}) y se aprovechó la no ortogonalidad, para agrupar diferentes clases de nodos IoT en una misma sub-banda.\newline

El cálculo de las tasas de los dispositivos URLLC y mMTC, se describen con las Ecuaciones~\ref{eqn:Ru} y ~\ref{eqn:Rm} respectivamente.\newline

Con la incorporacion del Modelo de Canal CI, el cálculo de las ganancias $h$ se dió de la siguiente manera:

\begin{equation}
    h =  10^{(\frac{-L_{p [dB]}^{CI}}{10})} \cdot \gamma\ [W]
    \label{eqn:h_canal}
\end{equation}
Donde:
\[\gamma \to ganancia\ desvanecimiento\ Rayleigh\]
\[\ L_{p [dB]}^{CI} \to \textit{Path\ Loss}\ Modelo\ CI \]

\subsection{Modelos de tráfico para servicios NB-IoT}

En la implementación de modelos de tráfico se tuvo en cuenta el modelo CMMPP y un modelo determinístico. El modelo CMMPP pudo tener una instancia distinta para cada una de las aplicaciones que se propusieron en \textit{Tabla~\ref{tab:appssim} }, a excepción de aquellas aplicaciones con transmisiones periódicas, para las cuales se utilizó un modelo determinístico\textit{.}\newline

Cada tipo de dispositivo tiene una instancia de \textit{procesos maestros}. Estos procesos, están modificando las probabilidades de cambio de estado de cada nodo según su proximidad a otros nodos que estén cambiando de estado. Por ejemplo, digamos que un nodo, que llamaremos N1, que detecta terremotos se encuentra en estado de reposo y su probabilidad de transición al estado de transmisión por alarma es del 1\%, ahora de pronto otro nodo (de la misma aplicación) en su proximidad cambia de estado a transmisión por evento, y unos instantes después la probabilidad de transición en nuestro nodo que antes era del 1\% aumenta hasta 90\%, lo que desencadena que este nodo también comience a transmitir al cambiar de estado unos momentos después. El ejemplo anterior, en la vida real se traduciría como un terremoto ocurriendo en un lugar y una gran cantidad de nodos que se encargan de detectarlo comenzando de pronto a transmitir con una indudable coordinación espacial y temporal.\newline

\begin{figure}[th]
\centering
\includegraphics[scale=1]{Figures/Cadena de Markov propuesta}
\decoRule
\caption[Cadena de Markov propuesta]{Cadena de Markov propuesta}
\label{fig:CMMPPpropuesta}
\end{figure}

Habiendo explicado esto, la implementación del modelo CMMPP en las aplicaciones, control de iluminación, detección de terremotos y genérico contará con un diseño de cadena de Markov como el que se presenta en la \textit{Figura~\ref{fig:}. }En esta figura se puede ver que se modelarán dos distintos estados para cada nodo IoT de estas aplicaciones. El primer estado, llamado normal corresponde al funcionamiento \textbf{normal} o principal de la aplicación, con una tasa de transmisión Lambda 1 (${\lambda }_1$). A la vez el segundo estado, llamado \textbf{Alerta} o Alarma con tasa de transmisión Lambda 2 (${\lambda }_2$), corresponde al estado que acudirán los dispositivos IoT de acuerdo con eventos de interes producidos aleatoriamente en el área de la célula. Es justamente con la ayuda de este estado que el modelo es capaz de simular la coordinación espacial y temporal de los dispositivos.

\begin{equation}
P_n\left[k\right]=\ _n\left[k\right] P_c+\left(1-\ _n\left[k\right]\right)\ P_u 
\label{eqn:Pn}
\end{equation}

\[donde:\ n\ corresponde\ al\ n-\textrm{é}simo\ nodo\] 
\[y\ k\ a\ un\ determinado\ instante\ de\ tiempo\] 

En la ecuación \ref{eqn:Pn} vemos la forma en la que se modulan las matrices de probabilidad de transición entre los estados. Entonces para calcular la matriz $P_n\left[k\right]$, es decir la perteneciente al nodo \textit{n }en el instante \textit{k, }necesitamos de las matrices $P_c$ y $P_u$ que corresponden al comportamiento coordinado y no coordinado respectivamente y del valor $_n\left[k\right]$. Las matrices $P_c$ y $P_u$ marcan el comportamiento que tendría el nodo en los casos extremos de coordinación o en la ausencia de ésta, esto debido a que $_n\left[k\right]$ varía entre [0 ,1], entonces si existe una perfecta coordinación entre nodos y este valor es 1, en algún momento, el segundo sumando de la ecuación \ref{eqn:Pn} sería 0 y $P_n\left[k\right]=\ P_c$. La propuesta para $P_c$ y $P_u$, tal y como se ha utilizado en \parencite{Gupta2018} y \parencite{Smiljkovic2014} fue:

\begin{equation}
P_{u} =  
\begin{bmatrix}
1 & 1 \\
0 & 0 
\end{bmatrix}
\end{equation}

\begin{equation}
P_{c} = 
\begin{bmatrix}
0 & 1 \\
1 & 0 
\end{bmatrix}
\end{equation}

Ahora se estudia el término $_n\left[k\right]$ que es el que se encarga de simular la correlación existente entre los nodos, el cual se compone de $_n\left[k\right]=\ {\delta }_n[k]$, siendo ${\delta }_n$ el término encargado de la coordinación espacial y $[k]$ el de la coordinación temporal. Para ${\delta }_n$ se utilizaron dos funcioness: una exponencial decreciente \textit{Decaying exponential} y una ventana de coseno alzado \textit{Rised-cosine window} según convino para cada aplicación, tal y como se propone en \parencite{Gupta2018}.  La ventana de coseno alzado permitió simular un término abrupto en la transmsión de las alarmas. Finalmente para $[k]$ se calculó su valor utilizando una función delta, que sólo toma el valor 1 cuando la alarma se ha propagado hasta el sitio del nodo. \newline

Mientras un nodo se encuentre en el estado normal, este genera paquetes a la tasa $\lambda_{normal}$. Un cambio de estado se da por una alarma que se propagó hasta el sitio del nodo. En la \textit{Figura~\ref{fig:CMMPP_Algoritmo}} se presenta el diagrama de flujo con el algoritmo que se utilizó como base para generar el tráfico que sigue este modelo, hizo falta la incorporación de tráfico periódico para las aplicaciones que así lo requirieron.\newline

\begin{figure}[th]
\centering
\includegraphics[scale=.7]{Figures/Generación de tráfico con modelo CMMPP}
\decoRule
\caption[Generación de tráfico con modelo CMMPP]{Generación de tráfico con modelo CMMPP}
\label{fig:CMMPP_Algoritmo}
\end{figure}

El segundo modelo se trata de uno determinístico, y fué el que se implementó en las aplicaciones cuya tasa de tráfico es periódica. De manera que lo único que se debe conocer es la tasa de tráfico de los nodos con la cual se calendarizan los nacimientos de paquetes. Las aplicaciones contempladas para este modelo son: el monitoreo del consumo de agua y electricidad en la ciudad, el monitoreo de la contaminación del aire y el control dinámico de los semáforos.\newline

\begin{table}
\caption{Caracterización de las aplicaciones seleccionadas}
\label{tab:AppsSimulacion}
\centering
\begin{tabular}{|p{1.4in}|p{0.7in}|p{0.7in}|p{0.7in}|p{0.4in}|p{1.8in}|} \\ 
\textbf{\textit{Número de Servicio y Nombre}} & \textbf{Tamaño de red} & \textbf{Tasa de tráfico} & \textbf{Demanda de QoS} & \textbf{Clase} & \textbf{Tamaño de paquete} \\ \hline \hline
\textit{1 - Control de iluminación\newline (Ciudad Inteligente) } & \footnotesize{ Grande, miles de dispositivos } & \footnotesize{ Aleatorio, poco frecuente } & \footnotesize{ Media, tolerante al retardo 15seg } & \footnotesize{ mMTC } & \footnotesize{ Activación aleatoria\newline \textbf{UL}: 20 bytes \textit{payload}\newline \textbf{DL}: ACK de 0 bytes } \\ \hline 
\textit{2 - Monitoreo del consumo de agua y electricidad en la ciudad\newline (Ciudad Inteligente) } & \footnotesize{ Media a grande, cientos a miles de dispositivos } & \footnotesize{ Periódico, 1 msj cada 10 min por dispositivo } & \footnotesize{ Baja, tolerante al retardo 1min } & \footnotesize{ mMTC } & \footnotesize{ Activación periódica\newline \textbf{UL}: distribución de Pareto con parámetro alfa = 2.5 y tamaño mínimo de carga útil de la aplicación = 20 bytes con un corte a 200 bytes\newline \textbf{DL}: ACK de 0 bytes 50\% de las veces. } \\ \hline 
\textit{3 - Detección de terremotos\newline (Ambiente Inteligente) } & \footnotesize{ Media a grande, cientos a miles de dispositivos } & \footnotesize{ Aleatorio, poco frecuente } & \footnotesize{ Alta, tolerante al retardo 3seg } & \footnotesize{ mMTC } & \footnotesize{ Activación aleatoria\newline \textbf{UL}: 20 bytes \textit{payload}\newline \textbf{DL}: ACK de 0 bytes } \\ \hline 
\textit{4 - Monitoreo de contaminación del aire\newline (Ambiente Inteligente) } & \footnotesize{ Media a grande, cientos a miles de dispositivos } & \footnotesize{ Periódico, 1 msj cada 15 min por dispositivo } & \footnotesize{ Media, tolerante al retardo 15seg } & \footnotesize{ mMTC } & \footnotesize{ Activación periódica\newline \textbf{UL}: distribución de Pareto con parámetro alfa = 2.5 y tamaño mínimo de carga útil de la aplicación = 20 bytes con un corte a 200 bytes\newline \textbf{DL}: ACK de 0 bytes 50\% de las veces. } \\ \hline 
\textit{5 - Control dinámico de semáforos\newline (Transporte y Movilidad Inteligentes) } & \footnotesize{ Grande, miles de dispositivos } & \footnotesize{ Periódico, 1 msj cada min por dispositivo } & \footnotesize{ Alta, tolerante al retardo 5seg } & \footnotesize{ mMTC } & \footnotesize{ Activación periódica\newline \textbf{UL}: distribución de Pareto con parámetro alfa = 2.5 y tamaño mínimo de carga útil de la aplicación = 20 bytes con un corte a 200 bytes\newline \textbf{DL}: ACK de 0 bytes 50\% de las veces. } \\ \hline 
\textit{6 - Otros Dispositivos mMTC}  & \footnotesize{ Grande, miles de dispositivos } & \footnotesize{ Aleatorio, poco frecuente } & \footnotesize{ Alta, tolerante al retardo 5seg } & \footnotesize{ mMTC } & \footnotesize{ Activación aleatoria\newline \textbf{UL}: 20 bytes \textit{payload}\newline \textbf{DL}: ACK de 0 bytes } \\ \hline 
\textit{7 - Dispositivos URLLC}  & \footnotesize{ Grande, miles de dispositivos } & \footnotesize{ Aleatorio, poco frecuente } & \footnotesize{ Alta, tolerante al retardo 3seg } & \footnotesize{ URLLC } & \footnotesize{ Activación aleatoria\newline \textbf{UL}: 20 bytes \textit{payload}\newline \textbf{DL}: ACK de 0 bytes } \\
\end{tabular}
\end{table}

Ahora, en la \textit{Tabla~\ref{tab:AppsSimulacion}} se presentan las aplicaciones mencionadas en la \textit{sección \ref{AppsEscenario}} junto a su caracterización. Adicionalmente se anexó la distinción de las aplicaciones en dos clases distintas, mMTC y URLLC que permitió realizar el algoritmo de agrupaciones NOMA. Se buscó que por cada nodo URLLC hubieran por lo menos 3 nodos mMTC. por lo tanto se utilizaron distintas intensidades de dispositivos que conservaran la relación. \newline


En la \textit{Figura~\ref{fig:EscenarioMTC}} se muestra una aproximación del escenario implementado, usando un agrupamiento con cuatro nodos.

\begin{figure}[th]
\centering
\includegraphics[scale=.65]{Figures/Escenario mIoT unicelda}
\decoRule
\caption[Ejemplo ilustrativo de un escenario mIoT unicelular aproximado, usando agrupaciones de 4 dispositivos]{Ejemplo ilustrativo de un escenario mIoT unicelular aproximado, usando agrupaciones de 4 dispositivos}
\label{fig:EscenarioMTC}
\end{figure}
%%%%%%%%%%%%%%%%%%%%%%%%%%%%%%%%%%%%%%%%%%%%%%%%%%%%%%%%%%%%%%%%%%%%%
\section{Interconexión de los 4 módulos del Simulador}

Los 4 módulos del sistema, cuyo diseño se ha discutido en la primera parte de este capítulo son: el despliegue de los dispositivos, el modelo del canal, el esquema de acceso múltiple al medio y los modelos de tráfico IoT. El diseño de la implementación de estos módulos se encuentra descrita en el diagrama de la \textit{figura~\ref{fig:interconexion4}}. \newline

El simulador se encuentra entonces dividido en 2 programas, el \textbf{Generador de Tráfico IoT} y el \textbf{Simulador de eventos discretos}. La razón por la que se diseñó esta arquitectura se debe a que el algoritmo de tráfico CMMPP requiere de un ciclo que avance periodicamente el tiempo, mientras que para el simulador principal se propuso un algoritmo de eventos discretos.\newline

Ambos sistemas fueron programados en Python 3.6 y fueron diseñados para operar  de manera modular. Es decir el programa Generador de Tráfico IoT es una herramienta funcional y modificable por sí sola para generar tráfico de tipo máquina con los modelos CMMPP y Periódico, con usuarios distribuidos a partir de un PPP. Los resultados de este programa podrían entonces ser utilizados en algúno otro, realizando los ajustes de interfaz necesarios. De la misma manera el Simulador de Eventos discretos que contiene el modelo del canal y el esquema de acceso múltiple al medio, podría procesar tráfico y usuarios generados y distribuidos en un programa distinto al presentado en este proyecto. Esta modularidad es posible en gran parte grácias a que la interfaz existente entre ambos programas es el uso de archivos .csv, los cuales son fáciles de generar y modificar.\newline

\begin{figure}[th]
    \centering
    \includegraphics[scale=.74]{Figures/interconexion4modulos.png}
    \decoRule
    \caption[Interconexión de los 4 módulos en un simulador]{Interconexión de los 4 módulos en un simulador}
    \label{fig:interconexion4}
\end{figure}

En el siguiente capítulo se aborda con más detalle la implementación de los dos programas que integran el simulador. \newline

\subsection{Definición de eventos}

El primero de los posibles eventos que el simulador procesa es el nacimiento de un nuevo paquete. Este evento tiene una correspondencia entera con el archivo de eventos (\textit{figura~\ref{fig:archivoeventos}}) puesto que cada linea en ese archivo resulta en un evento de nacimiento en este Simulador. La \textit{figura~\ref{fig:flownacimiento}} muestra la manera en que se procesa el evento, de ella se puede ver que si el dispositivo está disponible para transmitir se transmite un preámbulo (que contiene la información del paquete), el cual sirve para contender por recursos del canal. Finalmente se calendizará un proceso NOMA en el que posiblemente se le asignará un conjunto de recursos y comenzará a transmistir su paquete.\newline

\begin{figure}[th]
    \centering
    \includegraphics[scale=.7]{Figures/flownacimiento.png}
    \decoRule
    \caption[Procesamiento del evento: nacimiento de paquete]{Procesamiento del evento: nacimiento de paquete}
    \label{fig:flownacimiento}
\end{figure}

El siguiente posible evento es el de proceso NOMA. La \textit{figura~\ref{fig:flownoma}} muestra cómo se procesa en el simulador y se puede observar que se realiza el algoritmo NOMA (\textit{algoritmo~\ref{A2}}) con los dispositivos que están haciendo uso del canal y aquellos que quieren transmitir. Los dispositivos que tengan la suerte de contar con un cluster (un conjunto de recursos) comenzará a transmitir y creará o actualizará su evento de fin de transmisión, teniendo en cuenta su nueva tasa de transmisión. Aquellos dispositivos que no consigan recursos cancelarán sus eventos de final de transmisión, y sus paquetes serán considerados en la estadística de bloqueos por falta de recursos. \newline

\begin{figure}[th]
    \centering
    \includegraphics[scale=1]{Figures/flownoma.png}
    \decoRule
    \caption[Procesamiento del evento: proceso noma]{Procesamiento del evento: proceso noma}
    \label{fig:flownoma}
\end{figure}

El último evento se trata del fin de transmisión, en este evento un dispositivo termina de transmitir todo su paquete y un nuevo proceso NOMA es calendarizado para redistribuir los recursos. El dispositivo que recien termina de transmitir vuelve al estado \textit{idle} si no tiene paquetes en la cola, de lo contrario transmite el siguiente. Es aquí también cuando se calcula la tasa efectiva a la que se transmitió el paquete haciendo la resta entre el momento en que se comenzó a transmitir y el tiempo actual de la simulación. Esta tasa efectiva es considerada al calcular el porcentaje de paquetes que fueron transmitidos a las tasas requeridas.\newline

La condición de paro del simuador es que se llegue al tiempo límite de la simulación o que se terminen los eventos en la cola. Cuando pase cualquiera de esas dos cosas la simulación termina y un archivo .csv con \textit{logs} de los eventos es generado utilizando la librería $pandas$. Este archivo es descrito en la siguiente sección. \newline



%%%%%%%%%%%%%%%%%%%%%%%%%%%%%%%%%%%%%%%%%%%%%%%%%%%%%%%%%%%%%%%%%%%%%
\section{Parámetros generales del simulador}\label{parametrossimulador}

Para planear un modelo de sistema válido y coherente se incorporaron algunos parámetros generales de diseño revisados en los artículos \parencite{Shahini2019}, \parencite{Mostafa2019} y \parencite{Gupta2018}. La \textit{Tabla~\ref{tab:ParametrosGral}} describe el conjunto de parámetros de la simulación.\newline

Para la generación de variables aleatorias se usaron las librerías \textit{scipy, numpy ó random} en Python.\newline
\begin{table}
    \caption{Parámetros de la simulación a nivel de sistema}
    \label{tab:ParametrosGral}
    \centering
    \begin{tabular}{|m{6cm}|p{10cm}|} \\ 
    \textbf{Parámetro} & \textbf{Valor} \\ \hline  \hline 
    \textit{Escenario}  & \footnotesize{ UMa } \\ \hline 
    \textit{Ambiente}  & \footnotesize{ LoS } \\ \hline 
    \textit{Diseño celular}  & \footnotesize{ uni-celular } \\ \hline 
    \textit{Transmisión}  & \footnotesize{ UL } \\ \hline 
    \textit{Radio de celula}  & \footnotesize{ 200 m } \\ \hline 
    \textit{Movilidad de UEs}  & \footnotesize{ Nula - 0km/h } \\ \hline 
    \textit{Distribución de UEs } & \footnotesize{ Procesos puntuales de Poisson Homogeneos (HPPP) } \\ \hline 
    \textit{Modelo de canal de propagación `Path Loss' } & \footnotesize{ Modelo CI\newline $L^{CI}_p(f,d)_{\left[dB\right]}=32.4+\ 10\ n{\ log}_{10}\left(\frac{d}{d_0}\right)+{\ 20\ log}_{10}\left(d_0\right)+{20\ log}_{10}\left(f\right)+x^{CI}_{\sigma .}$ } \\ \hline 
    \textit{PLE}  & \footnotesize{ 2 } \\ \hline 
    \textit{$d_0$}  & \footnotesize{ 1m } \\ \hline 
    \textit{Modelo de Tráfico} & \footnotesize{ CMMPP y Periódico} \\ \hline 
    \textit{Esquema de acceso múltiple al medio } & \footnotesize{ NOMA usando agrupamientos } \\ \hline 
    \textit{Relación entre dispositivos mMTC y uRLLC } & \footnotesize{ 3 a 1 } \\ \hline 
    \textit{$K_{max}$ } & \footnotesize{ 1, 2, 3 y 4 } \\ \hline 
    \textit{Número de antenas BS } & \footnotesize{ 1 } \\ \hline 
    \textit{Número de antenas UE } & \footnotesize{ 1 } \\ \hline 
    \textit{Potencia máxima de transmisión de nodos uRLLC } & \footnotesize{ 23 dBm } \\ \hline 
    \textit{Potencia máxima de transmisión de nodos mMTC } & \footnotesize{ 23, 20 o 14 dBm } \\ \hline 
    \textit{Densidad de Ruido Térmico} & \footnotesize{ -174 dbm/Hz } \\ \hline 
    \textit{Frecuencia de operación}  & \footnotesize{ En banda y guarda de banda LTE estándar,\newline Banda de 2GHz } \\ \hline 
    \textit{Ancho de banda del sistema para un PRB (BW) } & \footnotesize{ 180 KHz } \\ \hline 
    \textit{Espacio entre sub-portadoras Uplink}  & \footnotesize{ 3.75kHz \textit{singletone y multitone} \parencite{Shahini2019}} \\ \hline 
    \textit{Tasa de datos máxima} & \footnotesize{ UL: 20kbps} \\  
    \end{tabular}
\end{table}

 
% Chapter 6

\chapter{Implementación} % Main chapter title

\label{Chapter6} % Change X to a consecutive number; for referencing this chapter elsewhere, use \ref{ChapterX}


%----------------------------------------------------------------------------------------
%	SECTION 
%----------------------------------------------------------------------------------------

\section{}

%----------------------------------------------------------------------------------------
%	SECTION 
%----------------------------------------------------------------------------------------

\section{}
\myworries{AQUI VA TODO LO REFERENTE A LO QUE HEMOS HECHO EN PT2}\newline
\myworries{TODO: Faltan palabras en ingles en italicas en este capitulo} 
% Chapter 7

\chapter{Resultados} % Main chapter title

\label{Chapter7} % Change X to a consecutive number; for referencing this chapter elsewhere, use \ref{ChapterX}

El objetivo de este capítulo fue el de brindar resultados para distintos escenarios de interés, con el fin de dar comparaciones analíticas con base en la variación de los parámetros de entrada del simulador.

%----------------------------------------------------------------------------------------
%	SECTION 
%----------------------------------------------------------------------------------------

\section{Escenario I} % Simulación con un solo un TTI - FER
\subsection{Descripción del escenario}
Este escenario se concentró en obtener resultados para un solo intervalo de tiempo de transmisión (TTI), con el fin de analizar el rendimiento del modelo de despliegue de UE y el modelo de canal en conjunto con NOMA y evaluar su rendimiento.
Se simuló el rendimiento de \textit{multitone} con diferentes clases de potencia para los dispositivos MTC y con diferentes tamaños de grupos ($kmax$) pero su desempeño no resultó ser importante ya que resultaba ser similar al de singletone. Por este motivo los resultados con la propuesta multitone no se reportaron. Sin embargo, en estos resultados se implementó un modo de operación hibrido donde se adoptó un modo de operación multitone solamente cuando el número de dispositivos es menor al número de grupos (48), esto con la finalidad de no desperdiciar recursos. Y singletone en los demás casos.
También es importante señalar que la relación entre dispositivos mMTC y uRLLC es de 3 a 1.
\subsection{Parámetros de entrada}
De acuerdo con los parámetros generales del modelo de sistema [véase Tabla~\ref{tab:ParametrosGral}], los criterios considerados para este escenario fueron los siguientes:
\begin{itemize} 
	\item $k_{max} \to $ 1, 2 , 3 y 4 grupos
	\item $p_{m}^{s} \to$ 23, 20, 14 dBm
\end{itemize}

\subsection{Resultados obtenidos}
En primera instancia, en el capítulo anterior se analizó el histograma de las pérdidas de canal, del Modelo CI y también del Modelo de canal que proponen en el artículo \parencite{Shahini2019} , de acuerdo con los histogramas se tiene que el valor promedio de las perdidas [dB] en el Modelo CI son de 80.35 dBs. En contrario con las pérdidas del modelo de canal del artículo, tienen un valor promedio de 74.67 dBs. Es decir nuestro canal tiene 3.7 (5.68 dBs) veces más perdidas en comparación del canal que se implementa en \parencite{Shahini2019}, por lo que se espera que el rendimiento sea menor, esto se puede observar en la Figura\ref{fig: NOMA_comprobacion_CI} donde se observa que en el caso de 192 usuarios el desempeño de la simulación del artículo es aproximadamente 140 usuarios, es decir 73\% de los usuarios alcanzan su tasa objetivo. Por el contrario en la evaluación de la simulación con el modelo de canal CI en NOMA se obtiene que aproximadamente 115 usuarios alcanzan su tasa objetivo, es decir, un 60\%. Hay un impacto del canal CI de aproximadamente 13\% del desempeño en comparación con el canal propuesto en \parencite{Shahini2019}.\newline


\begin{figure}[th]
    \centering
    \includegraphics[scale=.7]{Figures/ResultadosNOMA/NOMA_comprobacion_CI.png}
    \decoRule
    \caption[]{}
    \label{fig:NOMA_comprobacion_CI}
\end{figure}

En la Figura~\ref{ NOMA_evaluacion_K_Pm_Variable_3D } se evaluó el número de usuarios que alcanzaron su tasa objetivo, se realizaron comparaciones con respecto a tres tipos de clase de potencia para los mMTC y un variable número de dispositivos por grupo (kmax).\newline

\begin{figure}[th]
    \centering
    \includegraphics[scale=.7]{Figures/ResultadosNOMA/NOMA_evaluacion_K_Pm_Variable_3D.png}
    \decoRule
    \caption[]{}
    \label{fig:NOMA_evaluacion_K_Pm_Variable_3D}
\end{figure}

Empezando con kmax = 4, se puede observar que entre menor es la potencia de los dispositivos mMTC, el rendimiento de usuarios que alcanzan su tasa objetivo va decayendo y esto es por que al bajar su potencia los dispositivos mMTC varios de ellos comienzan a tener dificultades para alcanzar su tasa objetivo. Si se analiza el caso de 192 usuarios el desempeño con una potencia de usuarios MTC de 23dBm es de aproximadamente 115 usuarios que alcanzan su tasa objetivo, es decir, un 60\%. Y cuando la potencia de usuarios mMTC de 14dBm, 73 dispositivos alcanzan su tasa objetivo, un 38\%. Es decir el rendimiento decrece un 22\% de una potencia de 23 a 14 dBm.\newline

Con kmax=3, vemos que el rendimiento en general decae cuando son 150 usuarios y es porque en este caso, el máximo de usuarios que pueden ser atendidos es de 144, por lo que en los casos de 175 y 192 usuarios el rendimiento va bajando esto es por la relación 3 a 1 que se propuso en los parámetros de entrada. También conforme se baja la potencia de los dispositivos mMTC, se puede ver que el rendimiento decae. Si se analiza el caso de 150 usuarios el desempeño con una potencia de usuarios MTC de 23dBm es de aproximadamente 89 usuarios que alcanzan su tasa objetivo, es decir, un 59\%. Y cuando la potencia de usuarios mMTC de 14dBm, 67 dispositivos alcanzan su tasa objetivo, un 44\%. Es decir el rendimiento decrece un 15\% de una potencia de 23 a 14 dBm.\newline

Con kmax=2, se observa que el rendimiento en general decae cuando son 100 usuarios y es porque en este caso, el máximo de usuarios que pueden ser atendidos es de 96, por lo que en los casos mayores a 100 usuarios el rendimiento va bajando esto igualmente es por la relación 3 a 1. De igual manera también se puede ver que el rendimiento decae cuando se baja la potencia de los dispositivos mMTC. Si se analiza el caso de 100 usuarios el desempeño con una potencia de usuarios MTC de 23dBm es de aproximadamente 53 usuarios que alcanzan su tasa objetivo, es decir, un 53\%. Y cuando la potencia de usuarios mMTC de 14dBm, 49 dispositivos alcanzan su tasa objetivo, un 49\%. Es decir el rendimiento decrece solamente 4\% de una potencia de 23 a 14 dBm.\newline

Con kmax=1, de la misma forma se observa que el rendimiento en general decae cuando son 50 usuarios y es porque en este caso, el máximo de usuarios que pueden ser atendidos es de 48, por lo que en los casos mayores a 100 usuarios el rendimiento va bajando esto igualmente es por la relación 3 a 1. También conforme se baja la potencia de los dispositivos mMTC, se puede ver que el rendimiento no decae de manera significativa como lo fue en los otros casos. 

En las siguientes figuras se evaluo la misma métrica del número de usuarios que alcanzan su tasa objetivo pero ahora desde la perspectiva de cuantos de estos son uRLLC y cuantos mMTC.\newline


\begin{figure}[th]
    \centering
    \includegraphics[scale=.6]{Figures/ResultadosNOMA/Kmax4_DiferentesPM.png}
    \decoRule
    \caption[]{}
    \label{fig:Kmax4_DiferentesPM}
\end{figure}

\begin{figure}[th]
    \centering
    \includegraphics[scale=.6]{Figures/ResultadosNOMA/Kmax4_DiferentesPM_Porcentual.png}
    \decoRule
    \caption[]{}
    \label{fig:Kmax4_DiferentesPM_Porcentual}
\end{figure}


La Figura~\ref{fig:Kmax4_DiferentesPM} representa graficas de barras acerca de la evaluación del número de usuarios mMTC y uRLLC que alcanzaron su tasa objetivo en un TTI, esto con un Kmax = 4 (192 usuarios como máximo), se realizaron comparaciones con diferentes potencias de los dispositivos mMTC. Igualmente, En la Figura~\ref{fig:Kmax4_DiferentesPM_Porcentual} se representan graficas de barras acerca de la evaluación del número de usuarios mMTC y uRLLC que alcanzaron su tasa objetivoI, pero esta vez mostrando el porcentaje de dispositivos mMTC y uRLLC que alcanzan su tasa objetivo, de acuerdo a la relación 3 a 1 que se planteó en los parámetros de entrada.\newline

Primeramente, con una potencia de 23dBm para los dispositivos mMTC (color azul), se observa que entre mayor sea el número de usuarios la correlación entre dispositivos uRLLC y mMTC que alcanzan su tasa se va volviendo más desproporcional, esto se puede ver más claramente en la Figura~\ref{ Kmax4_DiferentesPM_Porcentual}. P. Ej., cuando son 25 usuarios la proporción uRLLC y mMTC es de 51\%- 98\% respectivamente y cuando el número de usuarios aumenta a 192 usuarios la proporción uRLLC y mMTC es de 2\%- 80\% respectivamente (esto es con base en la relación 3 a 1). Como se observa la proporción de uRLLC y mMTC que alcanzan su tasa es muy desproporcional o injusta.\newline

En los casos en donde la potencia de los usuarios mMTC es menor a la de los uRLLC, se observa una mejor proporción entre los tipos de dispositivos. P. Ej. Con una potencia de 14dBm para los dispositivos mMTC (color verde) la proporción de dispositivos uRLLC y mMTC es de 51\%- 64\% respectivamente y cuando el número de usuarios aumenta a 192 usuarios la proporción uRLLC y mMTC es de 16\%- 44\% respectivamente (esto es con base en la relación 3 a 1). Como se observa la proporción de uRLLC y mMTC que alcanzan su tasa es más proporcional comparado con el análisis de potencia de los usuarios mMTC en 23 dBm. \newline


\begin{figure}[th]
    \centering
    \includegraphics[scale=.6]{Figures/ResultadosNOMA/Kmax3_DiferentesPM.png}
    \decoRule
    \caption[]{}
    \label{fig:Kmax3_DiferentesPM}
\end{figure}

\begin{figure}[th]
    \centering
    \includegraphics[scale=.6]{Figures/ResultadosNOMA/Kmax3_DiferentesPM_Porcentual.png}
    \decoRule
    \caption[]{}
    \label{fig:Kmax3_DiferentesPM_Porcentual}
\end{figure}


Las Figuras~\ref{fig:Kmax3_DiferentesPM} y ~\ref{fig:Kmax3_DiferentesPM_Porcentual} representan graficas de barras acerca de la evaluación del número de usuarios mMTC y uRLLC que alcanzaron su tasa objetivo en un TTI, esto con un Kmax = 3 (144 usuarios como máximo), se realizaron comparaciones con diferentes potencias de los dispositivos mMTC.

Primeramente, con una potencia de 23dBm para los dispositivos mMTC (color azul), se observó que de igual manera, entre mayor sea el número de usuarios la correlación entre dispositivos uRLLC y mMTC que alcanzaban su tasa se va volviendo más desproporcional, esto se puede ver más claramente en la Figura~\ref{fig:Kmax3_DiferentesPM_Porcentual}. P. Ej., cuando son 25 usuarios la proporción uRLLC y mMTC es de 52\%- 98\% respectivamente y cuando el número de usuarios aumenta a 144 usuarios la proporción uRLLC y mMTC es de 14\%- 82\% respectivamente (esto es con base en la relación 3 a 1). Como se observa la proporción de uRLLC y mMTC que alcanzan su tasa sigue siendo desproporcional pero no tanto comparado con kmax = 4, esto es porque menos dispositivos son agrupados.\newline

En los casos en donde la potencia de los usuarios mMTC es menor a la de los uRLLC, se observó de la misma manera que estos obtienen una mejor proporción. P. Ej. Con una potencia de 14dBm para los dispositivos mMTC (color verde) la proporción de dispositivos uRLLC y mMTC es de 51\%- 64\% respectivamente y cuando el número de usuarios aumenta a 144 usuarios, la proporción uRLLC y mMTC es de 35\%- 44\% respectivamente (esto es con base en la relación 3 a 1). Como se observa la proporción de uRLLC y mMTC que alcanzan su tasa es más proporcional comparado con el análisis de potencia de los usuarios mMTC en 23 dBm. \newline



\begin{figure}[th]
    \centering
    \includegraphics[scale=.5]{Figures/ResultadosNOMA/Kmax2_DiferentesPM.png}
    \decoRule
    \caption[]{}
    \label{fig:Kmax2_DiferentesPM}
\end{figure}

\begin{figure}[th]
    \centering
    \includegraphics[scale=.5]{Figures/ResultadosNOMA/Kmax2_DiferentesPM_Porcentual.png}
    \decoRule
    \caption[]{}
    \label{fig:Kmax2_DiferentesPM_Porcentual}
\end{figure}

Las Figuras~\ref{fig:Kmax2_DiferentesPM} y ~\ref{fig:Kmax2_DiferentesPM_Porcentual} representan graficas de barras acerca de la evaluación del número de usuarios mMTC y uRLLC que alcanzaron su tasa objetivo en un TTI, esto con un Kmax = 2 (96 usuarios como máximo), se realizaron comparaciones con diferentes potencias de los dispositivos mMTC.\newline

Primeramente, con una potencia de 23dBm para los dispositivos mMTC (color azul), se observó que de igual manera, entre mayor sea el número de usuarios la correlación entre dispositivos uRLLC y mMTC que alcanzaban su tasa se va volviendo más desproporcional, esto se puede ver más claramente en la Figura~\ref{fig:Kmax2_DiferentesPM_Porcentual}. P. Ej., cuando son 25 usuarios la proporción uRLLC y mMTC es de 52\%- 98\% respectivamente y cuando el número de usuarios aumenta a 96 usuarios la proporción uRLLC y mMTC es de 28\%- 85\% respectivamente (esto es con base en la relación 3 a 1). Como se observa la proporción de uRLLC y mMTC que alcanzan su tasa sigue siendo desproporcional pero no tanto comparado con kmax = 3, esto es porque menos dispositivos son agrupados.\newline

En los casos en donde la potencia de los usuarios mMTC es menor a la de los uRLLC, se observó de la misma manera que estos obtienen una mejor proporción. P. Ej. Con una potencia de 14dBm para los dispositivos mMTC (color verde) la proporción de dispositivos uRLLC y mMTC es de 52\%- 64\% respectivamente y cuando el número de usuarios aumenta a 144 usuarios, la proporción uRLLC y mMTC es de 41\%- 47\% respectivamente (esto es con base en la relación 3 a 1). Como se observa la proporción de uRLLC y mMTC que alcanzan su tasa es más proporcional comparado con el análisis de potencia de los usuarios mMTC en 23 dBm. \newline


Para finalizar esta sección, se pudo examinar que de la Figura~\ref{fig:NOMA_evaluacion_K_Pm_Variable_3D}, se concluye que entre menor sea la potencia de los mMTC, el número de dispositivos que alcanzan su tasa objetivo es menor, pero al hacer el análisis de las gráficas de barras enfocadas en cuantos dispositivos uRLLC y mMTC fueron los que alcanzaron su tasa, se observó que la proporción de los tipos de dispositivos mejoraba.


%----------------------------------------------------------------------------------------
%	SECTION 
%----------------------------------------------------------------------------------------

\section{Escenario II} % Simulación completa con tráfico - 

Este segundo escenario pone a prueba la capacidad que tiene el programa principal \textbf{Simulador de modelos de tráfico para nodos IoT en una red celular de 5G}, de generar resultados que permitan comparar el efecto que tiene establecer distintos parámetros de la red en el \textit{throughput} del sistema y en el porcentaje de dispositivos que cumplen su tasa deseada. \newline

\subsection{Descripción del escenario}

En este este escenario se consideraron únicamente los dispoitivos de tipo URLLC y los llamados Otros dispositivos mMTC, esto para facilitar el cumplir con la relación de 3 a 1 entre ambos tipos de dispositivos. Entonces en el programa \textbf{Generador de tráfico IoT} se generó tráfico dentro de una célula de radio igual a 200 metros. La distribución de usuarios se hizo con la opción PPP y se establecieron las intensidades de la siguiente forma: La intensidad de los dispositivos mMTC se fijo en 0.3 $dispositivos/m^2$ y la de los dispositivos URLLC se fijo en 0.1 $dispositivos/m^2$. La relación de dispositivos se eligió 3 a 1 para después fijar las tasas de nacimientos de paquetes y de alarmas idénticas en ambos tipos de servicios y garantizar que el tráfico ofrecido al sistema conserve esa relación. La tasa de nacimiento de paquetes es $\lambda_{normal} = 0.0167 paquetes/seg.$ y la de nacimiento de alarmas es $\lambda_{alarma} = 0.1 alarmas/seg.$ para ambos tipos de dispositivos. Finalmente, las características en las que se transmiten las alarmas son compartidas también entre ambos dispositivos: $velocidad_{alarma} = 500 m/s$ y modelo de propagación espacial = \textit{Rised-cosine window} con $d_{th}=200$ y $d_{th}=100$. \newline

De manera que virtualmente, ambos tipos los dispositivos generan tráfico del mismo tipo, pero al haber 3 veces más dispositivos mMTC, en los algoritmos NOMA se conservará en promedio esta relación.

El tráfico generado por el \textbf{Generador de tráfico IoT} correspondió a 10 segundos y la iteración seleccionada para ser evaluada por el simulador de eventos discretos contenía 2 alarmas, una para cada tipo de dispositivo, lo que es el promedio esperado, dado que $\lambda_{alarma} = 0.1 alarmas/seg.$. Se seleccionó esta iteración por ser una buena representación de los parámetros ingresados. Finalmente se inició la simulación con 48 dispositivos ya utilizando el canal, 26 dispositivos mMTC y 12 URLLC, esto para que el sistema se encontrara ya en un equilibrio de operación.

\subsection{Parámetros de entrada}

La instancia de tráfico utilizada como entrada del simulador de eventos discretos, comprendía $37541$ dispositivos mMTC y $12611$ dispositivos URLLC. El tiempo de la simulación se fijo en $10$ segundos y las potencias máximas de transmisión en los dispositivos se fijaron como $23dBm$ para los URLLC y $20dBm$ para los mMTC. Finalmente se usó $d0=1m$, $PLE=2.0$ y el bloque de frecuencias que inicia en $2Ghz$. 

Se corrieron 4 rutinas en las que se variaron los valores de $k$ desde $1$ hasta $4$.

\subsection{Resultados obtenidos}

Para \textbf{k=1} se obtuvo:\newline
Tráfico ofrecido = $48182.749345 bytes/s$ \newline
Throughput = $31578.300434 bytes/s$ \newline
Tasas no cubiertas URLLC = $25.670367 porciento$ \newline
Tasas no cubiertas URLLC = $17.796712 porciento$ \newline

Para \textbf{k=2} se obtuvo:\newline
Tráfico ofrecido = $48182.749345 bytes/s$ \newline
Throughput = $33391.109178 bytes/s$ \newline
Tasas no cubiertas URLLC = $22.322801 porciento$ \newline
Tasas no cubiertas URLLC = $11.487657 porciento$ \newline

Para \textbf{k=3} se obtuvo:\newline
Tráfico ofrecido = $48182.749345 bytes/s$ \newline
Throughput = $34457.869843 bytes/s$ \newline
Tasas no cubiertas URLLC = $24.560205 porciento$ \newline
Tasas no cubiertas URLLC = $11.829281 porciento$ \newline

Para \textbf{k=4} se obtuvo:\newline
Tráfico ofrecido = $48182.749345 bytes/s$ \newline
Throughput = $35150.834596 bytes/s$ \newline
Tasas no cubiertas URLLC = $24.184458 porciento$ \newline
Tasas no cubiertas URLLC = $11.427156 porciento$ \newline 
% Chapter 8

\chapter{Conclusiones} % Main chapter title

\label{Chapter8} % Change X to a consecutive number; for referencing this chapter elsewhere, use \ref{ChapterX}

En este capítulo se concluye el presente proyecto, compilando una discusión y un análisis crítico de los resultados en general, presentando las posibilidades de evolución futura de las comunicaciones móviles, así como las propuestas de trabajo académico futuro a mediano plazo.\newline

%----------------------------------------------------------------------------------------
%	SECTION 
%----------------------------------------------------------------------------------------

Al analizar todos los elementos que hay en una simulación a nivel de sistema en redes celulares, se llegó a la conclusión que se puede realizar una buena representación modelando principalmente estos cuatro aspectos:

\begin{itemize}
    \item Modelo de despliegue
    \item Modelo de canal
    \item Esquema de acceso múltiple al medio
    \item Modelos de tráfico
\end{itemize}

El análisis individual de los modelos sentó las bases para los resultados que se obtuvieron de manera conjunta, en el análisis por separado se obtuvieron las siguientes conclusiones:

\begin{itemize}
    \item La opción mas realista para el modelado de la localización de usuarios en un plano es por medio de un PPP.
    \item Tras revisar los modelos de canal de distintos escenarios urbanos se concluyó que un modelo de canal idoneo es considerar pérdidas por trayectoria junto con el desvanecimiento rápido.  El modelo CI es un modelo de canal mas realista que predice las pérdidas en un ambiente urbano (UMa). El desvanecimiento tipo Rayleigh, es de suma importancia en anchos de banda pequeños como el de NB-IoT, debido a las variaciones rapidas que pueden ocurrir en el canal.
    \item Los esquemas no ortogonales han tenido un gran auge en la literatura cientifica, por el hecho de que consiguen aumentar la capacidad en los sistemas.
    \item Se llegó a la conclusión de que para el uso de esquemas no ortogonales con dos clases de dispositivos, existe una compensación entre la conectividad de usuarios y la potencia de los usuarios que tienen menos requerimientos de tasas.
    \item Se optó por escoger un modelo de tráfico fuente CMMPP ya que da una buena representación del tráfico espacial y temporal, que lo caracteriza por ser mas realista.
\end{itemize}


Es importante mencionar que este proyecto se hicieron algunas suposiciones que  difieren de la realidad:
\begin{itemize}
    \item No se consideraron los efectos de la interferencia intercelular, la que proviene de otras células. 
    \item Se idealizó la cobertura de las celulas a una circunferencia.
    \item Se tomó la consideración de que en el escenario por cada tres usuarios tipo MTC hay uno con requerimientos de tasas mas altos URLLC.
    \item El modelo de la generación de tráfico tipo máquina omitió los estados de ahorro de energía y \textit{sleep} en los que los dispositivos pueden estar.
    \item No se modeló la movilidad de los dispositivos.
    \item Se obvió la arquitectura del canal de bajada (\textit{downlink}) NB-IoT.
\end{itemize}

Dadas las limitaciones enlistadas previamente, se considera como trabajo a futuro las siguientes líneas de investigación:
\begin{itemize}
    \item NOMA para NBIoT con mejores estrategias en un modo de operación \textit{multitone}
    \item MIMO para esquemas NOMA
    \item Efectos del control de potencia en esquemas NOMA
    \item Análisis de tráfico tipo máquina en conjunto con un tráfico tipo humano.
    \item Una cuestión importante es el consumo de energía de los dispositivos, por lo tanto un acceso eficiente debería ser diseñado para minimizar los altos consumos de energía.
\end{itemize}

%----------------------------------------------------------------------------------------
%	SECTION 
%----------------------------------------------------------------------------------------

 

%----------------------------------------------------------------------------------------
%	THESIS CONTENT - APPENDICES
%----------------------------------------------------------------------------------------

\appendix % Cue to tell LaTeX that the following "chapters" are Appendices

% Include the appendices of the thesis as separate files from the Appendices folder
% Uncomment the lines as you write the Appendices

% Appendix A

\chapter{Distribuciones estadísticas en Telecomunicaciones} % Main appendix title

\label{AppendixA} % For referencing this appendix elsewhere, use \ref{AppendixA}

El objetivo de este capítulo fue revisar las distribuciones de probabilidad mas utilizadas en los sistemas de comunicaciones móviles para caracterizar los fenómenos más importantes en este ámbito, después se describió la implementación y puesta a prueba de la generación de las variables aleatorias utilizadas en el simulador, esto con el fin de brindar fiabilidad en los resultados obtenidos.

%----------------------------------------------------------------------------------------
%	SECTION 1
%----------------------------------------------------------------------------------------

\section{DISTRIBUCIONES DE PROBABILIDAD}

El uso de modelos estadísticos es importante para describir \parencite{Correia2018}:
\begin{itemize}
    \item Llamadas telefónicas y conexiones de datos
    \item Influencia del usuario en el rendimiento de la red
    \item Propagación no guiada en ambientes aleatorios
    \item Movilidad del usuario
\end{itemize}

Comúnmente se utilizan las siguientes distribuciones de probabilidad en telecomunicaciones \parencite{Correia2018}:

\begin{enumerate}
    \item Distribución Uniforme: Es usada para describir la fase de una señal. También, se ha utilizado para simular el despliegue de BSs \parencite{TurjmanSmallCells}.
    \item Distribución Normal (Gaussiana): Es usada para describir fluctuaciones alrededor de un valor medio, p.ej. \textit{shadowing}. Esta distribución no puede ser usada para describir entidades que no pueden ser negativas.
    \item Distribución Log-Normal: Es usada para describir entidades como la potencia de una señal, amplitudes, principalmente el desvanecimiento lento.
    \item Distribución Rayleigh: Es usada para describir el desvanecimiento rápido-intenso.
    \item Distribución Susuki: Describe conjuntamente el desvanecimiento lento y rápido.
    \item Distribución Rice: Es usada para describir el desvanecimiento rápido - no-intenso.
    \item Distribución Exponencial: Es ampliamente usada para describir la duración de diferentes fenómenos, principalmente asociados con el desvanecimiento de señales y las llamadas telefónicas.
    \item Distribución de Bernoulli: Es usada para describir la ocupación de canales de telecomunicaciones.
    \item Distribución binomial: Es usada para describir llamadas telefónicas.
\end{enumerate}

\subsection{Generación de números aleatorios}

La distribución uniforme (también llamada distribución rectangular) es una familia de curvas de dos parámetros que es notable porque tiene una función de distribución de probabilidad constante (PDF) entre sus dos parámetros delimitadores. La distribución uniforme se utiliza en técnicas de generación de números aleatorios, como el método de inversión \parencite{ UniformMatlab}.\newline

Se puede usar la distribución uniforme estándar para generar números aleatorios para cualquier otra distribución continua mediante el método de inversión. El método de inversión se basa en el principio de que las funciones de distribución acumulativa continua (CDFs) varían uniformemente durante el intervalo abierto $(0, 1)$ . Si $u$ es un número aleatorio uniforme en (0, 1) , entonces $x = F^{ -1} ( u )$ genera un número aleatorio $x$ a partir de la distribución continua con la CDF especificada $F$ \parencite{UniformMatlab}.\newline

En teoría de la probabilidad y estadística, hay varias relaciones entre las distribuciones de probabilidad. Estas relaciones se pueden clasificar en los siguientes grupos \parencite{univariateDist}:
\begin{itemize}
    \item Una distribución es un caso especial de otra con un espacio de parámetros más amplio.
    \item Transformaciones (función de una variable aleatoria).
    \item Combinaciones (función de varias variables).
    \item Relaciones de aproximación (límite).
    \item Relaciones compuestas (útiles para la inferencia bayesiana [\textit{Bayesian inference}]).
\end{itemize}

\begin{figure}[th]
    \centering
    \includegraphics[scale=0.63]{Figures/RelacionesProbabilidad}
    \decoRule
    \caption[Relaciones entre algunas de las distribuciones de probabilidad univariadas]{Las relaciones entre algunas de las distribuciones de probabilidad univariadas se ilustran con líneas conectadas, las líneas discontinuas significan relación aproximada. [Fuente: \parencite{univariateDist}}
    \label{fig:relacionesDistribuciones}
\end{figure}

\subsection{Procesos de Poisson}

Los procesos de Poisson son altamente utilizados para representar o modelar fenómenos en las telecomunicaciones, p. ej. la generación de llamadas telefónicas.\newline

Algunas propiedades de los procesos de Poisson son las siguientes \parencite{ PoissonMedium}:

\begin{itemize}
    \item Se compone de una secuencia de variables aleatorias $X1, X2, X3, ... Xk$, de modo que cada variable representa el número de ocurrencias de algún evento, durante un intervalo de tiempo.
    \item Es un proceso estocástico. Cada vez que ejecuta el proceso de Poisson, producirá una secuencia de resultados aleatorios diferentes según alguna distribución de probabilidad.
    \item Es un proceso discreto. Los resultados del proceso de Poisson son el número de ocurrencias de algún evento en el período de tiempo especificado, que sin duda es un número entero, es decir, un número discreto.
    \item Tiene incrementos independientes. Lo que esto significa es que el número de eventos que el proceso predice que ocurrirá en cualquier intervalo dado, es independiente del número en cualquier otro intervalo disjunto.
    \item Las variables constitutivas del proceso de Poisson $X1, X2, X3, ... Xk$ tienen una distribución idéntica.
    \item Las variables constitutivas del proceso de Poisson $X1, X2, X3, ... Xk$ tienen una distribución de Poisson , que viene dada por la Función Masa de Probabilidad (PMF):
\end{itemize}

\begin{equation}
    P_{X}(k)=\frac{e^{-\lambda}*\lambda ^{k}}{k!}
    \label{eqn:Poisson}
\end{equation}

La fórmula anterior nos da la probabilidad de ocurrencia de $k$ eventos en unidad de tiempo, dado que la tasa de ocurrencia promedio es $\lambda$ eventos por unidad de tiempo.

\subsubsection{Modelado de tiempos entre llegadas}

Los procesos de Poisson tienen una subestructura notable. Aunque el número de eventos ocurridos se modela usando una distribución de Poisson discreta, el intervalo de tiempo entre eventos consecutivos se puede modelar usando una distribución exponencial, que es una distribución continua \parencite{PoissonMedium}.\newline

Sean $X1, X2, X3, ... Xi$ variables aleatorias tales que:
\begin{itemize}
    \item $X1$ = el intervalo de tiempo entre el inicio del proceso y el primer evento, es decir, la primera llegada,
    \item $X2$ = el tiempo entre llegadas entre la primera y la segunda llegada,
    \item $X3$ = el tiempo entre llegadas entre la segunda y la tercera llegada , y así sucesivamente.
\end{itemize}
La distribución de la variable aleatoria $Xk$ que representa el tiempo entre llegadas entre la llegada $(k-1) th$ y $(k) th$ es \parencite{PoissonMedium}:
\begin{equation}
    X_{k}=Exponential(\lambda)
    \label{eqn:expon}
\end{equation}
La Función Densidad de Probabilidad (PDF) de la variable aleatoria $X_{k}$ es la siguiente:
\begin{equation}
    P_{X}(t)=\lambda e^{-\lambda t}
    \label{eqn:pdfexpon}
\end{equation}
Y describe la PDF de tiempos entre llegadas en un proceso de Poisson.

\section{IMPLEMENTACIÓN DE DISTRIBUCIONES ESTADÍSTICAS EN TELECOMUNICACIONES}



%% Appendix A

\chapter{Simulación - Geometría celular hexagonal} % Main appendix title

\label{AppendixA} % For referencing this appendix elsewhere, use \ref{AppendixA}

\section{Generación de despliegue Uniforme de usuarios}

\section{Análisis de Geometría Celular un una celda}
%\include{Appendices/AppendixC}

%----------------------------------------------------------------------------------------
%	BIBLIOGRAPHY
%----------------------------------------------------------------------------------------
\printbibliography[heading=bibintoc]

%----------------------------------------------------------------------------------------

\end{document}  
